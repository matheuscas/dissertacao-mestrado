\documentclass[template.tex]{subfiles}
\begin{document}

\xchapter{Conclusão}{}

Essa pesquisa propôs e avaliou um sistema fuzzy automatizado de mineração de opinião para classificar o sentimento geral das opiniões dos documentos. Nossa proposta utiliza o método de Wang-Mendel \cite{wang1992generating} para gerar regras fuzzy baseadas nas características dos documentos mais aptas entre quase 60 características definidas e extraídas nesse trabalho. Nós conseguimos resultados promissores de até 72,4\% de acurácia numa validação cruzada de 10 folds.

Essa pesquisa é possivelmente um dos primeiros trabalhos a aplicar a lógica fuzzy e o método de Wang-Mendel em mineração de opinião, mostrando resultados em bases de dados utilizados em trabalhos relacionados. Além disso, nossos resultados são comparáveis a de outros trabalhos que utilizam técnicas não fuzzy. Ainda, o classificador gerado nessa pesquisa, classifica documentos utilizando regras legíveis para seres humanos, utilizando conjuntos fuzzy simples, como "BAIXA", "ALTO", "POSITIVO" e "NEGATIVO". Essa pesquisa também contribuiu na investigação de características de documentos que podem ser relevantes para descrever e classificar documentos. Duas das características que mais se destacaram na classificação dos documentos foram definidas nesse trabalho.

Como trabalhos futuros, há muitos pontos a serem investigados como melhorias para esta pesquisa. Alguns são elicitados a seguir:

\begin{itemize}
\item Construir um conjunto de advérbios melhor, investigar mais a fundo a influência destes sobre adjetivos e avaliar se impactam nos resultados finais;
\item Melhorar o método de detecção de negação e como lidar melhor com esse fenômeno;
\item Melhorar como os conjuntos fuzzy são modelados para as variáveis de entrada das características dos documentos;
\item Investigar mais características que possam representar e classificar melhor os documentos;
\item Experimentar outros tipos de técnicas de seleção de características, para investigar a influência desses métodos na geração de regras fuzzy;
\item E buscar e experimentar a utilização de outros dicionários de opinião, com o fim de verificar a influência desdes na classificação dos documentos;
\end{itemize}

\ifcsdef{mainfile}{}{\bibliography{../pesquisa}}

\end{document}