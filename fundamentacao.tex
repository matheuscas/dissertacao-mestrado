\documentclass[template.tex]{subfiles}
\begin{document}

\xchapter{Fundamentação Teórica e revisão da literatura}{}

%\begin{itemize}
%\item A área de mineração de opiniões é recente (Pang and Lee, 2008);
%\item Tão recente que ainda existem problemas de terminologias (mineração de opinião, análise de sentimentos, mineração de sentimentos, análise afetiva, dentre outros);
%\item Definição de opinion mining (usar um cara forte para isso, como Lib Bing, Po Pang, Turney, etc.)
%
%\item Definição formal de opinião 'O':
%\begin{itemize}
%\item O = (g, s), onde \emph{g} é o alvo e \emph{s} é o sentimento associados a opinião
%\end{itemize}
%
%\item Níveis de mineração de opinião (cita e depois explica cada um deles e diz qual foi usado e por que. Além de, claro, falar de trabalhos que se encaixam em cada nível)
%\begin{itemize}
%\item Nível de documento;
%\item Nível de sentença;
%\item Nível de entidades e seus aspectos
%\end{itemize}
%
%\item angelo: apresente uma visão dos problemas de pesquisa da área (classificação de polaridade, subjetividade, etc), por ser por exemplificação se não tiver uma referência que faça um review sobre isso
%
%\item angelo: precisa falar sobre a etapa de pré-processamento e transformação, e daí comenter sobre as abordagens baseada em bag-of-words e baseada em features 
%
%\item angelo: comente em algum lugar sobre as abordagens baseada em bag-of-words e baseada em features 
%
%\item Lógica Fuzzy
%\begin{itemize}
%\item Muitas das informações que lidamos são imprecisas e vagas;
%\item A lógica clássica não consegue lidar com esse tipo de informação;
%\item Para isso,Zadeh (1965) propôs a Lógica Nebulosa para lidar com informações vagas e imprecisas
%
%\item Conjuntos Fuzzy
%\begin{itemize}
%\item A lógica nebulosa diz que um elemento pode fazer parte de mais de um conjunto com graus de pertinência para cada um deles (Uma opinião pode ser positiva e negativa ao mesmo tempo, mas com graus de pertinência para cada conjunto fuzzy)
%\item A lógica clássica, por outro lado, determina que um objeto pertençe ou não a um conjunto (Ou uma opinião é positiva ou negativa);
%\end{itemize}
%
%\item Wang-Mendel
%\begin{itemize}
%\item O que é?
%\item Explicar o método
%\end{itemize}
%
%\item Fala de outros algoritmos de classificação e agrupamento? Tem isso na qualificação, mas não sei se é mais pertinente (??????)
%
%\item Métodos de classificação com uso de regras
%\begin{itemize}
%\item CFRM
%\item GFRM
%\end{itemize}
%\end{itemize}
%\item Trabalhos relacionados
%\begin{itemize}
%\item Papers with opinion mining? (Angelo: Acho que alguns principais, mais para contextualizar as abordagens, independente e dependente de domínio, classificação com palavras e com features)
%\item Papers with both? (angelo: Sim, descrever rapidamente os encontrados e fazer contraste com a nossa proposta)
%\end{itemize}
%
%\end{itemize}

A pesquisa em mineração de opinião começou com detecção de subjetividade, com os trabalhos de \cite{carbonell1979subjective}, \cite{wilks1983beliefs} e \cite{wilson2004just}. Essa tarefa envolvia a detecção e separação das sentenças objetivas das subjetivas, que carregam as opiniões e sentimentos atrelados. Com o passar dos anos, começando nos anos 2000, foi que a linha de pesquisa de mineração de opinião alavancou, focando em classificar as opiniões em três categorias: negativo, positivo e neutro. A partir daí muitos trabalhos foram publicados nessa área, mas com diferentes denominações, como análise de sentimentos, mineração de sentimentos, classificação de opiniões, dentre outros. Somente em 2003, no trabalho de \cite{dave2003mining} é que o uso de mineração de opinião foi primeiro usado e, juntamente como análise de sentimento, cunhado por \cite{nasukawa2003sentiment} em 2003, é que o termo passou a ser largamente adotado. No entanto, atualmente, ambos os termos denotam o mesmo campo de pesquisa \cite{bing:2012, pang:2008}. Sendo assim, neste trabalho, ambos os termos serão utilizados alternadamente, mas, com o objetivo de simplificar a leitura e compreensão deste texto, o termo de mineração de opinião será majoritariamente utilizado.

\subsection{Definições e níveis de mineração de opinião}

Mineração de opinião é o campo de estudo que analisa as opiniões, sentimentos, avaliações, atitudes e emoções de pessoas direcionadas a entidades ou alvos, como produtos, serviços, organizações, indivíduos, problemas, eventos, tópicos e seus atributos \cite{bing:2012}. É uma área de pesquisa que vem sendo investigada em três principais níveis de análise: i) nível de análise de documento, ii) sentenças e iii) entidades e seus aspectos. O primeiro nível foca em classificar uma opinião de um documento expressando-a como positiva ou negativa. O segundo nível, o de sentenças, em vez de considerar o sentimento geral de um documento como todo, classifica as opiniões de cada sentença separadamente. E o último nível foca em descobrir todos os alvos existentes em sentenças e documentos e classificar as opiniões direcionadas a eles \cite{bing:2012}.

O nível de análise de documento é também denominado na literatura como uma tarefa de classificação de sentimentos em nível documento, pois considera todo o documento como uma unidade de informação \cite{bing:2012, pang:2008}. É importante salientar que, nesse nível de detalhamento, é assumido que o documento opinativo expressa opiniões direcionadas para somente um único assunto e somente possui um único autor das opiniões. Essa análise é feita normalmente sobre opiniões sobre produtos e serviços, pois cada avaliação, normalmente, foca somente em um único produto ou serviço e é escrito por somente uma pessoa \cite{bing:2012}.

Um grande número de trabalhos foram publicados na literatura sobre mineração de opiniões em nível de documento. \cite{gamon2004sentiment} utilizou opiniões de clientes registradas em pesquisas de opinião como conjunto de dados para mineração de opiniões. Neste trabalho foi utilizado o algoritmo de classificação SVM (\textit{Support Vector Machine}), que produz bons resultados em classificação textual \cite{joachims1998text}. Foram realizados também estudos para identificar quais características textuais eram mais relevantes para o treinamento do SVM e, por conseguinte, para minerar opiniões. \cite{mullen2004sentiment} fizeram um trabalho similar ao de \cite{gamon2004sentiment}, utilizando o algoritmo SVM e investigando as características textuais mais relevantes para melhorar a mineração e classificação das opiniões. Estes trabalhos utilizaram opiniões de filmes do site Epinions.com \footnote{Disponível em: http://www.epinions.com/} e opiniões de publicações da Pitchfork Media \footnote{Disponível em: http://pitchfork.com/}.

A mineração de opinião em nível de sentenças é uma abordagem que aumenta a granularidade da análise e determina se cada sentença de um ou mais documentos expressam opiniões positivas, negativas ou neutras. As definições do problema e da suposição principal deste nível são definidas a seguir \cite{bing:2012}: dada uma sentença "x", deve ser ser determinado quando "x" expressa uma opinião positiva, negativa, neutra ou nenhuma opinião; e dada uma sentença "x", esta deve conter somente uma única opinião de um único autor. Esse nível de análise é bastante utilizado como passo intermediário para o terceiro nível, o nível de entidade e aspectos. Analisando cada sentença individualmente é possível identificar as entidades e quais as opiniões estão sendo direcionadas à elas. 

O trabalho de \cite{Hu:2004} mostra uma sumarização das opiniões de produtos. Nele, sumarizar significa minerar as características dos produtos que possuem opiniões direcionadas a elas e classificar as opiniões como positivas ou negativas. Primeiramente, as sentenças que continham opiniões eram identificadas, utilizando um conjunto de palavras normalmente usado para expressar opiniões. Em seguida, foi definido o sentimento geral (positivo ou negativo) das opiniões, baseando-se no dicionário Wordnet \footnote{Disponível em: http://wordnet.princeton.edu/}. A sentimento geral de cada sentença foi determinado por um algoritmo específico proposto pelo trabalho. Os resultados obtidos foram comparados somente com trabalhos já realizados pelos próprios autores, os quais foram melhorados.Em \cite{kim2004determining} foi realizado um trabalho similar ao de \cite{Hu:2004}. Contudo, os autores das opiniões eram identificados.  A determinação do sentimento geral das opiniões encontradas nas sentenças foi feita da mesma forma como proposto por \cite{Hu:2004}. A diferença está nos algoritmos propostos que definem a orientação final de cada sentença, pois os autores propuseram três algoritmos de classificação e fizeram a análise de desempenho entre eles.

Também denominado nível de entidade e características, o nível de entidade e aspectos é o último nível de análise em mineração de opinião \cite{bing:2012}. Este nível possui duas tarefas principais \cite{bing:2012}: extração dos alvos das opiniões e a classificação das opiniões referentes a esses alvos. A primeira tarefa consiste em extrair os alvos das sentenças. Por exemplo, na sentença "A qualidade de voz desse telefone é muito boa", o alvo é a qualidade da voz e a entidade é o telefone (mais precisamente "este telefone"). A segunda tarefa consiste em classificar - como positivas, negativas ou neutras - as opiniões referentes aos aspectos e das entidades extraídas. No exemplo anterior, a opinião referente ao aspecto "qualidade de voz" da entidade "este telefone" é positiva \cite{bing:2012}.

Assim como feito em \cite{Hu:2004} e \cite{kim2004determining}, o trabalho realizado por \cite{ding2008holistic} também focou em minerar opiniões de produtos e suas características. Todavia, este trabalho propôs uma nova abordagem para minerar opinião. Em vez de utilizar, por exemplo, as técnicas associadas ao uso do Wordnet, este trabalho focou em tratar problemas de sentimento geral das palavras levando em consideração o contexto (outras sentenças, outros documentos, distância da opinião referente alvos, dentre outros); tratar conflitos entre opiniões numa sentença (e.g. opiniões contrárias); e utilizar padrões lingüísticos (e.g. regras de negação, clausulas \textit{but}) para tratar palavras especiais, frases e construções verbais. O resultado do artigo é um sistema, chamado de \textit{Opinion Observer}, que produz resultados melhores que os trabalhos relacionados na literatura, como em \cite{Hu:2004} e  \cite{kim2004determining}.

\subsection{Pré-processamento dos dados}

Em mineração de opinião a preparação dos dados é essencial. Antes de os dados serem transformados, analisados, selecionados e, então, classificados, eles precisam ser preparados para serem usados corretamente no processo de mineração de opinião, pois os dados não estão estruturados (e.g. texto livre) e as opiniões não estão discriminadas, já que se misturam às porções não opinativas do documento. As principais tarefas envolvidas no pré-processamento são: a marcação gramatical das palavras do texto (do inglês, \textit{Part of Speech Tagging}), definição dos n-grams a serem utilizados tokenização das palavras. N-gram é uma seqüência de $n$ itens dada uma outra seqüência de um texto \cite{dave2003mining}. 

A marcação gramatical das palavras do texto é o processo de identificação das classes gramaticais de todos os elementos textuais do documento \cite{brill1995transformation}. Até o presente momento da pesquisa, todos os artigos encontrados executaram essa tarefa, especificando ou não o tipo do marcador gramatical, como, por exemplo, pode ser visto em \cite{pang2002thumbs, turney2002thumbs, wilson2005recognizing, chaovalit2005movie}. O marcador utilizado nessa pesquisa, devido a ser o mais usado nos trabalhos relacionados foi o proposto em \cite{brill1995transformation}.

Depois do texto identificado e marcado é preciso definir quais n-grams serão selecionados para a próxima etapa. Adjetivos são centrais para identificar subjetividade e, por conseguinte, opiniões em textos. Em \cite{hatzivassiloglou2000effects} foi desenvolvido um algoritmo para determinar o sentimento final somente de adjetivos. Neste trabalho foi notado que há relação entre adjetivos e conjunções, como "but", "and", dentre outros. A conjunção "And", por exemplo, mantem a mesma polaridade da opinião expressa pelo adjetivo, enquanto que "but", essa polaridade é invertida, em geral. Utilizando um algoritmo de aprendizado de máquina, esse artigo conseguiu classificar os adjetivos com acurácias entre 78\% e 92\%, dependendo da quantidade de dados de treino disponíveis. Outros trabalhos como os de\cite{wiebe2000learning} também utilizaram somente adjetivos como indicadores de subjetividade e presença de conteúdo opinativo.

\cite{turney2002thumbs}, por sua vez, apontou que adjetivos isolados podem indicar opiniões, mas podem não ser suficientes para determinar o sentimento geral de documentos. Turney considera o contexto como fator determinante, exemplificando que "unpredictable" pode ser uma opinião negativa para automóveis, quando associado a "unpredictable steering" ou positivo quando for direcionado a filmes, quando associado a "unpredictable plot". Assim, Turney expande os adjetivos e acrescenta advérbios, verbos e substantivos associados aos adjetivos e entre eles, extraindo dos textos os chamados bigrams, n-grams compostos por dois elementos textuais. Utilizando \textit{mutual information} (PMI) para medir a força semântica entre duas palavras, técnicas de IR (\textit{Information Retrieval}) para determinar um valor numérico para a polaridade dos bigrams, e calculando a média final das polaridades dos bigrams, \cite{turney2002thumbs} alcançou uma média de acurácia de 74\% entre opiniões sobre automóveis e filmes.

\end{document}