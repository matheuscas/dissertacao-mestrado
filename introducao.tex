\xchapter{Introdução}{}

\begin{itemize}

\item Motivações
\begin{itemize}
\item As opiniões são as principais influenciadoras do comportamento humano (Liu, 2012);
\item Pessoas pedem opiniões a outras pessoas para consumo (mídia, produtos, etc.), posições políticas, religiosas, etc.;
\item Empresas precisam saber as opiniões de seus consumidores;
\item A internet e a web mudaram a forma de comunicação entre pessoas e organizações;
\item Há mais facilidade de acesso a fontes de informações e de disponibilização de opiniões (USAR DADOS PARA EMBASAR ESSA INFORMACAÇÃO);
\end{itemize}

\item Problemas
\begin{itemize}
\item Grande quantidade e diversidade de fontes de opiniões;
\item Diferentes formatos, problemas de sintaxe, gírias, dentro outros;
\item É difícil para uma pessoa comum extrair e resumir opiniões de centenas ou milhares de fontes (USAR EMBASAMENTO);
\item Opiniões são carregadas de sentimentos subjetivos e imprecisos;
\end{itemize}

\item Proposta / Definição
\begin{itemize}
\item É preciso utilizar uma metodologia para lidar com imprecisões e vagueza;
\item Lógica fuzzy (Zadeh, 1965): É uma metodologia da Inteligência Computacional que trata computacionalmente dados imprecisos e vagos;
\item ANGELO: poucos trabalhos sobre a aplicação de lógica fuzzy e/ou sistemas fuzzy foram encontrados, e os encontrados são limitados (tipo de limitação?), demonstrando uma lacuna de pesquisa sobre a aplicação de sistemas fuzzy para mineração de opiniões 
\item Por todos os problemas citadas, associadas as motivações, torna-se evidente a necessidade de sistemas automáticos de mineração de opiniões;
\item Definição de "Opinion Mining";
\item Pela vagueza das opiniões, a Lógica nebulosa pode ser útil para esses sistemas automatizados;
\item PROPOSTA: Propor e avaliar o desempenho de um sistema automatizado de mineração de opiniões baseado na lógica nebulosa;
\item DIFERENCIAL: 
\begin{itemize}
\item Geração de regras fuzzy;
\item Extração e definição de características dos documentos para geração das regras;
\item Apresentação do uso do método de Wang-Mendel no processo de mineração de opinião;
\end{itemize}

\item Perguntas de pesquisa:
\begin{itemize}
\item Como Lógica Nebulosa pode ser utilizada no processo de mineração de opiniões?
\item Quais são os ganhos, caso existam, do uso de lógica nebulosa em mineração de opiniões?
\item ANGELO: mais umas duas questões de pesquisa
\item ANGELO: depois de escrever os resultados e discussão, revisar esta seção 
\end{itemize}

\item Objetivos secundários:
\begin{itemize}
\item Investigar o estado da arte do uso de Lógica Nebulosa em mineração de opiniões;
\item Formalizar o processo de mineração de opiniões e propor uma aplicação da lógica nebulosa;
\item dentificar e compor uma base de dados para avaliação da proposta;
\item Analisar os resultados e compara-los com outros métodos da literatura;
\end{itemize}
\end{itemize}
\end{itemize}