\documentclass[template.tex]{subfiles}
\begin{document}

\xchapter{Resultados obtidos}{}

Nessa seção são descritos e discutidos os experimentos realizados e os resultados obtidos desses experimentos. O objetivo é não só comparar qual cenário obtém melhor \textit{acuracia} ou TPR e TNR, mas também discutir em quais contextos os classificadores produzem melhores ou piores resultados.

\section{Datasets}
\todo[inline]{Pelo que vi nos comentários da seção de metodologia essa seção deve sumir daqui e subir para para metodologia, certo?}
\todo[inline]{Isso mesmo, vai para metodologia, mas cuidado com duplicações de conteúdo lá}

Nós executamos nossos experimentos em três bases de dados. Duas já foram mencionadas anteriormente, que são a base de dados de filmes \cite{pang2004sentimental} e a de diferentes categorias da Amazon \cite{wang2011latent}. Ambas contém 2000 documentos que foram previamente classificadas pelo sentimento geral das opiniões como sendo positivos ou negativos. Além disso, estas bases são ditas equilibradas, pois tem a mesma quantidade de documentos positivos e negativos. 

Para a base da Amazon, a referência para definir o sentimento geral dos documentos foi a quantidade de estrelas recebidas pelo usuário: mais de três estrelas o documentos era considerado positivo e menos que isso, negativo. Os documentos com exatamente 3 estrelas foram removidos de nossa análise por não serem claros quanto ao sentimento geral expresso para serem usados como referência. Além disso, a base da Amazon, originalmente, possui 20000 documentos. Contudo ela está  desbalanceada, onde mais 75\% dos documentos são positivos e o restante negativos. Assim, sorteamos, aleatoriamente, 1000 documentos positivos e 1000 documentos negativos para equiparar ao mesmo volume de dados da base de filmes. Sobre a base de filmes, como já dissemos anteriormente, esta já tinha sido pré classificada por seus autores \cite{pang2004sentimental}.

A terceira base, um conjunto de documentos retirados do site Epinions por \cite{taboada2011lexicon}, é composta por 400 documentos de 8 categorias diferentes: livros, carros, computadores, panelas, hotéis, filmes, música e telefones. Cada categoria possui 50 documentos equilibrados, sendo 25 positivos e 25 negativos, classificados préviamente pelo autor da base. Essa base foi utilizada para verificarmos a efetividade de regras criadas em uma base de dados e usadas em outras, nesse caso, a base da Epinions. 

\section{Resultados experimentais}

Nós focamos em comparar os métodos de seleção de características e de inferência, variando as configurações em diferentes etapas do processo de mineração de opinião, comparando acurácia, TPR e TNR. Nós avaliamos a influência dos algoritmos de seleção de características, os sistemas de inferência fuzzy em si, a quantidade usada de conjuntos fuzzy, a eficiência das regras entre domínios e usadas em outro e as características mais selecionadas entre as bases utilizadas. 

\todo[inline]{usa folds ao inves de dobras, já troquei no parágrafo abaixo e refiz redação também}

Para cada base de dados, o processo de mineração de opinião é idêntico para as etapas de pré-processamento, transformação e extração de características. Aplicando validação cruzada de 10 folds, as 9 partes da base são utilizadas para treinamento são utilizadas para seleção de características, na modelagem dos conjuntos fuzzy e na construção da base de regras fuzzy. A parte restante é utilizada somente para teste, realizando a seleção da mesmas características escolhidas durante o treino e fornecendo os valores destas para o sistema fuzzy realizar a classificação. O mesmo processo é repetido para cada fold e os resultados, para todas as medidas usadas, são a média dos valores obtidos em cada fold de teste. 
%Conseqüentemente, todos os tipos de n-grams combinados com todas as técnicas de transformação descritas nesse trabalho passam pela seleção de características para encontrar quais delas são mais apropriadas para representar os documentos. 

\todo[inline]{falta mais discussão nas seções seguintes, apresente mais informações sobre o que ocorreu, e discuta os resultados a luz do comportamento do sistema}

\subsection{Avaliação dos algoritmos de seleção de características}

Para avaliar os algoritmos de seleção de características, CFS e C45, foi definido o seguinte cenário: 3 conjuntos fuzzy para as variáveis de entrada e o uso do MRFC para ambas bases de dados, filmes e a base mista da Amazon. Assim, deixando os demais parâmetros inalterados, nós podemos avaliar o desempenho dos algoritmos de seleção de características. Além de avaliar acurácia, TPR e TNR, foi avaliada também a quantidade média de características selecionadas para cada algoritmo de seleção. A tabela (\ref{table:movies}) e tabela (\ref{table:amazon}) mostram, respectivamente, os resultados para este cenário nas bases de filmes e da Amazon. 

\begin{table}[!h]
    \begin{tabular}{lll}
    Movies                                      & CFS                                       & c4.5                                  \\ \hline
    TNR                                         & 40.3\% $\pm$ 6.09\%               & 47.1\% $\pm$ 42.67\% \\
    TPR                                         & 64.2\% $\pm$ 30.55\%          & 61.7\% $\pm$ 43.93\% \\
    acuracia                                & 52.25\% $\pm$ 4.92\%           & 54.4\% $\pm$ 1.72\% \\
    Características selecionadas & 4.7 $\pm$ 1.1                             & 1                                     \\
    \end{tabular}
    \caption{Resultados da base de filmes}
    \label{table:movies}
\end{table}

\todo[inline]{a acurácia está péssima, quase jogar uma moeda, porque?}
\todo[inline]{o desvio padrão aqui chamada muita atenção, está muito muito alto, , mas o cfs tem desvio menos de TNR e TPR, mas o c4.5 consegui menor desvio de acurácia, porque?}
\todo[inline]{as caracteristicas selecionadas são sempre as mesmas? quantas diferentes? quais as mais populares? o CFS e c4.5 escolhem alguma caracteristica em comum? coloca no texto}
\todo[inline]{pega cada tabela para discutir o que ela apresenta, TNR, TPR, acurácia, caracteristicas selecionadas, não somente descreva mas tente discutir o que ocorreu, e complemente com informações que ajudam o leitor a saber o que está ocorrendo, como informando algo sobre as caracteristicas selecionadas, as regras geradas, etc}

%SVM Filmes

%---> Avg TPR:  69.1 %
%Standard Deviation:  7.3 %
%---> Avg TNR:  72.1 %
%Standard Deviation:  6.87677249878 %
%---> Avg acuracia:  70.6 %
%Standard Deviation:  3.44818792991 %

%WILCOXON
%two-tailed

%Result 1 - Z-value
%The Z-value is -1.1212. The p-value is 0.26272. The result is not significant at p≤ 0.01.
%
%Result 2 - W-value
%The W-value is 16.5. The critical value of W for N = 10 at p≤ 0.01 is 3. Therefore, the result is not significant at p≤ 0.01.

\begin{table}[!h]
    \begin{tabular}{lll}
    Movies                                              & CFS                               & c4.5                                  \\ \hline
    TNR                                                     & 51.3\% $\pm$ 13.56\%  & 34.0\% $\pm$ 43.26\%  \\
    TPR                                                 & 77.5\% $\pm$ 13.9\%       & 74.5\% $\pm$ 38.82\% \\
    acuracia                                        & 64.4\% $\pm$ 8.12\%       & 54.25\% $\pm$ 2.82\% \\\
    Características selecionadas        & 5.3 $\pm$ 0.64               & 1 $\pm$                                  \\
    \end{tabular}
    \caption{Resultados da base da Amazon}
    \label{table:amazon}
\end{table}

%SVM
%
%---> Avg TPR:  65.8 %
%Standard Deviation:  4.6432747065 %
%---> Avg TNR:  75.5 %
%Standard Deviation:  3.20156211872 %
%---> Avg acuracia:  70.65 %
%Standard Deviation:  2.99207286008 %

%WILCOSOX
%
%Result 1 - Z-value
%The Z-value is -2.3953. The p-value is 0.0164. The result is not significant at p≤ 0.01.
%
%Result 2 - W-value
%The W-value is 4. The critical value of W for N = 10 at p≤ 0.01 is 3. Therefore, the result is not significant at p≤ 0.01.

Em ambas as bases as diferenças entre os resultados não são significativas usando o teste \textit{Wilcoxon signed-rank} para $p <= 0.01$. \todo[inline]{o teste foi para acurácia? tem que dizer qual medida foi utilizada} Isso mostra que mesmo que o CFS utiliza quase 5 vezes mais características em ambas as bases e não consegue produzir resultados significativamente melhores, criando ainda regras mais complexas e de difícil compreensão para seres humanos. Assim, já que o algoritmo c4.5 somente precisou de 2 características, produzindo regras menos complexas e mais claras, para produzir resultados próximos ou iguais ao do CFS, decidiu-se em manter o c4.5 para os próximos cenários de avaliação. 
\todo[inline]{c4.5 precisou de 2 caract? sua tabela indica 1 caracteristica, então seriam caracteristicas diferentes para cada base? ou na mesma base, folds diferentes? apresenta o histograma destas 2 caracteristicas para ilustar}

\todo[inline]{vocÊ consegue justificar porque o classificador estava jogando quase tudo para uma classe? o problema é na features selecionada, na modelagem do conjunto fuzzy, no metodo de inferencia? explica um pouco}

Um outro comportamento comum para ambas as bases foi expressivo desvio padrão de cada medida. Foi possível perceber que a em cada dobra da validação cruzada, um forte tendência em classificar documentos como uma só classe gerando muitas classificações erradas para documentos da classe oposta. A tabela (\ref{table:amazon_folds}) mostra as matrizes de confusão de cada dobra para os resultados produzidos na base da Amazon. 
\todo[inline]{ao invés de matriz de confusão, apresente tnr, tpr e acurácia para cada fold, para ficar consistente com a escolha de medidas}

\begin{table}[!h]
    \begin{tabular}{lll}
    a                           & b                                 & classificado como                                  \\ \hline
    Dobra 0 \\    
    100 (TP)                    &0 (FN)                     & a = positivo(100) \\
    92 (FP)                 &8 (TN)                     & b = negativo (100) \\
    &&\\
    Dobra 1    \\
    22 (TP)                 &78 (FN)                    & a = positivo(100) \\
    0 (FP)                  &100 (TN)               & b = negativo (100) \\   
    &&\\ 
    Dobra 2    \\
    100 (TP)                    &0 (FN)                     & a = positivo(100) \\
    95 (FP)                 &5 (TN)                     & b = negativo (100) \\   
    &&\\
    Dobra 3    \\
    100 (TP)                    &0 (FN)                     & a = positivo(100) \\
    95 (FP)                 &5 (TN)                     & b = negativo (100) \\   
    &&\\
    Dobra 4    \\
    12 (TP)                 &88 (FN)                    & a = positivo(100) \\ 
    0 (FP)                  &100 (TN)                   & b = negativo (100) \\
    &&\\
    Dobra 5    \\
    99 (TP)                 &1 (FN)                     & a = positivo(100) \\ 
    96 (FP)                 &4 (TN)                     & b = negativo (100) \\
    &&\\
    Dobra 6    \\
    100 (TP)                    &0 (FN)                     & a = positivo(100) \\ 
    99 (FP)                 &1 (TN)                 & b = negativo (100) \\
    &&\\
    Dobra 7    \\
    12 (TP)                 &88 (FN)                    & a = positivo(100) \\ 
    0 (FP)                  &100 (TN)                   & b = negativo (100) \\
    &&\\
    Dobra 8    \\
    100 (TP)                    &0 (FN)                     & a = positivo(100) \\ 
    92 (FP)                 &8 (TN)                     & b = negativo (100) \\
    &&\\
    Dobra 9    \\
    100 (TP)                    &0 (FN)                     & a = positivo(100) \\ 
    91 (FP)                 &9 (TN)                     & b = negativo (100) \\
    &&\\
    \end{tabular}
    \caption{Resultados das dobras da base da Amazon}
    \label{table:amazon_folds}
\end{table}

É possível perceber que há alternância entre as dobras para as extremidades da classificação dos documentos: ou são todos (ou quase) positivos ou quase todos negativos. E ainda que um novo embaralhamento dos dados seja feito e uma nova seleção de características, o efeito se repete e os resultados também. 

\todo[inline]{ressalte que os folds foram definidos por amostragem estratificada, mantendo a mesma proporção de documentos positivos e negativos do dataset total, e assim cada fold também está balanceado, isso ajuda a mostrar que um eventual enviesamento dos folds não foi a causa deste comportamento}

\subsubsection{Avaliação do impacto da altura da árvore de decisão do algoritmo de seleção de características do c4.5}
\todo[inline]{acho melhor não fazer uma seção separada para arvore de altura  3, sugiro incorporar a seção anterior e incluir mais uma coluna nas tabelas, ficando c4.5 altura 1 e c4.5 altura 3}

É importante ressaltar que os resultados anteriores, para o c4.5, foram obtidos com o algoritmo otimizado, utilizando a ferramenta Weka [CITE], para que a árvore do algoritmo de seleção de características do algoritmo tivesse altura 1, para fins de simplificação das regras geradas. Dessa forma, decidimos avaliar o quanto o aumento da altura da árvore de decisão do algoritmo de seleção de características do c4.5 influenciaria na classificação dos documentos. Definimos que o limite da altura da árvore de decisão fosse 3 para não aumentar demais a complexidade das regras que seriam geradas. A tabela (\ref{table:movies_h3}) e tabela (\ref{table:amazon_h3}) mostram, respectivamente, os resultados para este cenário nas bases de filmes e da Amazon.

\todo[inline]{o que seria 'algoritmo otimizado'? não devemos 'forçar' a arvore a ficar com altura 3, devemos dizer que utilizamos as características  selecionadas foram aquelas existentes até altura 3, porque se a árvore crescer além disso, é só ignorar os demais nós}

 \begin{table}[!h]
    \begin{tabular}{lll}
    Movies                                            & c4.5 - Altura 3                                          & c4.5 - Altura 1                               \\ \hline
    TNR                                               & 55.6\% $\pm$ 43.49\%                            & 47.1\% $\pm$ 42.67\% \\
    TPR                                               & 53.4\% $\pm$ 44.50\%                            & 61.7\% $\pm$ 43.93\% \\
    acuracia                                      & 54.5\% $\pm$ 1.76\%                             & 54.4\% $\pm$ 1.72\% \\
    Características selecionadas      & 1.1 $\pm$ 0.3                                           & 1                                     \\
    \end{tabular}
    \caption{Resultados da base de filmes}
    \label{table:movies_h3}
\end{table}

%Result 1 - Z-value
%
%The Z-value is -1.7838. The p-value is 0.07508. The result is not significant at p≤ 0.01.
%
%Result 2 - W-value
%
%The W-value is 10. The critical value of W for N = 10 at p≤ 0.01 is 3. Therefore, the result is not significant at p≤ 0.01.

\begin{table}[!h]
    \begin{tabular}{lll}
    Movies                                              & c4.5 - Altura 3                               & c4.5 - Altura 1                              \\ \hline
    TNR                                                     & 73.7\% $\pm$ 35.86\%                  & 34.0\% $\pm$ 43.26\%  \\
    TPR                                                 & 47.1\% $\pm$ 34.22\%                  & 74.5\% $\pm$ 38.82\% \\
    acuracia                                        & 60.4\% $\pm$ 5.75\%                       & 54.25\% $\pm$ 2.82\% \\
    Características selecionadas        & 1.9 $\pm$ 0.7                                 & 1 $\pm$                                  \\
    \end{tabular}
    \caption{Resultados da base da Amazon}
    \label{table:amazon_h3}
\end{table}

\todo[inline]{nas tabelas que mostram resultado do c4.5 altura 3, indicam um valor baixo da quant media de caract selecionadas, a mesma caracteristica está sendo selecionada por vários nós? diga quais são e discuta no texto esse comportamento, isso pode indicar que uma característica é repeditamente relevante em vários splits dos dados (cada nó da árvore de decisão, divide a base para os nós seguintes)}

%Result 1 - Z-value
%
%The Z-value is -2.1915. The p-value is 0.02852. The result is not significant at p≤ 0.01.
%
%Result 2 - W-value
%
%The W-value is 6. The critical value of W for N = 10 at p≤ 0.01 is 3. Therefore, the result is not significant at p≤ 0.01.

Mais uma vez, os resultados apresentados em ambas as bases para o aumento da altura da árvore de decisão do algoritmo de seleção de características não produziram resultados significativamente melhores para $p <= 0.01$ no teste \textit{Wilcoxon signed-rank}. \todo[inline]{sempre precisa dizer qual medida está sendo avalida pelo testes estatístico, reveja aqui e no restante do texto} Assim, o crescimento da árvore, que resulta no aumento da complexidade das regras em relação a altura 1, não contribui para o aumento de performance do nosso classificador e na simplificação das regras geradas para a classificação. 

\subsection{Avaliação dos sistemas de inferência}

Nessa subseção é avaliado o desempenho dos sistemas de inferência escolhidos, o Método Geral do Raciocínio Fuzzy (MRFG) e o Método Clássico do Raciocínio Fuzzy (MRFC). Da mesma maneira que foi feita na seção anterior, nós fixamos os demais parâmetros do experimento para melhor avaliar os sistemas de inferência, mantendo o algoritmo de seleção de característica c4.5 (com altura 1, para continuar buscando a geração de regras claras e de fácil leitura para humanos) e 3 conjuntos fuzzy nas variáveis de entrada. A tabela (\ref{table:movies2}) e tabela (\ref{table:amazon2}) mostram, respectivamente, os resultados para este cenário nas bases de filmes e da Amazon.

\begin{table}[!h]
    \begin{tabular}{lll}
    ~                   & CFRM                              & GFRM \\ \hline
    TNR                 & 47.1\% $\pm$ 42.67\%   & 53.8\% $\pm$ 34.96\%    \\
    TPR             & 61.7\% $\pm$ 43.93\%   & 64.6\% $\pm$ 37.08\%   \\
    acuracia        & 54.4\% $\pm$ 1.72\%       & 59.2\% $\pm$ 1.83\%    \\
    \end{tabular}
    \caption{Resultados dos sistemas de inferência na base de filmes}
    \label{table:movies2}
\end{table}

%Result 1 - Z-value
%
%The Z-value is -2.8031. The p-value is 0.00512. The result is significant at p≤ 0.01.
%
%Result 2 - W-value
%
%The W-value is 0. The critical value of W for N = 10 at p≤ 0.01 is 3. Therefore, the result is significant at p≤ 0.01.

\begin{table}[!h]
    \begin{tabular}{lll}
    ~                   & CFRM                                  & GFRM \\ \hline
    TNR                 & 34.0\% $\pm$ 43.26\%      & 44.6\% $\pm$ 35.73\%    \\
    TPR             & 74.5\% $\pm$ 38.82\%      & 75.5\% $\pm$ 34.80\%    \\
    acuracia        & 54.25\% $\pm$ 2.82\%      & 60.05\% $\pm$ 2.37\%   \\
    \end{tabular}
    \caption{Resultados dos sistemas de inferência na base da Amazon}
    \label{table:amazon2}
\end{table}

%Result 1 - Z-value
%
%The Z-value is -2.8031. The p-value is 0.00512. The result is significant at p≤ 0.01.
%
%Result 2 - W-value
%
%The W-value is 0. The critical value of W for N = 10 at p≤ 0.01 is 3. Therefore, the result is significant at p≤ 0.01.

Os resultados mostram que Método Geral do Raciocínio Fuzzy (MRFG) aumenta o \textit{acuracia} sobre o MRFC, em ambas bases,  mantendo a seleção de características e a quantidade de conjuntos fuzzy inalterados. O resultado do teste \textit{Wilcoxon signed-rank} confirma a melhora significativa do MRFG sobre o MRFC, para $p <= 0.01$. Assim, nessa tarefa de classificação de somente duas classes, positivo e negativo, os resultados mostraram que uma melhor abordagem é considerar todas as regras de uma classe, em vez de uma única com maior grau. Daí, o MRFG foi a nossa escolha para prosseguir nos próximos experimentos com o fim de alcançar melhores resultados nesse trabalho.

\subsection{Avaliação de uso de pesos nas regras dos sistemas de inferência}

Em \cite{ishibuchi2001effect} foi mostrado que é possível aumentar a performance da classificação de regras fuzzy IF-THEN, aplicando pesos à elas, além do grau de compatibilidade das regras. No referido artigo, os autores descreveram o processo em que é possível melhorar o desempenho da classificação sem alterar os conjuntos fuzzy das variáveis de saída e de entrada. Baseando-se neste artigo, este trabalho calculou os pesos como se segue. Para cada regra da base de regras gerada, foi calculado o grau de compatibilidade com todos documentos do conjunto de teste. Se o documento fosse positivo, o grau de compatibilidade era acumulado em $\beta_{positivo}$; se o documento fosse negativo, o grau de compatibilidade era acumulado em $\beta_{negativo}$. Ao fim desse processo, caso ambos os betas fosse iguais, não haveria peso a ser considerado, já que as regras tem igual influência sobre o conjunto de dados que elas foram geradas. De outra forma, o peso da regra era definido pela equação \ref{eq:pesos}.

\begin{equation}
P_j = |\beta_{positivo} - \beta_{negativo}| / (\beta_{positivo} + \beta_{negativo})
\label{eq:pesos}
\end{equation}

onde $P_j$ é o peso da regra, para $0 <= P_j <= 1$. 

\todo[inline]{discuta qual é idéia por trás do uso de pesos, valorizar regras que de fato auxiliar a distinguir as duas classes e desvalorizar aquelas que não conseguem separar as classes}

Uma vez que todas as regras tiveram seus pesos calculados, estes são adicionados ao processo de classificação de ambos os métodos utilizados até aqui, o MRFG e o MRFC. O peso torna-se um fator multiplicador do grau de cada regra. Assim, o MRFC em vez de somente levar em consideração a regra com maior grau de compatibilidade com o documento de teste, vai, agora, levar em consideração a regra com maior grau multiplicado pelo peso da regra. O mesmo acontece para o MRFG ao considerar o grau médio entre as classes positivo e negativo. A tabela (\ref{table:movies2_pesos}) e tabela (\ref{table:amazon2_pesos}) mostram, respectivamente, os resultados para o uso dos pesos nas bases de filmes e da Amazon usando os parâmetros até agora estabelecidos. 

\todo[inline]{discuta mais esses resultados, há uma grande mudança qualitativa dos resultados com uso de pesos, principalmente porque o desvio padrão cai muito, quais foram as regras que ganharam pesos e quais perderam peso? quais os valores de pesos encontrados, são próximos dos valores extremos, 0 e 1? como isso ajuda a explicar os resultados ruins sem peso? discuta o desvio padrão também }

\todo[inline]{as tabelas são para classico ou geral?}
\begin{table}[!h]
    \begin{tabular}{lll}
    ~                   & S/ Pesos                              & C/ Pesos \\ \hline
    TNR                 & 53.8\% $\pm$ 34.96\%      & 70.4\% $\pm$ 7.11\%    \\
    TPR             & 64.6\% $\pm$ 37.08\%      & 69.7\% $\pm$ 9.81\%   \\   
    acuracia        & 59.2\% $\pm$ 1.83\%       & 70.05\% $\pm$ 4.00\%    \\
    \end{tabular}
    \caption{Resultados dos sistemas de inferência na base de filmes utilizando pesos nas regras}
    \label{table:movies2_pesos}
\end{table}

%Result 1 - Z-value
%
%The Z-value is -2.8031. The p-value is 0.00512. The result is significant at p≤ 0.05.
%
%Result 2 - W-value
%
%The W-value is 0. The critical value of W for N = 10 at p≤ 0.05 is 8. Therefore, the result is significant at p≤ 0.05.

\begin{table}[!h]
    \begin{tabular}{lll}
    ~                   & S/ Pesos                                  & C / Pesos \\ \hline
    TNR                 & 44.6\% $\pm$ 35.73\%      & 76.80\% $\pm$ 4.57\%    \\
    TPR             & 75.5\% $\pm$ 34.80\%      & 64.9\% $\pm$ 5.50\%    \\
    acuracia        & 60.05\% $\pm$ 2.37\%      & 70.85\% $\pm$ 3.09\%   \\
    \end{tabular}
    \caption{Resultados dos sistemas de inferência na base da Amazon utilizando pesos nas regras}
    \label{table:amazon2_pesos}
\end{table}

%Result 1 - Z-value
%
%The Z-value is -2.8031. The p-value is 0.00512. The result is significant at p≤ 0.05.
%
%Result 2 - W-value
%
%The W-value is 0. The critical value of W for N = 10 at p≤ 0.05 is 8. Therefore, the result is significant at p≤ 0.05.

Os resultados corroboraram as conclusões de \cite{ishibuchi2001effect} que é possível melhorar o desempenho da classificação sem alterar os conjuntos fuzzy das variáveis de saída e de entrada apenas aplicando pesos às regras geradas. O resultado do teste \textit{Wilcoxon signed-rank} também confirma a melhora significativa do MRFG usando pesos nas regras em vez de somente o grau de compatibilidade, para $p <= 0.01$.

\subsection{Avaliação da quantidade de conjuntos fuzzy usados}

Através das últimas seções, foram mostrados resultados utilizando 3 conjuntos fuzzy para modelar nossas variáveis lingüísticas. Seguindo a nossa decisão de usar o c4.5 com altura máxima da árvore de decisão para 1 para reduzir a complexidade das regras geradas e torna-las mais legíveis para seres humanos, nós tentamos reduzir a quantidade de conjuntos fuzzy, usando somente os conjuntos "Baixo" e "Alto", para as variáveis de entrada. Esse experimento tem por fim verificar se há aumento da performance da classificação com regras mais simples e gerais. A tabela (\ref{table:movies2_pesos_2f}) e tabela (\ref{table:amazon2_pesos_2fs}) mostram, respectivamente, os resultados para o uso de somente dois conjuntos fuzzy nas variáveis de entrada nas bases de filmes e da Amazon.

\begin{table}[!h]
    \begin{tabular}{lll}
    ~                   & 3 conjuntos fuzzy                             & 2 conjuntos fuzzy \\ \hline
    TNR                 & 70.4\% $\pm$ 7.11\%                   & 71.2\% $\pm$ 4.33\%    \\
    TPR             & 69.7\% $\pm$ 9.81\%                   & 70.6\% $\pm$ 3.35\%   \\
    acuracia        & 70.05\% $\pm$ 4.00\%              & 70.9\% $\pm$ 3.07\%    \\
    \end{tabular}
    \caption{Resultados dos sistemas de inferência na base de filmes utilizando 2 conjuntos fuzzy na entrada}
    \label{table:movies2_pesos_2fs}
\end{table}

\todo[inline]{parece haver um redução no desvio padrão com uso de 2 conjuntos, porque será que reduziu?}

%Result 1 - Z-value
%
%The Z-value is -1.2741. The p-value is 0.20408. The result is not significant at p≤ 0.05.
%
%Result 2 - W-value
%
%The W-value is 15. The critical value of W for N = 10 at p≤ 0.05 is 8. Therefore, the result is not significant at p≤ 0.05.

\begin{table}[!h]
    \begin{tabular}{lll}
    ~                   & 3 conjuntos fuzzy                             & 2 conjuntos fuzzy \\ \hline
    TNR                 & 76.80\% $\pm$ 4.57\%              & 77.2\% $\pm$ 3.54\%    \\
    TPR             & 64.9\% $\pm$ 5.50\%                   & 64.4\% $\pm$ 3.46\%   \\
    acuracia        & 70.85\% $\pm$ 3.09\%              & 70.8\% $\pm$ 3.27\%    \\
    \end{tabular}
    \caption{Resultados dos sistemas de inferência na base da Amazon utilizando 2 conjuntos fuzzy na entrada}
    \label{table:amazon2_pesos_2fs}
\end{table}

%Result 1 - Z-value
%
%The Z-value is -0.3568. The p-value is 0.71884. The result is not significant at p≤ 0.01.
%
%Result 2 - W-value
%
%The W-value is 24. The critical value of W for N = 10 at p≤ 0.01 is 3. Therefore, the result is not significant at p≤ 0.01.

A tabela (\ref{table:movies2_pesos_2fs}) mostra pequena melhora na base de filmes e empate técnico na base da Amazon. Mas, estatisticamente, os resultados não são significativamente diferentes para $p <= 1$. Todavia, a redução da quantidade dos conjuntos fuzzy para as variáveis de entrada produz regras mais simples e legíveis para serem humanos. Assim, nós temos regras menos complexas com a mesma performance de regras com maior número de antecedentes. 

%positive_to_negative_ratio_of_adjectives_sum_and_bigrams_with_adjectives
%positive_to_negative_ratio_of_unigrams_and_bigrams_sum
Em ambos conjuntos de dados somente duas características foram selecionadas pelo c4.5:
\begin{itemize}
\item Diferença entre a soma positiva e negativa dos adjetivos e dos bigrams formados por advérbio e adjetivo
\item Diferença entre a soma positiva e negativa dos unigrams e bigrams
\end{itemize}

Na base de filmes as duas foram necessárias para atingir os resultados mostrados na tabela ((\ref{table:movies2_pesos_2fs})). Isso remete ao fato de documentos de filmes serem mais difíceis de ser classificados, precisando de mais características para caracterizar corretamente as opiniões \cite{turney2002thumbs, pang2004sentimental, chaovalit2005movie, ohana2009sentiment}. Já na base da Amazon, somente a segunda característica foi utilizada para classificar os documentos. Embora o algoritmo de seleção de características do c4.5 tenha decidido que somente essa característica seja suficiente, não é difícil concluir que, sendo a base da Amazon bastante diversa (e que também inclui filmes), mais características podem ser necessárias para caracterizar melhor documentos tão variados.
Todavia, nós conseguimos classificar um pouco mais de 70\% dos documentos de filmes e da Amazon com poucas regras simples, legíveis para humanos, geradas pelo método de Wang-Mendel:

\begin{itemize}
\item IF a \textit{diferença entre a soma positiva e negativa dos unigrams e bigrams} is ALTO then POLARIDADE is POSITIVA
\item IF a \textit{diferença entre a soma positiva e negativa dos unigrams e bigrams} is BAIXO then POLARIDADE is NEGATIVA
\item IF a \textit{diferença entre a soma positiva e negativa dos adjetivos e dos bigrams formados por advérbio e adjetivo} is ALTO then POLARIDADE is POSITIVA
\item IF a \textit{diferença entre a soma positiva e negativa dos adjetivos e dos bigrams formados por advérbio e adjetivo} is BAIXO then POLARIDADE is NEGATIVA
\end{itemize}

As quatro regras foram utilizadas na base de filmes e as duas primeiras regras na base da Amazon.

\subsection{Avaliação do uso de regras entre domínios}
\todo[inline]{faz aplicação cruzada tambem, amazon na de filmes e vice versa}
\todo[inline]{valorize mais esta seção, o resultado é MUITO interessante, mas o texto não valoriza tanto, compare com os resultados de epinions com regras amazon e filmes. Quando aplicar amazon em filmes e vice versa compare com os resultados obtidos usando as regras da própria base. O uso do classificador de uma base em outra pode indicar a existencia de regras universais e realmente independentes de domínio, mas pode haver necessidade de ajuste na modelagem dos conjuntos ou em outro aspecto talvez}

Outra avaliação que fizemos foi a validação do uso de regras entre domínios, utilizando a base inteira da Epinions como teste. Foram usadas as regras geradas da base de filmes e da base da Amazon e aplicadas à base da Epinions. \todo[inline]{foram usadas as regras geradas para qual fold amazon ou filmes?} Para isso, a base da Epinions passou por pré-processamento, transformação e extração de características, para que as regras pudessem avaliar corretamente os documentos. É importante frisar que nenhuma adaptação foi feita às regras para que pudesse funcionar corretamente. Da maneira que foram geradas na bases de filmes e Amazon, foram utilizadas na classificação dos documentos da base da Epinions. A tabela \ref{table:epinions} mostra os resultados obtidos.
\todo[inline]{nenhuma mudança foi feita nas regras fuzzy, mas nenhuma foi feita também nas características selecionadas e na modelagem dos conjuntos, então reforça isso tambem}

\begin{table}[!h]
    \begin{tabular}{lll}
    ~               & Regras da Amazon                  & Regras dos Filmes \\ \hline
    TNR             & 47.05\% $\pm$ 0.35\%           & 61.7\% $\pm$ 1.41\%    \\
    TPR         & 90.5\% $\pm$ 1.11\%               & 85.8\% $\pm$ 1.22\%   \\
    acuracia    & 68.77\% $\pm$ .0.17\%             & 73.75\% $\pm$ 0.27\%    \\
    \end{tabular}
    \caption{Resultados da aplicação de regras da base de filmes e Amazon na base Epinions}
    \label{table:epinions}
\end{table}

Os resultados mostram que as regras geradas podem ser usadas entre domínios diferentes, produzindo resultados próximos ou melhores que os produzidos nas próprias bases, como pode ser visto na tabela \ref{table:epinions}.

\subsection{Outros resultados}

Cinquenta e sete características foram extraídas dos documentos para realizar os experimentos dessa pesquisa. Destas, duas foram constantemente selecionadas pelos algoritmos de seleção de características. Foram elas:
\todo[inline]{quantifica 'constantemente selecionadas', por exemplo em todas as configurações avaliadas elas sempre foram selecionadas? só com c4.5? }
\begin{itemize}
\item Diferença entre a soma positiva e negativa dos adjetivos e dos bigrams formados por advérbio e adjetivo
\item Diferença entre a soma positiva e negativa dos unigrams e bigrams
\end{itemize}

Essas características englobam unigrams e bigrams formados por advérbios e adjetivos. Esses resultados corroboram quase que completamente os n-grams propostos por \cite{turney2002thumbs}, formados em sua maioria por advérbios e adjetivos; reiteram também a importância central dos adjetivos na mineração e classificação de opiniões, como apontado por \cite{voll2007not}; além de reforçar o proposto por \cite{benamara2007sentiment} que advérbios tema significativa influência como modificadores de intensidade dos adjetivos.

Também comparamos a configuração final do nosso classificador com um método clássico de aprendizado de máquina muito utilizado na tarefa de mineração de opinião, o \textit{Support Vector Machine} (SVM), e que geralmente produz bons resultados, como ser visto em \cite{moraes2012document}, \cite{pang2002thumbs}, \cite{pang2004sentimental} e \cite{wilson2004just}. 

\begin{table}[!h]
    \begin{tabular}{lll}
    ~                   & Wang-Mendel 2 conjuntos fuzzy                     & SVM \\ \hline
    TNR                 & 71.2\% $\pm$ 4.33\%                                           & 72.9\% $\pm$ 5.80\%    \\
    TPR             & 70.6\% $\pm$ 3.35\%                                       & 68.6\% $\pm$ 7.04\%   \\
    acuracia        & 70.9\% $\pm$ 3.07\%                                       & 70.75\% $\pm$ 3.70\%    \\
    \end{tabular}
    \caption{Comparação entre os resultados do método de Wang-Mendel e SVM na base de filmes}
    \label{table:movies_svm}
\end{table}

%Result 1 - Z-value
%
%The Z-value is -0.5606. The p-value is 0.57548. The result is not significant at p≤ 0.01.
%
%Result 2 - W-value
%
%The W-value is 22. The critical value of W for N = 10 at p≤ 0.01 is 3. Therefore, the result is not significant at p≤ 0.01.

\begin{table}[!h]
    \begin{tabular}{lll}
    ~                       & Wang-Mendel 2 conjuntos fuzzy                             & SVM \\ \hline
    TNR                     & 77.2\% $\pm$ 3.54\%                                               & 75.5\% $\pm$ 3.20\%    \\
    TPR                 & 64.4\% $\pm$ 3.46\%                                               & 65.8\% $\pm$ 4.64\%   \\
    acuracia           & 70.8\% $\pm$ 3.27\%                                            & 70.65\% $\pm$ 2.99\%    \\
    \end{tabular}
    \caption{Comparação entre os resultados do método de Wang-Mendel e SVM na base da Amazon}
    \label{table:amazon_svm}
\end{table}

%Result 1 - Z-value
%
%The Z-value is -1.2232. The p-value is 0.22246. The result is not significant at p≤ 0.01.
%
%Result 2 - W-value
%
%The W-value is 15.5. The critical value of W for N = 10 at p≤ 0.01 is 3. Therefore, the result is not significant at p≤ 0.01.

Nossos resultados mostram que o classificador construído nesse trabalho equivale aos resultados do SVM, pois em ambas as bases as diferenças dos resultados não são significativas para $p <= 0.01$. Contudo, nosso classificador possui regras legíveis para seres humanos, é independente de domínio e não precisa de nova rodada de treino para diferentes bases, diferentemente do SVM.

\todo[inline]{falta uma seção final de discussão geral dos resultados para fechar esse capítulo, revisando os achados e fazendo uma avaliação geral de tudo}

\end{document}