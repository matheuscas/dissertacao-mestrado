\xchapter{Resultados obtidos}{}

Nessa seção são descritos e discutidos os experimentos realizados e os resultados obtidos desses experimentos. O objetivo principal não é comparar qual cenário obtém melhor \textit{accuracy} ou \textit{F1}, mas também discutir em quais contextos os classificadores produzem melhores ou piores resultados.

\section{Datasets}

Nós executamos nossos experimentos em três bases de dados. Duas já foram mencionadas anteriormente, que são a base de dados de filmes \cite{pang2004sentimental} e a de diferentes categorias da Amazon \cite{wang2011latent}. Ambas contém 2000 documentos que foram previamente classificadas pelo sentimento geral das opiniões como sendo positivos ou negativos. Além disso, estas bases são ditas equilibradas, pois tem a mesma quantidade de documentos positivos e negativos. 

Para a base da Amazon, a referência para definir o sentimento geral dos documentos foi a quantidade de estrelas recebidas pelo usuário: mais de três estrelas o documentos era considerado positivo e menos que isso, negativo. Os documentos com exatamente 3 estrelas foram removidos de nossa análise por não serem claros quanto ao sentimento geral expresso para serem usados como referência. Além disso, a base da Amazon, originalmente, possui 20000 documentos. Contudo ela está  desbalanceada, onde mais 75\% dos documentos são positivos e o restante negativos. Assim, sorteamos, aleatoriamente, 1000 documentos positivos e 1000 documentos negativos para equiparar ao mesmo volume de dados da base de filmes. Sobre a base de filmes, como já dissemos anteriormente, esta já tinha sido pré classificada por seus autores \cite{pang2004sentimental}.

A terceira base, um conjunto de documentos retirados do site Epinions por \cite{taboada2011lexicon}, é composta por 400 documentos de 8 categorias diferentes: livros, carros, computadores, panelas, hotéis, filmes, música e telefones. Cada categoria possui 50 documentos equilibrados, sendo 25 positivos e 25 negativos, classificados préviamente pelo autor da base. Essa base foi utilizada para verificarmos a efetividade de regras criadas em uma base de dados e usadas em outras, nesse caso, a base da Epinions. 

\section{Design dos experimentos}

Nós focamos em comparar os métodos de classificação MGRF e MCRF, variando as configurações em diferentes etapas do processo de mineração de opinião, comparando as medidas de \textit{accuracy} e \textit{F1}. Nós também avaliamos a influência dos algoritmos de seleção de características, os sistemas de inferência fuzzy em si, a quantidade usada de conjuntos fuzzy, a eficiência das regras geradas num domínio e usadas em outro e as características mais selecionadas entre as bases utilizadas. 

Para cada base de dados, o processo é idêntico para as etapas de pré-processamento, transformação e extração de características. A partir da seleção de características, as etapas seguintes foram executadas com validação cruzada de 10 dobras, utilizando somente as treinamento. Por exemplo, a dobra 1 não é usada para a seleção de características e é usada como teste nas etapas seguintes de classificação e avaliação. Mas as dobras restantes são usadas na seleção de características e na construção base de regras fuzzy para aquela dobra 1. O mesmo processo é repetido para cada dobra e nossos resultados, para todas as métricas usadas, são a média das dobras de teste. Conseqüentemente, todos os tipos de n-grams combinados com todas as técnicas de transformação descritas nesse trabalho passam pela seleção de características para encontrar quais delas são mais apropriadas para representar os documentos. 

\subsection{Avaliação dos algoritmos de seleção de características}

Para avaliar os algoritmos de seleção de características, CFS e C45, foi definido o seguinte cenário: 3 conjuntos fuzzy para as variáveis de entrada e o uso do MCRF para ambos os datasets, filmes e a base mista da Amazon. Assim, deixando os demais parâmetros inalterados, nós podemos avaliar o desempenho dos algoritmos de seleção de características. Além de avaliar \textit{accuracy}, \textit{recall}, \textit{precision} e \textit{F1}, foi avaliado também a quantidade média de características selecionadas para cada algoritmo de seleção. A tabela (\ref{table:movies}) e tabela (\ref{table:amazon}) mostram, respectivamente, os resultados para este cenário nas bases de filmes e da Amazon. 

\begin{table}[!h]
    \begin{tabular}{lll}
    Movies         				 & CFS                                 	 	 & c4.5                                  \\ \hline
    Precision                   & 53.65\% $\pm$ 6.09\% 			 & 73.09\% $\pm$ 21.24\% \\
    Recall                        & 64.2\% $\pm$ 30.55\% 		 & 52.3\% $\pm$ 44.78\% \\
    Accuracy                   & 52.25\% $\pm$ 4.92\% 			 & 54.2\% $\pm$ 1.76\% \\
    F1                  			 & 53.45\% $\pm$ 14.88\% 	     & 41.04\% $\pm$ 3.83\% \\
    Features selected      & 4.7 $\pm$ 1.1            			 & 1                                     \\
    \end{tabular}
    \caption{Resultados da base de filmes}
	\label{table:movies}
\end{table}

\begin{table}[!h]
    \begin{tabular}{lll}
    Movies         					& CFS                          		& c4.5                                  \\ \hline
    Precision                     & 61.99\% $\pm$ 9.01\% 	& 66.01\% $\pm$ 22.25\%  \\
    Recall                          & 77.5\% $\pm$ 13.9\% 		& 74.5\% $\pm$ 38.82\% \\
    Accuracy                     & 64.4\% $\pm$ 8.12\% 		& 54.25\% $\pm$ 2.82\% \\
    F1                                & 68.17\% $\pm$ 8.65\% 	& 55.42\% $\pm$ 19.43\% \\
    Features selected 		& 5.3 $\pm$ 0.64               & 1 $\pm$                                  \\
    \end{tabular}
    \caption{Resultados da base da Amazon}
	\label{table:amazon}
\end{table}

Como pode ser visto, a seleção de características com c4.5 usando o MCRF e três conjuntos fuzzy na entrada produziu melhor \textit{precision} e \textit{accuracy}, esta última mesmo que próxima, na base de filmes, usando menos da metade de características usadas no algoritmo CFS. Contudo, o inverso ocorreu na base da Amazon, onde o CFS com o MCRF produziu melhor resultado que o c4.5, mas utilizando ainda mais características para a geração das regras. Isso resulta em regras com 5 antecedentes, em média, menos legíveis e compreensíveis para um ser humano. Assim, já que o algoritmo c4.5 somente precisou de 2 características, produzindo regras menos complexas e mais claras, para produzir resultados melhores (considerando \textit{accuracy)}, no caso de filmes, e próximos, no caso da Amazon, foi decidido em manter o c4.5 para os próximos cenários de avaliação. 

Um outro comportamento, agora comum para ambas as bases foi o baixo balanço (\textit{F1}) entre a classificação dos documentos positivos e negativos. Foi possível perceber que a cada dobra da validação cruzada, documentos positivos eram classificados mais corretamente em detrimento aos negativos e vice-versa.
A tabela (\ref{table:amazon_folds}) mostra as matrizes de confusão de cada dobra para os resultados produzidos na base da Amazon. 

\begin{table}[!h]
    \begin{tabular}{lll}
    a         					& b                          		& classificado como                                  \\ \hline
	Dobra 0 \\    
    6 (TP)    				&94 (FN)      				& b = negative (100) \\
    0 (FP)    				&100 (TN)      				& b = negative (100) \\
	&&\\
	Dobra 1    \\
    5 (TP)    				&95 (FN)      				& b = negative (100) \\
    0 (FP)    				&100 (TN)      				& b = negative (100) \\
	&&\\ 
	Dobra 2    \\
    96 (TP)    				&4 (FN)      				& b = negative (100) \\
    87 (FP)    				&13 (TN)      				& b = negative (100) \\
	&&\\
    Dobra 3    \\
    98 (TP)    				&2 (FN)      				& b = negative (100) \\
    94 (FP)    				&6 (TN)      				& b = negative (100) \\
	&&\\
	Dobra 4    \\
    99 (TP)    				&1 (FN)      				& b = negative (100) \\
    88 (FP)    				&12 (TN)      				& b = negative (100) \\
	&&\\
	Dobra 5    \\
    4 (TP)    				&96 (FN)      				& b = negative (100) \\
    1 (FP)    				&99 (TN)      				& b = negative (100) \\
	&&\\
	Dobra 6    \\
    9 (TP)    				&91 (FN)      				& b = negative (100) \\
    0 (FP)    				&100 (TN)      				& b = negative (100) \\
	&&\\
	Dobra 7    \\
    97 (TP)    				&3 (FN)      				& b = negative (100) \\
    83 (FP)    				&17 (TN)      				& b = negative (100) \\
	&&\\
	Dobra 8    \\
    14 (TP)    				&86 (FN)      				& b = negative (100) \\
    2 (FP)    				&98 (TN)      				& b = negative (100) \\
	&&\\
	Dobra 9    \\
    95 (TP)    				&5 (FN)      				& b = negative (100) \\
    84 (FP)    				&16 (TN)      				& b = negative (100) \\
	&&\\
    \end{tabular}
    \caption{Resultados das dobras da base da Amazon}
	\label{table:amazon_folds}
\end{table}

É possível perceber que há alternância entre as dobras para as extremidades da classificação dos documentos: ou são todos (ou quase) positivos ou quase todos negativos. E ainda que um novo embaralhamento dos dados seja feito e uma nova seleção de características, o efeito se repete e os resultados pouco melhoram e pioram bastante em alguns casos. 

\subsubsection{Avaliação do impacto da altura da árvore de decisão do algoritmo de seleção de características do c4.5}

É importante ressaltar que os resultados anteriores, para o c4.5, foram obtidos com o algoritmo otimizado, utilizando a ferramenta Weka [CITE], para que a árvore tivesse altura 1, para fins de simplificação e aumento da generalização do algoritmo. Dessa forma, decidimos avaliar o quanto o aumento da altura da árvore de decisão do algoritmo de seleção de características do c4.5 influenciaria na classificação dos documentos. Definimos que o limite da altura da árvore de decisão fosse 3 para não aumentar demais a complexidade das regras que serão geradas. A tabela (\ref{table:movies_h3}) e tabela (\ref{table:amazon_h3}) mostram, respectivamente, os resultados para este cenário nas bases de filmes e da Amazon.

 \begin{table}[!h]
    \begin{tabular}{lll}
    Movies         				 & c4.5 - Altura 3                                 	 	 & c4.5 - Altura 1                               \\ \hline
    Precision                   & 71.77\% $\pm$ 19.87\% 			 & 73.09\% $\pm$ 21.24\% \\
    Recall                        & 53.4\% $\pm$ 44.50\% 		 & 52.3\% $\pm$ 44.78\% \\
    Accuracy                   & 54.5\% $\pm$ 1.76\% 			 & 54.2\% $\pm$ 1.76\% \\
    F1                  			 & 42.28\% $\pm$ 26.63\% 	     & 41.04\% $\pm$ 3.83\% \\
    Features selected      & 1.1 $\pm$ 0.3            			 & 1                                     \\
    \end{tabular}
    \caption{Resultados da base de filmes}
	\label{table:movies_h3}
\end{table}

\begin{table}[!h]
    \begin{tabular}{lll}
    Movies         					& c4.5 - Altura 3                          		& c4.5 - Altura 1                              \\ \hline
    Precision                     & 81.22\% $\pm$ 18.36\% 		& 66.01\% $\pm$ 22.25\%  \\
    Recall                          & 47.1\% $\pm$ 34.22\% 		& 74.5\% $\pm$ 38.82\% \\
    Accuracy                     & 60.4\% $\pm$ 5.72\% 			& 54.25\% $\pm$ 2.82\% \\
    F1                                & 47.62\% $\pm$ 19.17\% 	& 55.42\% $\pm$ 19.43\% \\
    Features selected 		& 1.9 $\pm$ 0.7               		& 1 $\pm$                                  \\
    \end{tabular}
    \caption{Resultados da base da Amazon}
	\label{table:amazon_h3}
\end{table}

Os resultados mostram comportamentos diferentes entre as bases, mas uma tendência já vista nos resultados anteriores. A base da Amazon precisa de mais características para poder classificar melhor os documentos. Esses resultados sugerem que para esse tipo de conjunto de dados diversificado (lembrando que a base da Amazon é composta de filmes, hotéis, GPS, telefones, dentre outros), faz-se necessário mais características para descrever melhor os documentos para que possam ser identificados corretamente pelo classificador. Por outro lado, os resultados pouco se alteraram na base de filmes, evidenciando que pouco será eficaz o acréscimo de mais características para descrever os documentos. Nesse caso, para a base de filmes, é necessário encontrar as características mais aptas a descreverem melhor os documentos. 

\subsection{Avaliação dos sistemas de inferência}



\subsubsection{Avaliação de uso de pesos nas regras dos sistemas de inferência}

\subsection{Avaliação da quantidade de conjuntos fuzzy usados}

\subsection{Avaliação do uso de regras entre domínios}

Usar regras geradas em um domínio e usar para classificar em outro.

\subsection{Avaliação das features mais selecionadas entre domínios}

Discutir os motivos de tais features terem sido mais selecionadas em vez das outras

\section{Outros resultados}

Seguindo a reta do artigo
