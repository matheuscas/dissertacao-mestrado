\xchapter{Resultados obtidos}{}

Nessa seção são descritos e discutidos os experimentos realizados e os resultados obtidos desses experimentos. O objetivo principal não é comparar qual cenário obtém melhor \textit{accuracy} ou \textit{F1}, mas também discutir em quais contextos os classificadores produzem melhores ou piores resultados.

\section{Datasets}

Falar dos datasets usados, características, algum tratamento inicial, como o da Amazon que diminui para 2000 e deixei equilibrado.
Filmes - Cornell
Amazon - Genérico
Epinions - 400 - Taboada

\section{Design dos experimentos}

Descrever de forma geralc como foram realizados os experimentos. Com a exceção de análises mais específicas como o uso de regras de uma base para a outra, o processo é o mesmo: pre-processamento, transformacao, extracao, selecao (com cross-fold), classificacao (com cross-fold) e avaliacao (com cross-fold).

\subsection{Avaliação dos algoritmos de seleção de features}

\subsubsection{Avaliação do impacto da arvore de decisao do C45}

\subsection{Avaliação dos sistemas de inferência}

\subsection{Avaliação da quantidade de conjuntos fuzzy usados}

\subsection{Avaliação do uso de regras entre domínios}

Usar regras geradas em um domínio e usar para classificar em outro.

\subsection{Avaliação das features mais selecionadas entre domínios}

Discutir os motivos de tais features terem sido mais selecionadas em vez das outras

\section{Outros resultados}

Seguindo a reta do artigo
