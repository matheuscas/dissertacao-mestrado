%% Template para dissertação/tese na classe UFBAthesis
%% versão 0.9.2
%% (c) 2005 Paulo G. S. Fonseca
%% (c) 2012 Antonio Terceiro
%% www.dcc.ufba.br/~terceiro/ufbathesis

\documentclass[msc, a4paper, classic, pt]{ufbathesis}
\usepackage[utf8]{inputenc}

%% Preâmbulo:
%% coloque aqui o seu preâmbulo LaTeX, i.e., declaração de pacotes,
%% (re)definições de macros, medidas, etc.

\title{<TÍTULO DA OBRA>}
\date{<DATA DA DEFESA>}
\author{<NOME DO AUTOR>}
\adviser{<NOME DO(DA) ORIENTADOR(A)>}
\coadviser{<NOME DO(DA) CO-ORIENTADOR(A)>}

\begin{document}

% Folha de rosto
\dmccfrontpage{MMCC-Msc-XXXX}
% Se seu trabalho não for uma tese de doutorado do DMCC, apague a linha
% acima e use \frontpage

%%
%% Parte pré-textual
%%
\frontmatter

% Portada (apresentação)
\dmccpresentationpage
% Se seu trabalho não for uma tese de doutorado do DMCC, apague a linha
% acima e use \presenationpage

% Ficha catalográfica
\authorcitationname{<SEU NOME EM CITAÇÕES>} % e.g. Terceiro, Antonio Soares de Azevedo
\advisercitationname{<NOME DO SEU ORIENTADOR EM CITAÇÕES>} % e.g. Chavez, Christina von Flach Garcia
\coadvisercitationname{<NOME DO SEU CO-ORIENTADOR EM CITAÇÕES>} % e.g. Mendonca, Manoel Gomes de
\catalogtype{<TIPO DE TRABALHO>} % e.g. ``Tese (doutorado)''
\catalogtopics{<TOPICOS PARA FICHA CATALOGRAFICA>} % e.g. ``1. Complexidade Estrutural. 2. Engenharia de Software''
\catalogcdd{<NUMERO CDD>} % e.g. ``CDD 20.ed. XXX.YY'' (esse número vai lhe ser dado pela biblioteca)
\catalogingsheet

% Termo de aprovação - exemplo
% Modifique com os membros da sua banca
\approvalsheet{Salvador, <DIA> de <MÊS> de <ANO>}{
  \comittemember{Profa. Dra. Professora 1}{Universidade XYZ}
  \comittemember{Prof. Dr. Professor 2}{Universidade 123}
  \comittemember{Profa. Dra. Professora 3}{Universidade ABC}
  \comittemember{Prof. Dr. Professor 4}{Universidade HJKL}
  \comittemember{Profa. Dra. Professora 5}{Universidade QWERTY}
}

% Agradecimentos
% Se preferir, crie um arquivo à parte e o inclua via \include{}
\acknowledgements
<DIGITE OS AGRADECIMENTOS AQUI>

% Resumo em Português
% Se preferir, crie um arquivo à parte e o inclua via \include{}
\resumo
<DIGITE O RESUMO AQUI>
% Palavras-chave do resumo em Português
\begin{keywords}
<DIGITE AS PALAVRAS-CHAVE AQUI>
\end{keywords}

% Resumo em Inglês
% Se preferir, crie um arquivo à parte e o inclua via \include{}
\abstract
% Palavras-chave do resumo em Inglês
\begin{keywords}
<DIGITE AS PALAVRAS-CHAVE AQUI>
\end{keywords}

% Sumário
% Comente para ocultar
\tableofcontents

% Lista de figuras
% Comente para ocultar
\listoffigures

% Lista de tabelas
% Comente para ocultar
\listoftables

%%
%% Parte textual
%%
\mainmatter

% É aconselhável criar cada capítulo em um arquivo à parte, digamos
% "capitulo1.tex", "capitulo2.tex", ... "capituloN.tex" e depois
% incluí-los com:
% \include{capitulo1}
% \include{capitulo2}
% ...
% \include{capituloN}
%
% Importante: Use \xchapter ao invés de \chapter, conforme exemplo abaixo.

\xchapter{Introdução}{}

\begin{itemize}

\item Motivações
\begin{itemize}
\item As opiniões são as principais influenciadoras do comportamento humano (Liu, 2012);
\item Pessoas pedem opiniões a outras pessoas para consumo (mídia, produtos, etc.), posições políticas, religiosas, etc.;
\item Empresas precisam saber as opiniões de seus consumidores;
\item A internet e a web mudaram a forma de comunicação entre pessoas e organizações;
\item Há mais facilidade de acesso a fontes de informações e de disponibilização de opiniões (USAR DADOS PARA EMBASAR ESSA INFORMACAÇÃO);
\end{itemize}

\item Problemas
\begin{itemize}
\item Grande quantidade e diversidade de fontes de opiniões;
\item Diferentes formatos, problemas de sintaxe, gírias, dentro outros;
\item É difícil para uma pessoa comum extrair e resumir opiniões de centenas ou milhares de fontes (USAR EMBASAMENTO);
\item Opiniões são carregadas de sentimentos subjetivos e imprecisos;
\end{itemize}

\item Proposta / Definição
\begin{itemize}
\item É preciso utilizar uma metodologia para lidar com imprecisões e vagueza;
\item Lógica fuzzy (Zadeh, 1965): É uma metodologia da Inteligência Computacional que trata computacionalmente dados imprecisos e vagos;
\item ANGELO: poucos trabalhos sobre a aplicação de lógica fuzzy e/ou sistemas fuzzy foram encontrados, e os encontrados são limitados (tipo de limitação?), demonstrando uma lacuna de pesquisa sobre a aplicação de sistemas fuzzy para mineração de opiniões 
\item Por todos os problemas citadas, associadas as motivações, torna-se evidente a necessidade de sistemas automáticos de mineração de opiniões;
\item Definição de "Opinion Mining";
\item Pela vagueza das opiniões, a Lógica nebulosa pode ser útil para esses sistemas automatizados;
\item PROPOSTA: Propor e avaliar o desempenho de um sistema automatizado de mineração de opiniões baseado na lógica nebulosa;
\item DIFERENCIAL: 
\begin{itemize}
\item Geração de regras fuzzy;
\item Extração e definição de características dos documentos para geração das regras;
\item Apresentação do uso do método de Wang-Mendel no processo de mineração de opinião;
\end{itemize}

\item Perguntas de pesquisa:
\begin{itemize}
\item Como Lógica Nebulosa pode ser utilizada no processo de mineração de opiniões?
\item Quais são os ganhos, caso existam, do uso de lógica nebulosa em mineração de opiniões?
\item ANGELO: mais umas duas questões de pesquisa
\item ANGELO: depois de escrever os resultados e discussão, revisar esta seção 
\end{itemize}

\item Objetivos secundários:
\begin{itemize}
\item Investigar o estado da arte do uso de Lógica Nebulosa em mineração de opiniões;
\item Formalizar o processo de mineração de opiniões e propor uma aplicação da lógica nebulosa;
\item dentificar e compor uma base de dados para avaliação da proposta;
\item Analisar os resultados e compara-los com outros métodos da literatura;
\end{itemize}
\end{itemize}
\end{itemize}

\xchapter{Revisão da literatura}{}

\begin{itemize}
\item A área de mineração de opiniões é recente (Pang and Lee, 2008);
\item Tão recente que ainda existem problemas de terminologias (mineração de opinião, análise de sentimentos, mineração de sentimentos, análise afetiva, dentre outros);
\item Definição de opinion mining (usar um cara forte para isso, como Lib Bing, Po Pang, Turney, etc.)

\item Definição formal de opinião 'O':
\begin{itemize}
\item O = (g, s), onde \emph{g} é o alvo e \emph{s} é o sentimento associados a opinião
\end{itemize}

\item Níveis de mineração de opinião (cita e depois explica cada um deles e diz qual foi usado e por que. Além de, claro, falar de trabalhos que se encaixam em cada nível)
\begin{itemize}
\item Nível de documento;
\item Nível de sentença;
\item Nível de entidades e seus aspectos
\end{itemize}

\item Lógica Fuzzy
\begin{itemize}
\item Muitas das informações que lidamos são imprecisas e vagas;
\item A lógica clássica não consegue lidar com esse tipo de informação;
\item Para isso,Zadeh (1965) propôs a Lógica Nebulosa para lidar com informações vagas e imprecisas

\item Conjuntos Fuzzy
\begin{itemize}
\item A lógica nebulosa diz que um elemento pode fazer parte de mais de um conjunto com graus de pertinência para cada um deles (Uma opinião pode ser positiva e negativa ao mesmo tempo, mas com graus de pertinência para cada conjunto fuzzy)
\item A lógica clássica, por outro lado, determina que um objeto pertençe ou não a um conjunto (Ou uma opinião é positiva ou negativa);
\end{itemize}

\item Wang-Mendel
\begin{itemize}
\item O que é?
\item Explicar o método
\end{itemize}

\item Fala de outros algoritmos de classificação e agrupamento? Tem isso na qualificação, mas não sei se é mais pertinente (??????)

\item Métodos de classificação com uso de regras
\begin{itemize}
\item CFRM
\item GFRM
\end{itemize}
\end{itemize}
\item Trabalhos relacionados
\begin{itemize}
\item Papers with opinion mining? (Angelo: Acho que alguns principais, mais para contextualizar as abordagens, independente e dependente de domínio, classificação com palavras e com features)
\item Papers with both? (angelo: Sim, descrever rapidamente os encontrados e fazer contraste com a nossa proposta)
\end{itemize}

\end{itemize}

\xchapter{Metodologia}{}

\begin{itemize}
\item O processo de mineração de opinião
\begin{itemize}

\item Preprocessing
\begin{itemize}
\item Definição
\item Reiterar o tipo de análise escolhida: Document Level Analysis
\item Datasets utilizados (Cornell e Amazon - descreve as caracteristicas de cada dataset, como quantidade, balanceamento, natureza, etc.)

\item Explicar as tarefas envolvidas:
\begin{itemize}
\item Remoção de sentenças que contem "modals"
\item POS Tagging
\item Tokenization -> bag-of-words model (n-grams e explica os tipos de n-grams)
\item Resume o subcapitulo e finaliza falando da saida dessa fase para a proxima
\end{itemize}
\end{itemize}

\item Transformation
\begin{itemize}
\item O que é
\item Uso de dicionários de opinião para transformar n-grams e numeros
\item Explicar porque o uso de dicionarios de opiniao é bom para nosso trabalho
\item Sentiwordnet (falar sobre explicar, genericamente, como é criado, como é estruturado, valores associados aos termos)
\item Falar do trabalho de Guerrine que propoe diferentes formulas para determinar a polaridade de um n-gram. E que são melhores que a mais frequente do SWN
\item Falar e explicar os intensificadores
\item Negação
\item Frequencia de n-grams como transformador da polaridade
\item Compensação de bias negativo
\item Resume o subcapitulo e finaliza falando da saida dessa fase para a proxima 
\end{itemize}

\item Feature extraction
\begin{itemize}
\item O que é
\item Por que extrair features
\item Corpus-based vs domain independent 
\item Features usadas de outros trabalhos. Citar todas e falar o que é cada uam delas
\item Features derivadas e criadas por este trabalho. Fala de cada uma delas. 
\item Resume o subcapitulo e finaliza falando da saida dessa fase para a proxima
\end{itemize}

\item Feature selection
\begin{itemize}
\item O que é
\item Por que selecionar features
\item Falar dos algoritmos de seleção utilizados (explicar em linhas gerais como eles fazem a seleção)
\begin{itemize}
\item Correlation$-$based Feature Selection (CFS)
\item C4.5 decision tree
\end{itemize}
\item Resume o subcapitulo e finaliza falando da saida dessa fase para a proxima 
\end{itemize}

\item Classification
\item Evaluation
\end{itemize}
\end{itemize}


\xchapter{Resultados obtidos}{}

\xchapter{Considerações finais e trabalhos futuros}{}

\backmatter

% Apêndices
% Comente se não houver apêndices
\appendix

% É aconselhável criar cada apêndice em um arquivo à parte, digamos
% "apendice1.tex", "apendice.tex", ... "apendiceM.tex" e depois
% incluí-los com:
% \include{apendice1}
% \include{apendice2}
% ...
% \include{apendiceM}


% Bibliografia
% É aconselhável utilizar o BibTeX a partir de um arquivo, digamos "biblio.bib".
% Para ajuda na criação do arquivo .bib e utilização do BibTeX, recorra ao
% BibTeXpress em www.cin.ufpe.br/~paguso/bibtexpress
\bibliographystyle{abnt-alf}
\bibliography{biblio}

%% Fim do documento
\end{document}