%% Template para dissertação/tese na classe UFBAthesis
%% versão 0.9.2
%% (c) 2005 Paulo G. S. Fonseca
%% (c) 2012 Antonio Terceiro
%% www.dcc.ufba.br/~terceiro/ufbathesis

\documentclass[msc, a4paper, classic, pt]{ufbathesis}
\usepackage[utf8]{inputenc}
\usepackage{array,graphicx}
\usepackage{url}
\usepackage{todonotes}
\usepackage[normalem]{ulem}
\usepackage{float}
\usepackage{algorithm}
\usepackage{algorithmic}
\usepackage{subfiles}
\usepackage{etoolbox}
\usepackage{booktabs}
\usepackage{pifont}

\newcommand*\rot{\rotatebox{90}}
\newcommand*\OK{\ding{51}}

%\newcommand{\mainfile}{}

\newcounter{example}[section]
\newenvironment{example}[1][]{\refstepcounter{example}\par\medskip
   \noindent \textbf{Exemplo~\theexample. #1} \rmfamily}{\medskip}


%% Preâmbulo:
%% coloque aqui o seu preâmbulo LaTeX, i.e., declaração de pacotes,
%% (re)definições de macros, medidas, etc.

\title{Aplicação de um sistema fuzzy para classificação de opinião em diferentes domínios}
\date{19/05/2015}
\author{Matheus Cardoso de Andrade Silva}
\adviser{Prof. Dr. Angelo Loula}
\coadviser{Prof. Dr. Matheus Giovanni Pires}

\begin{document}

% Folha de rosto
%\dmccfrontpage{MMCC-Msc-XXXX}
\frontpage
% Se seu trabalho não for uma tese de doutorado do DMCC, apague a linha
% acima e use \frontpage

%%
%% Parte pré-textual
%%
\frontmatter

% Portada (apresentação)
\dmccpresentationpage
% Se seu trabalho não for uma tese de doutorado do DMCC, apague a linha
% acima e use \presenationpage

% Ficha catalográfica
\authorcitationname{CARDOSO, M.} % e.g. Terceiro, Antonio Soares de Azevedo
\advisercitationname{LOULA, A.} % e.g. Chavez, Christina von Flach Garcia
\coadvisercitationname{PIRES, M.G.} % e.g. Mendonca, Manoel Gomes de
\catalogtype{Dissertação (mestrado)} % e.g. ``Tese (doutorado)''
\catalogtopics{(1. Mineração de dados. 2. Lógica difusa. 3. Conjuntos difusos. 4. Sistemas difusos)} % e.g. ``1. Complexidade Estrutural. 2. Engenharia de Software''
% Os topicos forma retirados de: http://consorcio.bn.br/scripts/odwp032k.dll?t=xs&pr=assuntos_pr&db=assuntos&use=kw_livre&disp=list&sort=off&ss=new&arg=mineracao+de+dados+&x=0&y=0
\catalogcdd{(NUMERO CDD)} % e.g. ``CDD 20.ed. XXX.YY'' (esse número vai lhe ser dado pela biblioteca)
\catalogingsheet

% Termo de aprovação - exemplo
% Modifique com os membros da sua banca
\approvalsheet{Feira de Santana, 19 de Outubro de 2015}{
  \comittemember{Profa. Dra. Heloisa de Arruda Camargo}{Universidade Federal de São Carlos}
  \comittemember{Prof. Dr. João B. Rocha Junior}{Universidade Estadual de Feira de Santana}
  \comittemember{Prof. Dr. Angelo Conrado Loula}{Universidade Estadual de Feira de Santana}
%  \comittemember{Prof. Dr. Professor 4}{Universidade HJKL}
%  \comittemember{Profa. Dra. Professora 5}{Universidade QWERTY}
}

% Agradecimentos
% Se preferir, crie um arquivo à parte e o inclua via \include{}
\acknowledgements
Àqueles que foram \textit{sine qua non} a realização desde trabalho. 

Meus genuínos agradecimentos a meus orientadores, Angelo Loula e Matheus Pires, pela paciência, diligência e competência prestadas a mim. Decididamente, sem a orientação de vocês, este trabalho não teria sido realizado. Muito obrigado, meus caros.

À meu tio, Marcelo Cordeiro, que me acolheu sem nenhuma obrigação ou necessidade e me deu suporte para recomeçar a vida e terminar o mestrado. Certamente eu não estaria aqui hoje sem a ajuda que você me deu. Valeu, titio!

À mãe do meu filho, Luana Lira, que neste último ano, mesmo sem saber, ajudou-me, ao mesmo tempo, a cuidar do meu Ben e do meu mestrado. Obrigado, Luana.

E à todos os demais que potencialmente jamais lerão essa breve nota e que se sentirem, de maneira surpreendente, ressentidos por não os terem mencionado, minhas desculpas, mas minha memória se foi. Mas as lembranças sempre ficarão.

% Resumo em Português
% Se preferir, crie um arquivo à parte e o inclua via \include{}
\resumo
Opiniões são centrais em quase todas as atividades humanas, porque exercem relevante influência sobre o comportamento das pessoas. A internet e a web criaram mecanismos que tornaram possível que as pessoas pudessem compartilhar suas opiniões e para que eias, e também organizações, pudessem encontrar facilmente mais informações sobre as opiniões e experiências de outros indivíduos para ajudar em tomadas de decisão. Ainda assim, opiniões envolvem sentimentos que são descrições textuais vagas e imprecisas. Devido à natureza destes dados, a Lógica Fuzzy pode ser uma abordagem promissora para lidar com esses tipos de informações. Assim, este trabalho propõe a criação e a avaliação de sistema fuzzy para realizar um processo de mineração e classificação de opinião em diferentes domínios. Diversas características foram extraídas dos documentos e algoritmos de seleção de características foram aplicados para selecionar as mais aptas para representar e classificar os documentos. Com base nas características selecionadas, o método de Wang-Mendel (WM) e variados métodos de inferência foram utilizados para gerar as regras fuzzy e classificar documentos. Os resultados obtidos mostraram que a proposta é promissora, pois o desempenho das regras geradas se equipararam ou superaram trabalhos correlatos na literatura, demonstraram bons resultados entre domínios diferentes e ainda se igualaram a resultados de métodos clássicos de aprendizado de máquina, como o \textit{Support Vector Machine}.
% Palavras-chave do resumo em Português
\begin{keywords}
mineração de opinião; classificação de sentimento; lógica fuzzy; sistemas de inferência fuzzy; extração de características; seleção de características; métodos de raciocínio fuzzy
\end{keywords}

% Resumo em Inglês
% Se preferir, crie um arquivo à parte e o inclua via \include{}
\abstract
Opinions are central in almost all human activities, because they are a relevant influence on people’s behavior. The internet and the web have created mechanisms that made possible for people to share their opinions and for other people and organizations to find out more about opinions and experiences from individuals and help in decision making. Still, opinions involve sentiments that are vague and inaccurate textual descriptions. Hence, due to data's nature, Fuzzy Logic can be a promising approach. This paper proposes a fuzzy system to perform opinion classification across different domains. Many features has been extracted from documents and algorithms of feature selection was applied to select the most fitted ones to represent and classify documents. Over the selected features, the Wang-Mendel (WM) method and several fuzzy inference methods were used to generate fuzzy rules and classify documents. The results have showed that this proposal is promising, matching or winning comparisons from related works results, showing good results across domains and evened the results compared with a classic machine learning technique, the Suport Vector Machine.
% Palavras-chave do resumo em Inglês
\begin{keywords}
opinion mining; sentiment classification; fuzzy logic; fuzzy inference systems; feature extraction; feature selection; fuzzy reasoning methods
\end{keywords}

% Sumário
% Comente para ocultar
\tableofcontents

% Lista de figuras
% Comente para ocultar
\listoffigures

% Lista de tabelas
% Comente para ocultar
\listoftables

%%
%% Parte textual
%%
\mainmatter

% É aconselhável criar cada capítulo em um arquivo à parte, digamos
% "capitulo1.tex", "capitulo2.tex", ... "capituloN.tex" e depois
% incluí-los com:
% \include{capitulo1}
% \include{capitulo2}
% ...
% \include{capituloN}
%
% Importante: Use \xchapter ao invés de \chapter, conforme exemplo abaixo.

%\documentclass[template.tex]{subfiles}
\begin{document}

\xchapter{Introdução}{}

\begin{itemize}

\item Motivações
\begin{itemize}
\item As opiniões são as principais influenciadoras do comportamento humano (Liu, 2012);
\item Pessoas pedem opiniões a outras pessoas para consumo (mídia, produtos, etc.), posições políticas, religiosas, etc.;
\item Empresas precisam saber as opiniões de seus consumidores;
\item A internet e a web mudaram a forma de comunicação entre pessoas e organizações;
\item Há mais facilidade de acesso a fontes de informações e de disponibilização de opiniões (USAR DADOS PARA EMBASAR ESSA INFORMACAÇÃO);
\end{itemize}

\item Problemas
\begin{itemize}
\item Grande quantidade e diversidade de fontes de opiniões;
\item Diferentes formatos, problemas de sintaxe, gírias, dentro outros;
\item É difícil para uma pessoa comum extrair e resumir opiniões de centenas ou milhares de fontes (USAR EMBASAMENTO);
\item Opiniões são carregadas de sentimentos subjetivos e imprecisos;
\end{itemize}

\item Proposta / Definição
\begin{itemize}
\item É preciso utilizar uma metodologia para lidar com imprecisões e vagueza;
\item Lógica fuzzy (Zadeh, 1965): É uma metodologia da Inteligência Computacional que trata computacionalmente dados imprecisos e vagos;
\item ANGELO: poucos trabalhos sobre a aplicação de lógica fuzzy e/ou sistemas fuzzy foram encontrados, e os encontrados são limitados (tipo de limitação?), demonstrando uma lacuna de pesquisa sobre a aplicação de sistemas fuzzy para mineração de opiniões 
\item Por todos os problemas citadas, associadas as motivações, torna-se evidente a necessidade de sistemas automáticos de mineração de opiniões;
\item Definição de "Opinion Mining";
\item Pela vagueza das opiniões, a Lógica nebulosa pode ser útil para esses sistemas automatizados;
\item PROPOSTA: Propor e avaliar o desempenho de um sistema automatizado de mineração de opiniões baseado na lógica nebulosa;
\item DIFERENCIAL: 
\begin{itemize}
\item Geração de regras fuzzy;
\item Extração e definição de características dos documentos para geração das regras;
\item Apresentação do uso do método de Wang-Mendel no processo de mineração de opinião;
\end{itemize}

\item Perguntas de pesquisa:
\begin{itemize}
\item Como Lógica Nebulosa pode ser utilizada no processo de mineração de opiniões?
\item Quais são os ganhos, caso existam, do uso de lógica nebulosa em mineração de opiniões?
\item ANGELO: mais umas duas questões de pesquisa
\item ANGELO: depois de escrever os resultados e discussão, revisar esta seção 
\end{itemize}

\item Objetivos secundários:
\begin{itemize}
\item Investigar o estado da arte do uso de Lógica Nebulosa em mineração de opiniões;
\item Formalizar o processo de mineração de opiniões e propor uma aplicação da lógica nebulosa;
\item dentificar e compor uma base de dados para avaliação da proposta;
\item Analisar os resultados e compara-los com outros métodos da literatura;
\end{itemize}
\end{itemize}
\end{itemize}

\end{document}
%
%\documentclass[template.tex]{subfiles}
\begin{document}

\xchapter{Fundamentação Teórica e revisão da literatura}{}

%\begin{itemize}
%\item A área de mineração de opiniões é recente (Pang and Lee, 2008);
%\item Tão recente que ainda existem problemas de terminologias (mineração de opinião, análise de sentimentos, mineração de sentimentos, análise afetiva, dentre outros);
%\item Definição de opinion mining (usar um cara forte para isso, como Lib Bing, Po Pang, Turney, etc.)
%
%\item Definição formal de opinião 'O':
%\begin{itemize}
%\item O = (g, s), onde \emph{g} é o alvo e \emph{s} é o sentimento associados a opinião
%\end{itemize}
%
%\item Níveis de mineração de opinião (cita e depois explica cada um deles e diz qual foi usado e por que. Além de, claro, falar de trabalhos que se encaixam em cada nível)
%\begin{itemize}
%\item Nível de documento;
%\item Nível de sentença;
%\item Nível de entidades e seus aspectos
%\end{itemize}
%
%\item angelo: apresente uma visão dos problemas de pesquisa da área (classificação de polaridade, subjetividade, etc), por ser por exemplificação se não tiver uma referência que faça um review sobre isso
%
%\item angelo: precisa falar sobre a etapa de pré-processamento e transformação, e daí comenter sobre as abordagens baseada em bag-of-words e baseada em features 
%
%\item angelo: comente em algum lugar sobre as abordagens baseada em bag-of-words e baseada em features 
%
%\item Lógica Fuzzy
%\begin{itemize}
%\item Muitas das informações que lidamos são imprecisas e vagas;
%\item A lógica clássica não consegue lidar com esse tipo de informação;
%\item Para isso,Zadeh (1965) propôs a Lógica Nebulosa para lidar com informações vagas e imprecisas
%
%\item Conjuntos Fuzzy
%\begin{itemize}
%\item A lógica nebulosa diz que um elemento pode fazer parte de mais de um conjunto com graus de pertinência para cada um deles (Uma opinião pode ser positiva e negativa ao mesmo tempo, mas com graus de pertinência para cada conjunto fuzzy)
%\item A lógica clássica, por outro lado, determina que um objeto pertençe ou não a um conjunto (Ou uma opinião é positiva ou negativa);
%\end{itemize}
%
%\item Wang-Mendel
%\begin{itemize}
%\item O que é?
%\item Explicar o método
%\end{itemize}
%
%\item Fala de outros algoritmos de classificação e agrupamento? Tem isso na qualificação, mas não sei se é mais pertinente (??????)
%
%\item Métodos de classificação com uso de regras
%\begin{itemize}
%\item CFRM
%\item GFRM
%\end{itemize}
%\end{itemize}
%\item Trabalhos relacionados
%\begin{itemize}
%\item Papers with opinion mining? (Angelo: Acho que alguns principais, mais para contextualizar as abordagens, independente e dependente de domínio, classificação com palavras e com features)
%\item Papers with both? (angelo: Sim, descrever rapidamente os encontrados e fazer contraste com a nossa proposta)
%\end{itemize}
%
%\end{itemize}

\section{Mineração de Opinão}

Mineração de opinião é o campo de estudo que analisa as opiniões, sentimentos, avaliações, atitudes e emoções de pessoas direcionadas a entidades ou alvos, como produtos, serviços, organizações, indivíduos, problemas, eventos, tópicos e seus atributos \cite{bing:2012}. A pesquisa em mineração de opinião começou com detecção de subjetividade, com os trabalhos de \citeonline{carbonell1979subjective}, \citeonline{wilks1983beliefs} e \citeonline{wilson2004just}. Essa tarefa envolvia a detecção e separação das sentenças objetivas das subjetivas, que carregam as opiniões e sentimentos atrelados. Com o passar dos anos, começando nos anos 2000, foi que a linha de pesquisa de mineração de opinião alavancou, focando em classificar as opiniões em três categorias: negativo, positivo e neutro. A partir daí muitos trabalhos foram publicados nessa área, mas com diferentes denominações, como análise de sentimentos, mineração de sentimentos, classificação de opiniões, dentre outros. 

Somente em 2003, no trabalho de \citeonline{dave2003mining}, é que o termo mineração de opinião foi usado e, juntamente com análise de sentimento, cunhado por \citeonline{nasukawa2003sentiment}, é que o termo passou a ser largamente adotado. No entanto, atualmente, ambos os termos denotam o mesmo campo de pesquisa \cite{bing:2012, pang:2008}. Sendo assim, neste trabalho, ambos os termos serão utilizados alternadamente, mas, com o objetivo de simplificar a leitura e compreensão deste texto, o termo de mineração de opinião será majoritariamente utilizado.

\section{Níveis de mineração de opinião}

%\todo[inline]{coloca exemplos aqui, pode obter documentos, comentários e sentenças de sites se achar melhor, há somente um exemplo aqui}
%\todo[inline]{matheus: nao entendi direito, mas eu adicionei exemplos em cada paragrafo. Veja se foi isso que quis dizer.}

Mineração de opiniões é uma área de pesquisa que vem sendo investigada em três principais níveis de análise: i) nível de análise de documento, ii) sentenças e iii) entidades e seus aspectos. O primeiro nível foca em classificar uma opinião de um documento expressando-a como positiva ou negativa. O segundo nível, o de sentenças, em vez de considerar o sentimento geral de um documento como todo, classifica as opiniões de cada sentença separadamente. E o último nível foca em descobrir todos os alvos existentes em sentenças e documentos e classificar as opiniões direcionadas a eles \cite{bing:2012}.

O nível de análise de documento é também denominado na literatura como uma tarefa de classificação de sentimentos em nível documento, pois considera todo o documento como uma unidade de informação \cite{bing:2012, pang:2008}. É importante salientar que, nesse nível de detalhamento, é assumido que o documento expressa opiniões direcionadas para somente um único assunto e somente possui um único autor das opiniões. Essa análise é feita normalmente sobre opiniões sobre produtos e serviços, pois cada avaliação, normalmente, foca somente em um único produto ou serviço e é escrito por somente uma pessoa \cite{bing:2012}. Por exemplo, opiniões sobre filmes retiradas do IMDB ou Rotten Tomatoes \footnote{\url{www.rottentomatoes.com/}} ou de produtos da Amazon, são utilizadas nesse nível, pois as opiniões são normalmente direcionadas a somente ao filme ou produto em questão. 

%Um grande número de trabalhos foram publicados na literatura sobre mineração de opiniões em nível de documento. \citeonline{gamon2004sentiment} utilizou opiniões de clientes registradas em pesquisas de opinião como conjunto de dados para mineração de opiniões. Neste trabalho foi utilizado o algoritmo de classificação SVM (\textit{Support Vector Machine}), que produz bons resultados em classificação textual \cite{joachims1998text}. Foram realizados também estudos para identificar quais características textuais eram mais relevantes para o treinamento do SVM e, por conseguinte, para minerar opiniões. \cite{mullen2004sentiment} fizeram um trabalho similar ao de \citeonline{gamon2004sentiment}, utilizando o algoritmo SVM e investigando as características textuais mais relevantes para melhorar a mineração e classificação das opiniões. Estes trabalhos utilizaram opiniões de filmes do site Epinions.com \footnote{Disponível em: http://www.epinions.com/} e opiniões de publicações da Pitchfork Media \footnote{Disponível em: http://pitchfork.com/}.

A mineração de opinião em nível de sentenças é uma abordagem que aumenta a granularidade da análise e determina se cada sentença de um ou mais documentos expressam opiniões positivas, negativas ou neutras. As definições do problema e da suposição principal deste nível são definidas a seguir \cite{bing:2012}: dada uma sentença, deve ser ser determinado quando ela expressa uma opinião positiva, negativa, neutra ou nenhuma opinião, considerando que a sentença deve conter somente uma única opinião de um único autor. Esse nível de análise é bastante utilizado como passo intermediário para o terceiro nível, o nível de entidade e aspectos. Analisando cada sentença individualmente é possível identificar as entidades e quais as opiniões estão sendo direcionadas à elas.  Uma aplicação para o nível de sentenças é na extração de opiniões onde o tema é livre e várias opiniões sobre diferentes assuntos emergem, como em fóruns de discussão e redes sociais. 

%O trabalho de \citeonline{Hu:2004} mostra uma sumarização das opiniões de produtos. Nele, sumarizar significa minerar as características dos produtos que possuem opiniões direcionadas a elas e classificar as opiniões como positivas ou negativas. Primeiramente, as sentenças que continham opiniões eram identificadas, utilizando um conjunto de palavras normalmente usado para expressar opiniões. Em seguida, foi definido o sentimento geral (positivo ou negativo) das opiniões, baseando-se no dicionário Wordnet \footnote{Disponível em: http://wordnet.princeton.edu/}. A sentimento geral de cada sentença foi determinado por um algoritmo específico proposto pelo trabalho. Os resultados obtidos foram comparados somente com trabalhos já realizados pelos próprios autores, os quais foram melhorados.Em \citeonline{kim2004determining} foi realizado um trabalho similar ao de \citeonline{Hu:2004}. Contudo, os autores das opiniões eram identificados.  A determinação do sentimento geral das opiniões encontradas nas sentenças foi feita da mesma forma como proposto por \citeonline{Hu:2004}. A diferença está nos algoritmos propostos que definem a orientação final de cada sentença, pois os autores propuseram três algoritmos de classificação e fizeram a análise de desempenho entre eles.

Também denominado nível de entidade e características, o nível de entidade e aspectos é o último nível de análise em mineração de opinião \cite{bing:2012}. Este nível possui duas tarefas principais \cite{bing:2012}: extração dos alvos das opiniões e a classificação das opiniões referentes a esses alvos. A primeira tarefa consiste em extrair os alvos das sentenças. Por exemplo, na sentença "A qualidade de voz desse telefone é muito boa", o alvo é a qualidade da voz e a entidade é o telefone (mais precisamente "este telefone"). A segunda tarefa consiste em classificar - como positivas, negativas ou neutras - as opiniões referentes aos aspectos e das entidades extraídas. No exemplo anterior, a opinião referente ao aspecto "qualidade de voz" da entidade "este telefone" é positiva \cite{bing:2012}.

%Assim como feito em \citeonline{Hu:2004} e \citeonline{kim2004determining}, o trabalho realizado por \citeonline{ding2008holistic} também focou em minerar opiniões de produtos e suas características. Todavia, este trabalho propôs uma nova abordagem para minerar opinião. Em vez de utilizar, por exemplo, as técnicas associadas ao uso do Wordnet, este trabalho focou em tratar problemas de sentimento geral das palavras levando em consideração o contexto (outras sentenças, outros documentos, distância da opinião referente alvos, dentre outros); tratar conflitos entre opiniões numa sentença (e.g. opiniões contrárias); e utilizar padrões lingüísticos (e.g. regras de negação, clausulas \textit{but}) para tratar palavras especiais, frases e construções verbais. O resultado do artigo é um sistema, chamado de \textit{Opinion Observer}, que produz resultados melhores que os trabalhos relacionados na literatura, como em \citeonline{Hu:2004} e  \citeonline{kim2004determining}.

\section{Etapas na mineração de opinião}

%\todo[inline]{faz uma breve introdução nesta seção, dizendo que o processo de mineração de opinião tipicamente envolve diversas etapas que podem envolver  iniciam por pré-processamento dos dados e em alguns trabalhos por extração de características até alcançar a classificação}
%\todo[inline]{matheus: feito}

O processo de mineração de opinião tipicamente envolve algumas etapas, as quais consistem desde a preparação dos dados dos documentos até a classificação destes. É comum encontrar nos trabalhos relacionados as seguintes etapas para minerar e classificar opiniões:  i) definição do domínio, ii) pré-processamento, iii) transformação, iv) seleção de características, v) classificação e vi) análise dos resultados \cite{moraes2012document}. 

A definição do domínio é a fase em que são selecionadas os tipos de dados que serão utilizados para o estudo, como filmes, hotéis, produtos, etc. O pré-processamento é a etapa em que as bases escolhidas são estruturadas para serem utilizadas nas próximas etapas, como a eliminação de termos indesejados, erros de grafia, dentre outros. A transformação é o momento em que os termos estruturados do pré-processamento são transformados em dados numéricos. A extração e seleção de características envolve a obtenção de características descritivas dos documentos a partir dos dados anteriores, assim como, a seleção das características mais relevantes. É importante destacar que a extração de características não é tão comum de ser encontrada. De fato, é típico encontrar trabalhos que não é feita qualquer extração e a seleção é realizada diretamente sobre os dados oriundos do pré-processamento. A etapa de classificação consiste na classificação dos documentos, utilizando as características selecionadas na etapa anterior. E, por fim, na etapa de avaliação é realizada a análise dos resultados obtidos pelo classificador.

As próximas seções detalharão os principais conceitos envolvidos em cada etapa do processo de mineração de opinião.

\subsection{Pré-processamento dos dados}

Em mineração de opinião, a preparação dos dados é essencial. Antes de os dados serem transformados, analisados, selecionados e, então, classificados, eles precisam ser preparados para serem usados corretamente no processo de mineração de opinião. Textos são dados não estruturados e as opiniões se misturam às porções não opinativas do documento. As principais tarefas envolvidas no pré-processamento são: a marcação gramatical das palavras do texto (do inglês, \textit{Part of Speech Tagging}), definição dos n-grams a serem utilizados e a tokenização das palavras. N-gram é uma seqüência de $n$ itens dada uma seqüência de um texto \cite{dave2003mining}. 

A marcação gramatical das palavras do texto é o processo de identificação das classes gramaticais de todos os elementos textuais do documento \cite{brill1995transformation}. Até o presente momento da pesquisa, todos os artigos encontrados executaram essa tarefa, especificando ou não o tipo do marcador gramatical, como pode ser visto em \citeonline{pang2002thumbs, turney2002thumbs, wilson2005recognizing, chaovalit2005movie}. O marcador utilizado nessa pesquisa, devido a ser o mais usado nos trabalhos relacionados, foi o proposto em \citeonline{brill1995transformation}.

Depois do texto identificado e marcado é preciso definir quais n-grams serão selecionados para a próxima etapa. Adjetivos são centrais para identificar subjetividade e, por conseguinte, opiniões em textos. Em \citeonline{hatzivassiloglou2000effects} foi desenvolvido um algoritmo para determinar o sentimento final somente de adjetivos. Neste trabalho foi notado que há relação entre adjetivos e conjunções, como "but", "and", dentre outros. A conjunção "And", por exemplo, mantem a mesma polaridade da opinião expressa pelo adjetivo, enquanto que "but", essa polaridade é invertida, em geral. Utilizando um algoritmo de aprendizado de máquina, esse artigo conseguiu classificar os adjetivos com acurácias entre 78\% e 92\%, dependendo da quantidade de dados de treino disponíveis. Outros trabalhos como os de \citeonline{wiebe2000learning} também utilizaram somente adjetivos como indicadores de subjetividade e presença de opiniões.

\citeonline{turney2002thumbs}, por sua vez, apontou que adjetivos isolados podem indicar opiniões, mas podem não ser suficientes para determinar o sentimento geral de documentos. Ele ainda considera o contexto como fator determinante, exemplificando que "unpredictable" pode ser uma opinião negativa para automóveis, quando associado a "unpredictable steering" ou positivo quando for direcionado a filmes, quando associado a "unpredictable plot". Assim, \citeonline{turney2002thumbs} expande os adjetivos e acrescenta advérbios, verbos e substantivos associados aos adjetivos, extraindo dos textos os chamados bigrams, n-grams compostos por dois elementos textuais. \citeonline{turney2002thumbs} alcançou uma média de acurácia de 74\% entre opiniões sobre automóveis e filmes. 

%\todo[inline]{não fale sobre nossa proposta ou metodologia aqui, fale de forma mais geral, além disso nós usamos o SWN, então não usamos só o texto. Tem váripos }
%\todo[inline]{matheus: removi e fiz um link rapido do paragrafo anterior com o proximo que fala da transformacao.}
%Contudo, nosso trabalho procura classificar o sentimento geral das opiniões dos documentos sem utilizar qualquer outra informação, exceto os próprios textos. Assim, nosso classificador desconhece que o texto de um dado documento se refere a filmes, hotéis ou automóveis. Os trabalhos de \citeonline{wilson2005recognizing}, \citeonline{benamara2007sentiment}, \citeonline{taboada2008extracting} e \citeonline{taboada2011lexicon}, mostraram que o uso de advérbios associados a adjetivos produzem melhores resultados que o uso isolado de adjetivos. Nesses trabalhos, dentre as diferentes nomenclaturas, os advérbios são chamados de modificadores de intensidade dos adjetivos, aumentando a polaridade do adjetivo ou diminuindo. 
%
%Portanto, como nossa proposta não utiliza contexto e adjetivos e advérbios, associados entre si, são reconhecidamente bons elementos para encontrar e mapear conteúdo opinativo, eles foram escolhidos como os n-grams a serem selecionados na etapa de pré-processamento e enviados para a etapa de transformação.

%\subsection{Transformação}

%- Abordagens de aprendizado supervisionado, em geral, não utilizam técnicas de transformação dos n-grams em valores numéricos, pois eles utilizam os proprios n-grams para o processo de seleção de características.
%
%- Abordagens de aprendizado não-supervisionado utilizam dicionários opinativos
%
%- Os dicionários opinativos podem ser construídos manualmente e automaticamente
%
%- pro e contras dos manuais
%
%- pros e contras dos automaticos
%
%- Sentiwordnet, General Inquirer, dentre outros
%
%- O SWN é mais palavras e maior cobertura
%
%- Tecnicas de transformação: posição do n-gram, frequência do n-gram
%
%- Tecnicas de transformação para n-grams de negação: inversão, shift, bias compensation 

Uma vez definidos os n-grams a serem utilizados no processo de mineração de opinião, é na etapa de transformação que uma representação numérica é computada a partir dos n-grams obtidos da etapa de pré-processamento. Diferentes técnicas são utilizadas na literatura para calcular essa representação numérica dos n-grams. \citeonline{turney2002thumbs}, por exemplo, utilizou uma técnica proposta em \citeonline{turney2001mining} chamada PMI-IR, que utiliza \textit{Pointwise Mutual Information} (PMI) e \textit{Information Retrieval} (IR) para medir a similaridade de pares de palavras ou frases. 

%A polaridade de uma ou par de palavras, assim como foi chamada a representação numérica pelo autor, era calculada comparando a similaridade dos pares de palavras com uma palavra de referência positiva  (\textit{excellent}) e com uma palavra de referência negativa (\textit{poor}). A polaridade do n-gram é negativa se o termo por mais similar à referência negativa e positiva se mais similar à referência positiva. \citeonline{wilson2005recognizing} e \citeonline{voll2007not} utilizaram a mesma técnica de \cite{turney2002thumbs}.

%Em \citeonline{tsutsumi2007movie}, a representação numérica dos n-grams é chamado de escore das palavras. Nesse artigo, o escore de uma palavra $w_i$ foi calculada pela fórmula \ref{eq:w_score}.
%
%\begin{equation}
%Score_w(w_i) = \log(\frac{pos(w_i) + 1}{\sum pos} \cdot \frac{\sum neg}{neg(w_i) + 1})
%\label{eq:w_score}
%\end{equation}
%
%onde $pos(w_i)$ e $neg(w_i)$ são a frequencia de uma palavra $w_i$ em opiniões positivas e negativas, respectivamente. $\sum pos$ e $\sum neg$ são o número de palavras em opiniões positivas e negativas, respectivamente. Se $Score_w$ for maior que zero, a polaridade é positiva, senão negativa.

Já \citeonline{taboada2008extracting}, utilizou de dicionários de opinião, que contém os termos e os respectivos graus opinativos. Contudo, esses dicionários foram criados pelos autores do artigo. \citeonline{ohana2009sentiment}, por outro lado, utilizaram um dicionário criado automaticamente, o Sentiwordnet \cite{esuli2006sentiwordnet}. É um dicionário de opiniões criado pela anotação automática dos sentimentos de cada \textit{synset} (conjuntos de sinônimos) do Wordnet, outro dicionário na língua inglesa \cite{fellbaum2005wordnet}. Segundo \citeonline{ohana2009sentiment}, dicionários manuais estão sujeitos ao enviesamento do autor, possuem alto tempo gasto para construí-los e, em geral, tem menor cobertura que dicionários criados automaticamente. \citeonline{ohana2009sentiment} também citou outros dicionários de opiniões criados automaticamente, como \textit{General Inquirer} \cite{stone1966general} \footnote{Disponível em: \url{http://www.wjh.harvard.edu/~inquirer}}, \textit{Subjectivity Clues} \cite{wilson2005recognizing} e \textit{Grefenstette} \cite{grefenstette2004coupling}, mas mostrou que o Sentiwordnet tem cobertura maior frente a estes, com mais de 28000 termos cobertos, contra 4216, 7650 e 2258 dos dicionários citados, respectivamente. Esta pesquisa decidiu pelo uso do Sentiwordnet \cite{esuli2006sentiwordnet}.

Dicionários opinativos, contudo, somente conseguem definir a polaridade de unigrams, ngrams compostos somente por um termo. Para tratar a definição da polaridade de bigrams e trigrams, por exemplo, é preciso analisar a influência entre os termos que compõem o ngram. Se tratando de advérbios e adjetivos, os advérbios são termos que alteram o grau de polaridade de uma adjetivo, seja intensificando (e.g. \textit{very good}) ou amenizando (e.g. \textit{somewhat sleazy}) \cite{quirk1985comprehensive}. Há também inumeras maneiras de tratar a influência entre os termos num bigram, uma delas, e foi a escolhida para essa pesquisa, foi a de \citeonline{taboada2008extracting}. \citeonline{taboada2008extracting} dividiu os advérbios em amplificadores e amenizadores. Os amplificadores aumentam a polaridade do adjetivo e os amenizadores diminuem. De maneira similar ao que fizeram com os dicionários anteriores, os advérbios tiveram associados, manualmente, percentuais de modificação, onde os amplificadores tem percentuais positivos e os amenizadores, negativos. Por exemplo, \textit{sleazy} tem escore ou polaridade -3 e \textit{somewhat}, um amenizador, tem percentual igual a -30\%. O bigram \textit{somewhat sleazy} terá polaridade final de $-3 + (3 \cdot 30\%) = -2$.

Outro fenômeno referente ao relacionamento entre termos num ngram é a negação. A negação ocorre quando n-grams de negação (e.g. advérbios) se associam a um ou mais n-grams, como \textit{nothing special} ou \textit{not very good}. Há, mais uma vez, diferentes maneiras de se tratar uma negação, como a inversão e o deslocamento de polaridade. A inversão inverte o sinal da polaridade do n-gram (e.g. \textit{not sleazy} resultará na polaridade +3). O deslocamento da polaridade desloca o valor da polaridade do n-gram em direção a polaridade oposta por um valor fixo (na implementação desse artigo, foi defino por 4). Assim, por exemplo, em vez de \textit{not sleazy} resultar em +3, a polaridade resultante será de $-3 + 4 = +1$ \cite{taboada2008extracting}.

Há outras abordagens de transformação das polaridades dos termos relativas ao texto, como a freqüência e ao enviesamento das opiniões \cite{taboada2011lexicon}. O tratamento da frequência de termos é a diminuição da polaridade dos termos opinativos pela quantidade de vezes que eles aparecem no texto, resultando numa polaridade $pol = pol \cdot 1/n$. A repetição de termos opinativos sugere que o autor das opiniões carece de comentários adicionais e se utiliza de uma palavra opiniativa genérica. Além disso, existe uma tendencia natural de seres humanos em favor de uma linguagem positiva \cite{boucher1969pollyanna}, resultando conseqüentemente, no enviesamento na classificação de opiniões baseadas em dicionários \cite{kennedy2006sentiment}. Assim, nós aplicamos um aumento de 50\% sobre qualquer n-gram negado que ocorresse nos textos \cite{taboada2011lexicon}.


%No caso de \citeonline{taboada2008extracting} foram utilizados 4 dicionários: adjetivos, verbos, substantivos e advérbios. Os três primeiros foram criados manualmente, e o último criado automaticamente a partir do dicionário de adjetivos, mas todos eles tiveram seus termos associados, manualmente, a um valor dentro de uma escala entre -5 a 5, onde -5 denota extrema negatividade, 0 sem polaridade (ou neutro) e 5, extrema positividade. Esses dicionários foram criados por um dos autores do artigo, aumentados por outro autor e tiveram a consistência checada por mais três pesquisadores. Diferentemente dos trabalhos de \citeonline{turney2002thumbs} e de \citeonline{tsutsumi2007movie}, \citeonline{taboada2008extracting} tratou o relacionamento entre os n-grams utilizados, como a influência de advérbios sobre os adjetivos, e como eles alteram a polaridade final destes. Nesse artigo, essa influência foi chamada de intensificação.

%Seguindo a classificação feita por \citeonline{quirk1985comprehensive}, 

%Além disso, o trabalho de \citeonline{taboada2008extracting} também tratou o fenômeno da negação. A negação ocorre quando n-grams de negação (e.g. advérbios) se associam a um ou mais n-grams, como \textit{nothing special} ou \textit{not very good}. Há diferentes maneiras de se tratar uma negação, como a inversão e o deslocamento de polaridade, proposto nesse mesmo artigo. A inversão inverte o sinal da polaridade do n-gram (e.g. \textit{not sleazy} resultará na polaridade +3). O deslocamento da polaridade 	desloca o valor da polaridade do n-gram em direção a polaridade oposta por um valor fixo (na implementação desse artigo, foi defino por 4). Assim, por exemplo, em vez de \textit{not sleazy} resultar em +3, a polaridade resultante será de $-3 + 4 = +1$.

 
%
%\citeonline{ohana2009sentiment} também trata a negação, usando uma versão do algoritmo \textit{NegEx} \cite{chapman2001evaluation}, embora não trate dos intensificadores e amenizadores. Todavia, \citeonline{ohana2009sentiment} apresenta o conceito de que a polaridade de um termo pode ser maior ou menor a depender de sua posição no texto. A polaridade do termo é alterada conforme mostrado na equação \ref{eq:adj_score}.
%
%\begin{equation}
%Score_{adj} = Score_{adj} \cdot \frac{t_i}{T} \cdot C
%\label{eq:adj_score}
%\end{equation}
%
%Onde $C$ é uma constante, $t_i$ a posição do termo $t$ relativa ao total de termos $T$ no documento.

\subsection{Extração e seleção de características}

%- Poucos trabalhos foram encontrados que extraem características
%
%- Que trabalhos foram esses, quais tipos de caracteristicas eles extrairam, como fizeram e o que fizeram com elas?
%
%- Selecao de caracteristicas é a etapa de selectionar melhores caracteristicas e diminuir a dimensionalidade dos vetores para melhorar a performance da classificacao
%
%- Não ha metodo de selecao de caracteristica que tenha se mostrado predominante na linha de pesquisa. Information Gain parece ser promissor. (Rodrigo Morais)
%
%- Cita trabalhos que fizeram seleção, qual tecnica usaram, como fizeram e como utilizaram
%
%- Cita tambem o trabalho que usou o CFS e o C45

%\subsection{Extração de características}

%\todo[inline]{precisa criar coesão com mineração de opinião, começa (ficou truncado) }
%\todo[inline]{matheus: voce alterou aqui? pq senao, eu n entendi onde não tem a coesão}

Usualmente, os dados da etapa de transformação já são considerados as características e são utilizados diretamente na seleção. Porém, uma etapa adicional poderia ser a extração de características dos documentos, que consiste em obter as características que caracterizam os documentos, independentemente do conteúdo específico destes. Essa é uma etapa encontrada em poucos trabalhos relacionados nessa linha de pesquisa. Os trabalhos de \citeonline{wilson2005recognizing} e de \citeonline{ohana2009sentiment} foram um dos poucos encontrados na literatura que definiram e extraíram as características dos documentos, e posteriormente, as utilizaram para a classificação dos mesmos.
%\todo[inline]{esse comentário sobre 'nossa pesquisa', deve ficar na metodologia}
%\todo[inline]{matheus: comentario jaz}
%Nossa pesquisa utilizou grande parte das categorias de características de \citeonline{ohana2009sentiment} para definir e extrair as nossas próprias.

%Em \citeonline{wilson2005recognizing}, que foca em classificar o sentimento geral das opiniões em nível de sentenças, definiu 5 categorias de características: palavras, modificação, estrutura, sentença e documento. As características de palavras e de modificação são os próprios n-grams, nesse caso, trigrams, uma sequência de três palavras num outra sequencia textual. \citeonline{wilson2005recognizing} estrutura os documentos numa árvore de dependencia entre os elementos do texto de um documento e alguns tipos de relacionamento entre os elementos são elencados como características, como por exemplo, se um elemento é o sujeito de uma sentença ou não. As características de sentença se aproximam mais das características utilizadas em nossa pesquisa. \citeonline{wilson2005recognizing} utiliza o dicionário de opiniões \textit{Subjectivity Clues} criado pelos próprios autores e com ele, procuraram registrar como características a quantidade de "pistas de subjetividade" na sentenças. Foram elas:
%
%\begin{itemize}
%\item Quantidade de:
%\begin{enumerate}
%\item "pistas fortes de subjetividade" (\textit{strongsubj}) na sentença corrente
%\item "pistas fortes de subjetividade" (\textit{strongsubj}) na sentença anterior
%\item "pistas fortes de subjetividade" (\textit{strongsubj}) na sentença seguinte
%\item "pistas fracas de subjetividade" (\textit{weaksubj}) na sentença corrente
%\item "pistas fracas de subjetividade" (\textit{weaksubj}) na sentença anterior
%\item "pistas fracas de subjetividade" (\textit{weaksubj}) na sentença seguinte
%\item adjetivos numa sentença
%\item advérbios numa sentença (exceto \textit{not})
%\item números cardinais numa sentença
%\item pronomes numa sentença
%\item modais numa sentença (exceto \textit{will})
%\end{enumerate}
%\end{itemize}
%
%Por fim, na categoria de características de documento, somente uma foi definida que foi o tópico do documento, que pode estar em 15 tipos diferentes de tópicos definidos pelos autores. 

%O trabalho de \citeonline{ohana2009sentiment} foi outro que extraiu características dos documentos para classifica-los. Eles definiram 5 categorias: geral, normalização, diferença entre escores positivos e negativos, negação e escores por segmento dos documentos. Nossa pesquisa utilizou grande parte das categorias de características de \citeonline{ohana2009sentiment} para definir e extrair as nossas próprias. As características foram as seguintes:
%\begin{enumerate}
%\item Gerais:
%\begin{enumerate}
%\item Soma dos escores dos adjetivos positivos e negativos
%\item Soma dos escores dos advérbios positivos e negativos
%\item Soma dos escores dos verbos positivos e negativos
%\end{enumerate}
%\item Normalização:
%\begin{enumerate}
%\item Soma normalizada dos escores dos adjetivos positivos e negativos
%\item Soma normalizada dos escores dos advérbios positivos e negativos
%\item Soma normalizada dos escores dos verbos positivos e negativos
%\end{enumerate}
%\item Diferença entre os escores positivos e negativos de:
%\begin{enumerate}
%\item Adjetivos
%\item Advérbios
%\item Verbos
%\end{enumerate}
%\item Escores por segmento dos documentos (cada documento foi segmentado em 10 partições de tamanhos iguais em número de termos):
%\begin{enumerate}
%\item Todas as características acima para cada segmento dos documentos
%\end{enumerate}
%\item Negação
%\begin{enumerate}
%\item Percentual de termos negados nos documentos
%\end{enumerate}
%\end{enumerate}

%\subsection{Seleção de características}

A seleção de características é a etapa onde as características mais relevantes são selecionadas, e que serão usadas na classificação dos documentos. Com a redução da quantidade de características, espera-se que o classificador seja mais eficiente e efetivo \cite{moraes2012document}. Medidas de seleção de características comuns nos trabalhos relacionados são \textit{document frequency} \cite{pang2002thumbs}, \textit{mutual information} \cite{turney2002thumbs} e \textit{information gain} \cite{wiebe2006word}. Contudo, nenhuma delas tem sido largamente aceita como a melhor medida de seleção de características para mineração de opiniões, embora \textit{information gain} tenha mostrado resultados competitivos \cite{moraes2012document}.
%\todo[inline]{estes não são metodos de seleção, são medidas que podem ser utilizadas para seleção}
%\todo[inline]{matheus: ok, alterado} 
%\todo[inline]{novamente não podemos falar aqui sobre 'nossa pesquisa', além disso aqui vai faltar uma cola para contextualizar porque falar de c4.5 e CFS se você não encontrou trabalhos de mineração de opinião que usem. Uma possibilidade seria dizer algo, em novo parágracomo: Quando se trata de seleção de características para sistemas fuzzy, comumente são apresentados métodos como c4.5 e o CFS para esta tarefa. E cita pelos menos uns 3 trabalhos que fizeram isso.}
%\todo[inline]{matheus: ok, mas não tem nem tres trabalhos com isso. Na verdade, cintra2008fuzzy foi o unico e que foi encontrado pelo prof. Matheus.}

O objetivo do nosso trabalho em realizar a seleção de características é gerar regras simples de classificação de sentimento, ou seja, com o menor número de antecedentes possível, para facilitar o entendimento delas por parte do usuário. Uma característica do algoritmo usado nesse trabalho para a construção das regras é que a quantidade de antecedentes é igual ao número de características selecionadas. Este algoritmo foi proposto por \citeonline{wang1992generating} (este algoritmo é detalhado na Seção \ref{metodo-wang-mendel}). Portanto, quanto menor o número de características, menor será a quantidade de antecedentes das regras.

Um dos trabalhos encontrados na literatura que realizam a seleção de características, e posteriormente, constroem as regras para um sistema fuzzy usando o algoritmo de Wang-Mendel, é o trabalho de \citeonline{cintra2008fuzzy}. Os autores compararam o desempenho de sua proposta com algoritmos clássicos de seleção de características, como CFS \cite{hall1999correlation} e o c4.5 \cite{quinlan2014c4}. 
%\todo[inline]{coloca na metodologia: Seguindo a proposta de \citeonline{cintra2008fuzzy} de avaliação de métodos de seleção de características aplicados a construção de sistemas fuzzy, decidimos usar o c4.5 e o CFS para a seleção de características neste trabalho.}

%\todo[inline]{mgpires: Matheus, insira esta referência: J. R. Quinlan. C4.5 Programs for Machine Learning. Morgan Kaufmann, CA, 1988. para o c4.5}
%\todo[inline]{matheus: mudei}
%\todo[inline]{traga a explicação sobre cfs e c4.5 que está na metodologia para cá}
%\todo[inline]{matheus: trouxe}

\todo[inline]{matheus: acrescimo sobre os tipos de algoritmos de saude}

Os métodos de seleção características são tipicamente apresentados em duas classes: \textit{filters} e \textit{wrappers}. \textit{Filters} são métodos que selecionam características independentes do modelo, como o CFS. E os métodos do tipo \textit{wrappers}, como o c4.5, por outro lado, analisam os subconjuntos de características, permitindo detectar as possíveis interações entre as características \cite{guyon2003introduction}.

A hipótese central do CFS é que um bom conjunto de características contém características que são altamente correlacionadas com a classe, mas sem qualquer correlação umas com as outras. O CFS é um algoritmo que junta essa hipótese com uma medida de correlação apropriada e uma heurística de estratégia de busca \cite{hall1999correlation}. Dessa etapa de seleção, o CFS produz uma lista das características mais aptas a classificar os documentos da base. 

O algoritmo C4.5, por outro lado, gera uma árvore de decisão que é normalmente usada para a tarefa de classificação \cite{quinlan19934}. Todavia, para construir essa árvore de decisão, esse algoritmo seleciona as melhores características entre as existentes a cada nó gerado. Portanto, uma árvore de decisão é construída, e o resultado é uma lista das características mais aptas ordenadas por relevância. Sendo assim, quando se usa o C4.5 para seleção de características, a quantidade de características selecionadas depende da altura da árvore, determinada por um parâmetro.

\subsection{Classificação}

%- Etapa que classifica os documentos com as caracteristicas selecionadas nas etapas de extração e seleção de características
%
%- Visao geral das tecnicas usadas na literatura com citacao
%
%- Fala de alguns artigos com um pouco mais de detalhes
%
%- Em relacao a fuzzy, poucos trbalhos foram encontrados utilziando essa abordagem. Menos ainda, dentre os encontrados, proviam resultados ou metodologias consistentes. 
%
%- Cita, se houver, e fala detalhadamente do artigo que utiliza sistemas fuzz e/ou baseado em regras
%
%- Fala dos assuntos centrais que permeiam o trabalho e cria sub secoes para eles
%- Logica Fuzzy e conjuntos Fuzzy
%- Sistemas fuzzy 
%- Regras Fuzzy IF-Then
%- Wang-Mendel
%- Metodos de classificacao

É nessa etapa que o sentimento geral das opiniões dos documentos são classificadas em positivo ou negativo. Diferentes técnicas de classificação podem ser encontradas na literatura, mas o \textit{Support Vector Machine} (SVM) \cite{pang2002thumbs, pang2004sentimental, tsutsumi2007movie, prabowo2009sentiment}, Naive Bayes \cite{pang2002thumbs, pang2004sentimental}, soma \cite{ohana2011domain, avancco2014lexicon} e a média \cite{turney2002thumbs, voll2007not, taboada2008extracting, taboada2011lexicon} das polaridades foram as abordagens mais encontradas nos trabalhos relacionados.

O SVM é uma popular técnica de aprendizado supervisionado, largamente utilizada na área de mineração de opinião e, possivelmente, é o método que produz os resultados com maior acurácia dentre os métodos existentes na literatura \cite{moraes2012document}. O SVM é um método de aprendizado linear que procura um hiperplano ótimo para separar duas classes, além de buscar maximizar a distância para o ponto de treino mais próximo de cada classe, com o fim de alcançar melhor performance de generalização/classificação nos dados de teste \cite{friedman2001elements}.

%No trabalho de \citeonline{pang2002thumbs}, por exemplo, o SVM é comparado com Naive Bayes e \textit{Maximum Entropy} na classificação de opiniões sobre filmes, usando uma base de dados criada pelos próprios autores. Os resultados mostraram que Naive Bayes tende a produzir os piores resultados e o SVM os melhores resultados. Em \citeonline{pang2004sentimental}, uma nova proposta de aprendizado de máquina para mineração de opiniões é apresentada para extrair dos textos somente porções subjetivas. Os autores utilizam Naive Bayes e SVM para testar a eficácia da proposta e o SVM produziu os melhores resultados, produzindo entre 87\% e 90\% de acurácia na classificação dos documentos.

A técnica de soma é a simples comparação entre a soma das polaridade positivas e negativa dos n-grams. Se a soma positiva for maior que a negativa, o documento é classificado com positivo, de outra forma, negativo. O trabalho realizado em \citeonline{ohana2011domain} utiliza a técnica da soma, comparando com o SVM. A técnica da média consiste em calcular a média resultante das polaridades. Se essa média for menor que zero, o documento é classificado como negativo e, de outra maneira, positivo \citeonline{turney2002thumbs, taboada2008extracting, pang2004sentimental}. Essas técnicas foram incorporadas 

%a comparação das somas das polaridades produziu uma acurácia de quase 72\% de acurácia. Em \citeonline{avancco2014lexicon} também é avaliada uma estratégia de uso de vários dicionários de opiniões, utilizando a comparação das somas das polaridades. Diferentemente de todos os outros trabalhos relacionados, este abordou a mineração de opinião em bases de dados da língua portuguesa. 
%Os resultados mostraram que a estratégia de uso de vários dicionários também é promissora para a língua portuguesa, produzindo 74\% de acurácia.

%No trabalho de \citeonline{turney2002thumbs}, já descrito nessa seção, o uso da média como técnica de classificação, produziu uma acurácia média de 74\% sobre uma base de opiniões com 410 documentos com diferentes domínios (automóveis, bancos, filmes, dentre outros). A acurácia vai desde 66\% para filmes até 84\% para automóveis. 
%Em \citeonline{taboada2008extracting}, técnica da média também é utilizada. O artigo apresenta uma abordagem que explora a influência de partes dos textos sobre os principais elementos indicativos de opiniões, os adjetivos: amplificadores e amenizadores, incluindo o tratamento da negação dos termos. Os resultados foram promissores, alcançando 76\% de acurácia numa base de filmes \cite{pang2004sentimental} até 90\% numa base de músicas.
Não obstante, em relação a aplicação de lógica fuzzy na classificação no processo de mineração de opinião, poucos trabalhos relacionados foram encontrados na literatura. Isso pode indicar que ainda é limitada a pesquisa sobre o uso dessa metodologia, embora a lógica fuzzy seja reconhecida como uma abordagem apropriada para lidar com dados imprecisos e vagos, como opiniões e sentimentos.
%Essa pouca quantidade de trabalhos pode indicar uma área de pesquisa em aberto em mineração de opinião. O resto dessa seção aborda os principais conceitos relacionados a lógica fuzzy, aplicada a este trabalho, relacionando, quando possível, trabalhos relacionados a nossa pesquisa e/ou aos assuntos contidos nelas.

\section{Sistemas Fuzzy}
\label{secao-sistemas-fuzzy}

Sistemas Fuzzy (SF) é uma das mais importantes áreas de aplicação da Teoria de Conjuntos Fuzzy. Geralmente, estes sistemas são estruturados por meio de regras fuzzy, os quais são conhecidos por Sistemas Baseados em Regras Fuzzy (SBRF). Os SBRF constituem uma extensão dos sistemas clássicos baseados em regras, pois os antecedentes e os consequentes das regras são compostos por sentenças lógicas fuzzy, ao invés das sentenças clássicas \cite{herrera2008genetic}. Este tipo de sistema tem sido amplamente usado para a resolução de diversos tipos de problema, tais como, problemas de controle \cite{mamdani1974application, mamdani1975experiment}, modelagem \cite{pedrycz1996fuzzy}, classificação \cite{ishibuchi1994construction, ishibuchi1995selecting}, diagnóstico médico \cite{sivasankar2010knowledge} e mineração de dados \cite{ishibuchi2005classification}.

Uma das principais razões para o uso de SBRF é a habilidade na modelagem do conhecimento vago e incerto, e a facilidade de expressar o comportamento do sistema em uma linguagem de fácil compreensão para os seres humanos. As regras possuem o seguinte formato:

\begin{center}
\textbf{SE} \textit{antecedente} \textbf{ENTÃO} \textit{consequente}
\end{center}

Os antecedentes descrevem uma condição (premissa), enquanto a parte consequente descreve uma conclusão ou uma ação que pode ser esboçada quando as premissas se verificam. Os antecedentes definem uma região fuzzy no espaço das variáveis de entrada do sistema e os consequentes descrevem uma região no espaço das variáveis de saída do sistema. 

Em uma descrição mais extensa, um sistema fuzzy é composto pelos seguintes componentes básicos: a interface de entrada (fuzificação), a base de conhecimento, o sistema de inferência e a interface de saída (desfuzificação) \cite{gomide1995conceitos}. A interface de entrada toma os valores das variáveis de entrada, transformando-os em conjuntos fuzzy, de modo que possam se tornar instâncias de variáveis linguísticas. A base de conhecimento consiste de uma base de regras, caracterizando a estratégia de controle e suas metas. A base de dados armazena as definições necessárias sobre discretizações e normalizações dos universos de discurso, as partições fuzzy dos espaços de entrada e saída e as definições das funções de pertinência. O sistema de inferência processa os dados fuzzy de entrada, juntamente com as regras, de modo a inferir as ações de controle fuzzy. A interface de saída transforma as ações de controle fuzzy inferidas em ações de controle não fuzzy. A figura \ref{fig:sistema_nebuloso} ilustra os componentes de um sistema fuzzy.

\begin{figure}[tbh]  
  \caption{Modelo de sistema fuzzy \cite{herrera2008genetic}.}
  \label{fig:sistema_nebuloso}
  \centering
    \includegraphics[width=0.9\textwidth]{sistema_nebuloso}
\end{figure}

Há duas formas básicas de se definir a base de regras de um SBRF: manual e automática. A definição manual depende inteiramente do conhecimento dos especialistas sobre o problema em questão, e por outro lado, na definição automática, são utilizados algoritmos para a construção das regras. Neste trabalho optamos por usar a forma automática, através do algoritmo proposto por \citeonline{wang1992generating}.

%A ideia central do método de Wang-Mendel é gerar regras fuzzy a partir de pares de dados numéricos, utilizando a modelagem das variáveis lingüísticas em conjuntos fuzzy, criando, assim, uma base comum de regras para o SBRF. Esse método é composto basicamente por 5 passos: 1) modelagem das variáveis linguísticas de entrada e de saída em conjuntos fuzzy e divisão dos dados em pares; 2) geração das regras a partir da modelagem do passo anterior; 3) Calculo de um grau para cada regra gerada; 4) remoção das regras conflitantes e 5) criação da base comum de regras fuzzy.

%A primeira etapa de modelagem das variáveis linguísticas é transforma-las em conjuntos fuzzy. Variáveis linguísticas podem ser "alto", "baixo", "medio", "neutro", "positivo", dentre outras. É nessa etapa que os projetistas do SBRF mapeiam essas variáveis em conjuntos fuzzy. Uma vez mapeadas as variáveis, os dados precisam ser divididos em pares, entrada e saída. Ao fim dessa etapa, têm-se um conjunto de dados numéricos, divididos em pares e mapeados para os conjuntos fuzzy. 

%A etapa seguinte, de geração de regras, é onde as primeiras regras são geradas. Cada par de dados origina uma regra fuzzy e essa regras são passadas para a próxima etapa. Na etapa 3 é onde se calcula os chamados graus de compatibilidade das regras. Nessa etapa, todas as regras geradas avaliam cada par de dados existentes, calculando o quanto compatível aquela regra é com o par de dados em questão. Após essa etapa, um valor numérico, chamado de grau de polaridade, é associado a cada regra avaliada  \cite{wang1992generating}.

%Uma vez geradas as regras e calculados os graus de compatibilidade, é preciso remover regras contraditórias e decidir entre regras iguais. É no quarto passo que é feita essa filtragem das regras. Na geração indiscriminada de regras, é comum surgirem regras que tenham antecedentes iguais e conseqüentes diferentes, como "Se a opinião é ALTA então ela é POSITIVA" e "Se a opinião é ALTA então ela é NEGATIVA". A decisão de qual regra utilizar é baseada no grau associado a cada regra. A que tiver o maior grau associado, é selecionada para a base final de regras. Além disso, regras repetidas (com antecedentes e conseqüentes idênticos) são também descartadas. Por fim, o último passo é a apresentação da base comum de regras geradas.

Existem vários modelos de sistemas fuzzy, sendo que a distinção entre eles se dá no consequente das regras. Entre os modelos mais conhecidos estão o Mamdani \cite{mamdani1975experiment}, que utiliza conjuntos fuzzy nos consequentes das regras, Takagi-Sugeno \cite{takagi1985fuzzy}, no qual o consequente é representado por uma função das variáveis de entrada, e o Tsukamot \cite{tsukamoto1979approach}, que utiliza funções de pertinência monotônicas no consequente. Dentre os poucos trabalhos encontrados que aplicam conceitos fuzzy em mineração de opinião, todos seguiram o modelo de Mamdani.

Para se determinar a saída de um SBRF é necessário definir um mecanismo de inferência fuzzy, que também é conhecido por Método de Raciocínio Fuzzy. Há dois métodos muito utilizados em sistemas de classificação fuzzy \cite{cintra2012genetic}, propostos por \citeonline{cordon1999proposal}, o Método de Raciocínio Fuzzy Geral (MRFG) e o Método de Raciocínio Fuzzy Clássico (MRFC).

O MRFC classifica um exemplo usando a regra que tem o maior grau de compatibilidade com o exemplo. Seja $e_p = (a_{p1}, a_{p2}, \cdots, a_{pm})$ um exemplo para ser classificado, e {$R_1,R_2, \cdots, R_s$} o conjunto de regras do sistema de classificação, cada uma com $m$ antecedentes e $A_{il_i}(a_{p_i}), i = 1, \cdots, m$, ser o grau de pertinência de uma característica $a_{p_i}$ para o conjunto fuzzy $i$ da regra $R_k$, o MRFC aplica os seguintes passos para classificar o exemplo $e_p$ \cite{cintra2012genetic}:

\begin{enumerate}
\item Calcula o grau de compatibilidade entre o exemplo $e_p$ e cada regra $R_k$, para $k = 1, \cdots, s$ e a norma \textbf{t}, através da equação \ref{eq:compat}:
\begin{equation}
Compat(R_k, e_p) = \textbf{t}(A_{1l_1}(a_{p_1}), A_{2l_2}(a_{p_2}), \cdots, A_{ml_m}(a_{p_m}))
\label{eq:compat}
\end{equation}
\item Encontra a regra $R_{kmax}$ que tem o maior grau de compatibilidade com o exemplo:
\begin{equation}
Compat(R_{kmax}, e_p) = max{Compat(R_k, e_p)}, k = 1,2, \cdots, s
\end{equation}
\item Associa a classe $c_j$ ao exemplo $e_p$, onde $c_j$ é a classe definida pela regra máxima $R_{kmax}$
\end{enumerate}

A figura \ref{fig:mrfc} ilustra esse processo.

\begin{figure}[H]  
  \caption{Método de Raciocínio Fuzzy Clássico (MRFC)}
  \label{fig:mrfc}
  \centering
    \includegraphics[width=0.7\textwidth]{mrfc}
\end{figure}

O MRFG, por outro lado, classifica um exemplo considerando todos os demais exemplos em relação as classes finais. Os seguintes passos mostram como MRFG funciona \cite{cintra2012genetic}:

\begin{enumerate}
\item De maneira similar ao MRFC, o MRFG calcula os graus de compatibilidade entre o exemplo $e_p$ e cada regra $R_k$, para $k = 1, \cdots, s$.
\item Calcula o valor de classificação $Class_c$, para cada classe. $Class_c$ é definida como a agregação dos graus de compatibilidade de todas as regras da classe $c_i$, e representa a compatibilidade do exemplo com todas as regras da classe $c_i$. Ele pode ser definido como:
\begin{equation}
Class_c = f{Compat(R_k, e_p)|c_i \, \acute{e}\, a\, classe\, da\, regra\, R_k}
\end{equation}
onde $f$ é o operador de agregação.
\item A classe com o maior grau é assinalada para o exemplo $e_p$.
\end{enumerate}

A figura \ref{fig:mrfg} ilustra esse processo.

\begin{figure}[H]  
  \caption{Método de Raciocínio Fuzzy Geral (MRFG)}
  \label{fig:mrfg}
  \centering
    \includegraphics[width=0.7\textwidth]{mrfg}
\end{figure}

\section{Trabalhos relacionados}

Nesta seção serão discutidos trabalhos representativos e relacionados sobre mineração de opinião. É importante ressaltar novamente que poucos trabalhos foram encontrados na literatura sobre a aplicação de Lógica Fuzzy e extração de características de documentos para classificação de opinião. 

Um dos primeiros trabalhos a se destacar na área de mineração de opinião foi o de \citeonline{turney2002thumbs}, apresentando uma abordagem de classificação não supervisionada. Similar à tarefa de classificar documentos como positivos ou negativos, \citeonline{turney2002thumbs} propôs classificar opiniões como "recomendadas" (\textit{thumbs up} ou "não recomendadas" (\textit{thumbs down}). A classificação de um documento contendo as opiniões é feita através da média da polaridade do sentimento geral das opiniões das palavras num documento que continha adjetivos e advérbios. \citeonline{turney2002thumbs} conseguiu resultados de até 74\% em média entre os domínios que utilizou em sua pesquisa.

Paralelamente, \citeonline{pang2002thumbs} foi um dos primeiros a apresentar uma abordagem que utiliza técnicas clássicas de aprendizado de máquina para mineração de opinião. Eles compararam o desempenho entre os métodos \textit{Naive Bayes}, \textit{Maximum Entropy} e \textit{Support Vector Machine} (SVM). Esse trabalho mostrou que tais métodos produzem altas taxas de acurácia, alcançando 82,9\% de acurácia usando palavras isoladas (unigrams) com o SVM. Ainda mostraram que técnicas de aprendizado supervisionado produzem resultados melhores que técnicas de aprendizado não supervisionado. Contudo, a proposta apresentada por \citeonline{pang2002thumbs} é dependente do domínio utilizado, produzindo resultados muito ruins em outros tipos de dados, demandando outras rodadas de treinamento do classificador, aumentando custo e tempo para classificar os documentos.

Um pouco mais relacionados a esta pesquisa, os trabalhos de \citeonline{wilson2005recognizing}, \citeonline{taboada2008extracting}, e \citeonline{ohana2009sentiment} usaram um grande número de características dos documentos. Essas características envolveram desde a contagem de adjetivos e advérbios em frases e no documento inteiro, tuplas de palavras (bigrams e trigrams), como advérbios e adjetivos combinados, até a soma das polaridades dos n-grams, dentre outros. \citeonline{taboada2008extracting} e \citeonline{ohana2009sentiment} foram um dos poucos a apresentar o uso de dicionários de opiniões, nesse caso, o Sentiwordnet \cite{esuli2006sentiwordnet} que associa às palavras valores numéricos referentes as polaridades opiniativas. Na classificação de documentos como positivos ou negativos, estes trabalhos apresentaram 65,7\% de acurácia em \citeonline{wilson2005recognizing}, 80,6\% em \citeonline{taboada2008extracting} e 69,35\% em \citeonline{ohana2009sentiment}.

Em relação a aplicação da lógica fuzzy, nós encontramos o trabalho de \citeonline{nadali2010sentiment}. Ele propõe um modelo de lógica fuzzy para executar classificação de opinião de clientes em cinco classes: muito fraca, fraca, moderada, muito forte e forte. Além disso, ele apresentou uma metodologia que indica o uso de sistemas de inferência, conjuntos fuzzy que modelam as cinco classes mencionadas e a criação manual de regras fuzzy. Contudo, o artigo de \citeonline{nadali2010sentiment} não apresentou resultados ou qualquer discussão a respeito.

Outro trabalho encontrado foi o de \citeonline{ballhysa2012fuzzy} que propõe uma abordagem fuzzy para descobrir opiniões em postagens de blogs, determinando a polaridade do sentimento geral da postagem. Os autores apresentam conceitos fuzzy, como conjuntos fuzzy, operações entre conjuntos fuzzy, propõem um conjunto de medidas fuzzy e uma agregação fuzzy dessas medidas, embora um sistema fuzzy de inferência não seja utilizado. Contudo, as medidas fuzzy propostas parecem corresponder a medidas não fuzzy (\textit{crip}), mostrando que há, de fato, aplicação própria da lógica fuzzy no trabalho. Além disso, existe uma superficial descrição dos resultados obtidos de um conjunto de dados criados por eles mesmos e sem comparação com nenhum outro.

\citeonline{bing:2012} foi um dos poucos a apresentar um proposta fuzzy com resultados claros. Ele utiliza a lógica fuzzy  juntamente com ontologias de domínio para minerar opiniões de avaliações de produtos. Este trabalho aborda o problema da sobrecarga de informações que a web possui e a impossibilidade de catalogar as informações manualmente. Frente a essa motivação, o trabalho de \citeonline{bing:2012} apresenta um nova proposta oriunda da junção de conjuntos fuzzy e ontologias de domínio que os autores chamam de árvore de sentimentos de ontologias de domínios fuzzy (tradução livre de \textit{fuzzy domain ontology sentiment tree} - FDOST). Essa proposta extrai atributos de produtos das opiniões, constrói uma arvore de relacionamentos entre eles e, utilizando conjuntos fuzzy, classifica e associa os sentimentos. Dois são os conjuntos fuzzy utilizados neste trabalho: um conjunto de palavras de sentimentos e outro de palavras especificas do domínio (neste caso, \textit{laptops}). Eles são utilizados para construir o modelo da ontologia do domínio que, por sua vez, será utilizado para construir as árvores de relacionamento entre os atributos dos produtos e os sentimentos associados. Os resultados obtidos mostram que a utilização dos conjuntos fuzzy teve melhor desempenho que a mesma proposta sem o uso deles. 

E em \citeonline{mouthami2013sentiment} um novo algoritmo, chamado de Classificação de Sentimentos Fuzzy (da tradução livre de \textit{Sentiment Fuzzy Classification}), é proposto para melhorar a precisão da classificação de sentimentos de opiniões, utilizando lógica fuzzy, POS-Tags (\textit{Part of Speech Tags}) e uma base de dados de filmes como base de avaliação. Além do uso de lógica fuzzy, este trabalho contribui em mapear as etapas básicas do processo de mineração de opiniões, baseando-se nas tarefas executadas nos trabalhos relacionados descritos na pequisa dos autores. Conjuntos fuzzy são utilizados na etapa de classificação, onde são definidos três deles: conjunto fuzzy positivo, negativo e neutro. O trabalho,contudo, não apresentou resultados. 

Foi recorrente encontrar artigos com propostas de conceitos fuzzy pra mineração de opinião que não apresentavam resultados os discutiam apropriadamente - em alguns casos a metodologia nem era bem definida. Essa pesquisa difere destes e de outros trabalhos relacionados por apresentar e aplicar apropriadamente conceitos e técnicas fuzzy em mineração de opinião. Nós modelamos as variáveis fuzzy e construímos um sistema de inferência fuzzy baseado em características dos documentos. Nós executamos ainda nossos testes em conjuntos de dados já utilizados nos trabalhos relacionados, permitindo comparações diretas. Além disso, nós propomos uma fase de extração e seleção de características pouco encontradas em outros trabalhos, onde definimos, extraímos e selecionamos um grande número de características dos documentos, baseando-se nos poucos trabalhos que já fizeram esse tipo de extração.

O próximo capítulo apresenta a metodologia utilizada nessa pesquisa, descrevendo cada etapa realizada e as técnicas relevantes usadas nessas etapas.

\ifcsdef{mainfile}{}{\bibliography{pesquisa}}

\end{document}
%
%\xchapter{Metodologia}{}

%O processo de mineração de opinião
%\begin{itemize}

%\item Preprocessing
%\begin{itemize}
%\item Definição
%\item Reiterar o tipo de análise escolhida: Document Level Analysis
%\item Datasets utilizados (Cornell e Amazon - descreve as caracteristicas de cada dataset, como quantidade, balanceamento, natureza, etc.)
%
%\item Descrever e justificar as tarefas envolvidas:
%\begin{itemize}
%\item O texto é dividido em sentenças
%\item POS Tagging - Brill's Tagger
%\item Blocos irrealis são filtrados ou não (atualmente NÃO FILTRAM)
%\item Defino quais ngrams serão extraídos
%\begin{itemize}
%\item Defino o uso do tipo de negação (FAR NEGATION - REFER)
%\item Os verbos foram recentemente removidos do universo
%\item Depois de testar, MANUALMENTE, diferentes configurações de n-grams, mantendo os demais parametros fixos, restaram adjetivos e adverbios, com unigrams, bigrams e trigrams
%\end{itemize}
%
%\item Retorna um vetor de n-grams (bag-of-words)
%\item Resume o subcapitulo e finaliza falando da saída dessa fase para a próxima
%\end{itemize}
%\end{itemize}
%\end{itemize}

O processo de mineração de opinião é comumente definido em seis etapas: a i) definição do domínio, ii) pré-processamento, iii) transformação, iv) seleção de características, v) classificação e vi) análise dos resultados \cite{moraes2012document}. Cada etapa inclui diferentes tarefas que, dado um documento na entrada do processo, seja possível classifica-lo e avaliar a classificação realizada.

\begin{figure}[h]
\caption{Etapas do processo de mineração de opinião.}
\centering
\includegraphics[scale=0.35]{opinion_mining_process.png}
\label{figura:processo_mineracao}
\end{figure}

\section{Definição do domínio e o pré-processamento dos dados}

Domínios diversos foram escolhidos para serem analisados por essa pesquisa, dentre eles filmes, livros, carros, computadores, panelas, hotéis, músicas, celulares, mp3, pen-drives, dispositivos gps, wifi e câmeras fotográficas. Essa diversidade de domínios é importante para que nós possamos corroborar nossa proposta de criar um classificador independente de domínio. Todas as bases de dados são da língua inglesa, pois os trabalhos relacionados a esta pesquisa utilizaram domínios nessa língua.

Para filmes, nós utilizamos a largamente utilizada \footnote{Vide https://www.cs.cornell.edu/people/pabo/movie-review-data/otherexperiments.html} base de dados, versão 2.0, desenvolvida e pré-classificada por \cite{pang2004sentimental}, com 2000 críticas de filmes, sendo metade positivas e outra metade negativas - bases assim, são conhecidas como balanceadas. As opiniões dos domínios de livros, carros, computadores, panelas, hotéis, músicas, celulares estão reunidos numa base de dados balanceada produzida por \cite{taboada2011lexicon}, retiradas do site Epinions \footnote{Vide http://www.epinions.com/}, com cerca de 400 críticas. E para as opiniões dos domínios de mp3, pen-drives, dispositivos gps, wifi e câmeras fotográficas, nós utilizamos uma base de dados balanceada de 2000 críticas retiradas do site da Amazon. 

Após a definição de um ou mais domínios é preciso, antes de inciar a etapa de pré-processamento, definir o nível da análise que será feita sobre os documentos. Existem três níveis básicos de análise de documentos em mineração de opinião: i) nível de análise de documento, ii) sentenças e iii) entidades e seus aspectos. O primeiro nível foca em classificar a opinião geral de um documento expressando-a como positiva ou negativa. O segundo nível, o de sentenças, em vez de considerar o sentimento geral das opiniões presentes em um documento como todo, classifica as opiniões de cada sentença separadamente. E o último nível foca em descobrir todos os alvos existentes nas sentenças do documento, e classifica as opiniões direcionadas a eles \cite{bing:2012}. Este trabalho decidiu utilizar o nível de análise de documento \cite{joachims1998text, pang2002thumbs, gamon2004sentiment, mullen2004sentiment, pang2004sentimental, cui2006comparative}.

A etapa de pré-processamento envolve tarefas como a tokenização dos documentos, marcação gramatical das palavras (do inglês, \textit{Part of Speech Tagging} ou POST), filtragem de sentenças com modais e definição os n-grams que serão utilizados para construir um modelo que represente o documento. A tokenização dos documentos é a tarefa que divide o conteúdo de cada documento em sentenças e, por sua vez, em palavras para que o marcador gramatical (ou \textit{tagger}) possa identificar as classes gramaticais das palavras do documento. O marcador gramatical usado foi o discutido em \cite{brill1995transformation} e usado em trabalhos relacionados a esta pesquisa \cite{chaovalit2005movie, taboada2008extracting, taboada2011lexicon}. A tarefa seguinte é a remoção de sentenças que possuem verbos modais. Segundo \cite{taboada2011lexicon}, modais como "would", "could", dentre outros, presentes numa sentença indicam que as palavras que aparecem juntamente com eles podem não ser confiáveis para serem usadas na definição do sentimento geral de opiniões de um documento. 

A tarefa final é definir quais n-grams serão usados para compor o modelo que representará o documento. O tipo de modelo utilizado nessa pesquisa é o popular saco de palavras (\textit{bag-of-words}), onde cada documento é representado por um vetor de termos (ou n-grams) do documento \cite{moraes2012document}. N-grams são termos que podem ser unigrams (uma palavra), bigrams (duas palavras) ou trigrams (três palavras). Nós definimos 5 tipos de n-grams: adjetivos e advérbios como unigrams; advérbios com adjetivos (e.g. \textit{very good}), advérbios com advérbios como bigrams; e a combinação de dois advérbios e um adjetivo como trigram (e.g. \textit{not very nice}) \cite{pang2002thumbs, turney2002thumbs, taboada2008extracting, karamibekr2012verb}. Nós também extraímos tipos especiais de bigrams e trigrams que são os n-grams negados (e.g. \textit{not bad}, \textit{nothing special}). A extração desses tipos de n-grams também é conhecido com detecção de negação e, por si só, é um linha de pesquisa completa, indo além do escopo deste trabalho. Nós utilizamos uma versão simplificada da técnica usada em \cite{taboada2011lexicon}. 

Ao fim do estágio de pré-processamento, cada documento é transformado num vetor de n-grams que é passado para a etapa de transformação. 

\section{Transformação}

%\item Transformation
%\begin{itemize}
%\item O que é
%\item Uso de dicionários de opinião para transformar n-grams e numeros
%\item Explicar porque o uso de dicionarios de opiniao é bom para nosso trabalho
%\item Sentiwordnet (falar sobre explicar, genericamente, como é criado, como é estruturado, valores associados aos termos)
%\item Falar do trabalho de Guerrine que propoe diferentes formulas para determinar a polaridade de um n-gram. E que são melhores que a mais frequente do SWN
%\item Falar e explicar os intensificadores
%\item Negação
%\item Frequencia de n-grams como transformador da polaridade
%\item Compensação de bias negativo
%\item Resume o subcapitulo e finaliza falando da saida dessa fase para a proxima 
%\end{itemize}

A etapa de transformação cria representações numéricas para os vetores de n-grams da etapa de pré-processamento, onde cada n-gram é associado a um valor numérico, um grau de polaridade opinativo, usando um dicionário de opiniões \cite{ballhysa2012fuzzy, moraes2012document, mouthami2013sentiment}. O dicionário base de opiniões usado nesse trabalho foi o SentiWordNet 3.0 \footnote{Disponível em: http://sentiwordnet.isti.cnr.it/. Download de 22/01/2013} \cite{baccianella2010sentiwordnet}.

\subsection{SentiWordNet 3.0}

O SentiWordNet 3.0 (SWN) é a terceira versão d SentiWordNet 1.0, apresentada por \cite{esuli2006sentiwordnet}. É um dicionário criado pela anotação automática de todos os conjuntos de sinônimos cognitivos, chamados de \textit{synsets}, de outro dicionário, o WordNet, versão 3.0 \cite{fellbaum2005wordnet}. Esses synsets são anotados de acordo sua positividade, negatividade e neutralidade. Cada \textit{synset} $s$ é associado a três valores numéricos $Pos(s)$, $Neg(s)$ e $Obj(s)$ que indicam o quanto positivos, negativos e "objetivos" (ou neutros) são os termos existentes no \textit{synset} \cite{baccianella2010sentiwordnet}.
Por exemplo, o synset [estimable (J, 3)] \footnote{Nós adotamos a convenção usada no WordNet e por \cite{esuli2006sentiwordnet} e \cite{baccianella2010sentiwordnet} que usa o termo entre colchetes para representar um synset; assim [poor (J, 7)] não só se refere ao termo "poor", mas ao synset formado pelos adjetivos {inadequate(J,2), poor(J,7), short(J,4)}}, correspondente ao sentido de “may be computed or estimated” do adjetivo "estimable", tem o grau de neutralidade $Obj(s)$ de 1.0 (e de positividade $Pos(s)$ e negatividade $Neg(s)$ igual a zero), enquanto que o synset [estimable (J, 1)], correspondente ao sentido de “deserving of respect or high regard”, tem o grau de positividade $Pos(s)$ de 0.75, $Neg(s)$ de 0.0 e $Obj(s)$ de 0.25. Cada um dos valores estão num intervalo $[0, 1]$ e a soma dos três é igual a 1. Assim, é possível que existam (e existem) \textit{synsets} que tenham todos os três graus diferentes de zero, indicando que há termos positivos, negativos e objetivos ao mesmo tempo \cite{esuli2006sentiwordnet}.

Inicialmente, nós definimos a polaridade resultante $Pol(n)$ de um termo (ou unigram) $n$, marcado gramaticalmente (e.g. adjetivo, advérbio, etc.), como $Pol(n) = Pos(s) - Neg(s)$ ,subtraindo o grau de positividade do grau de negatividade do synset $s$ mais frequente a que pertence o unigram. Por exemplo, o synset mais freqüente de [good (J,1)] somente possui o termo \textit{good}, com grau $Pos(s)$ de 0.75, $Neg(s)$ de 0.0 e $Obj(s)$ de 0.25. A polaridade resultante $Pol(n)$ será de 0.5. Contudo, essa abordagem, de utilizar o synset mais frequente, não é melhor que escolher, aleatoriamente, um dos synsets. Além disso, o uso de polaridades de palavras fora de contexto (do inglês \textit{word’s prior polarity}), típicas de abordagens \textit{bag-of-words}, deve ter um tratamento diferente para se adequar ao problema da perda do contexto causada na etapa de pré-processamento, com a tokenização do documento  \cite{guerini2013sentiment}. Assim, nós utilizamos um dicionário derivado do SWN produzido por  \cite{guerini2013sentiment} com todas as polaridades fora de contexto calculadas de todos os termos do SentiWordNet 3.0. A polaridade resultante $Pol(n)$ agora é a polaridade fora de contexto do unigram. 

O uso da abordagem de polaridades fora de contexto tem a vantagem de não depender de profunda análise semântica ou desambiguação (uma linha de pesquisa completa, por si só) para assinalar um valor opinativo para um termo, além de ser independente de domínio, como proposto nesse trabalho \cite{guerini2013sentiment}. Além disso, dicionários construídos automaticamente, como o SWN e a variação de Guerrine, superam problemas existentes de dicionários criados manualmente, como em \cite{taboada2008extracting, taboada2011lexicon} que tendem a se restringir a um número muito menor de termos, consomem tempo para serem construídos e podem sofrer enviesamento de seus autores \cite{ohana2009sentiment}. Outra vantagem de usar técnicas baseadas em dicionários é que dispensa o uso de grandes bases de dados para aprendizado ou motores de buscas com funcionalidades especiais para a tarefa \cite{khan2011sentiment}. A principal desvantagem, todavia, é que a fase de  transformação depende inteiramente da cobertura do dicionário utilizado \cite{khan2011sentiment}. 

\subsection{Definição da polaridade}

\subsubsection{Unigrams}

Dado um unigram do vetor de n-grams de um documento, a polaridade deste é definida pela polaridade fora de contexto provida pelo dicionário de opiniões. Caso o unigram não seja encontrado no dicionário, o lema deste é utilizado. Se ainda assim, o dicionário não cobrir o unigram em questão, ele é descartado do processo.
Nós também analisamos a ocorrência do unigram no documento. A enésima ocorrência $n$ de um unigram no documento terá somente $1/n$ da polaridade original retirada do dicionário de opiniões. Por exemplo:

\textit{Overall, the movie was great. The acting was great, the history was great, and the direction was just plain great.}

A repetição da palavra \textit{great} sugere que a pessoa que escreveu a crítica acima não tinha um termo específico que expressasse outras opiniões e utilizou \textit{great} de maneira genérica. Outra razão para suavizar a polaridade de unigrams que se repetem é que uma palavra que aparece regularmente tem a mais probabilidade de ser um termo neutro \cite{taboada2011lexicon}. 

Outra análise relacionada a definição da polaridade de um unigram, foi o enviesamento positivo. Classificadores que utilizam dicionários de opiniões geralmente mostram uma tendência para classificar o sentimento geral de um documento para a positividade, devido a tendência natural do ser humano de utilizar linguagem positiva e evitar termos negativos \cite{boucher1969pollyanna, kennedy2006sentiment}. Para tentar dirimir esse enviesamento, nós utilizamos uma abordagem proposta por \cite{taboada2011lexicon} que consiste em aumentar em 50\% a polaridade de todo unigram negativo. 

\subsubsection{Bigrams e trigrams}

O dicionário de opiniões provê a polaridade somente para unigrams. Para bigrams e trigrams, tivemos de considerar a influência entre as palavras desses n-grams. Conforme vimos na etapa de pré-processamento, os bigrams e trigrams propostos nesse trabalho são compostos por advérbios e adjetivos ou somente por advérbios. Os advérbios que antecedem os adjetivos ou advérbios nos n-grams são chamados de modificadores, intensificadores ou atenuadores, pois alteram a polaridade das palavras que os acompanham \cite{voll2007not}. Os intensificadores aumentam o grau da polaridade do unigram, seja ele positivo ou negativo, e os atenuadores diminuem. Por exemplo, 

\begin{itemize}
\item \label{itm:very_exem} $Pos(\textit{good}) = 0,72259$; $Pos(\textit{very good}) = 0,90323$
\item \label{itm:really_exem} $Neg(\textit{bad}) = -0.44006$; $Neg(\textit{really bad}) = -0,5060$
\end{itemize}

Os advérbios \textit{very} e \textit{really} modificam as polaridades de \textit{good} e \textit{bad}. Diferentemente de dicionários de opiniões, como o SWN, não conseguimos encontrar, até o presente momento da pesquisa, nenhuma fonte disponível de intensificadores e amenizadores. Diferentes autores como \cite{voll2007not}, \cite{taboada2008extracting}, \cite{taboada2011lexicon} e \cite{pimpalkar2013sentimental} citam alguns modificadores ou até mencionam que construíram, manualmente, dicionários próprios de modificadores, mas não disponibilizaram tais recursos para serem reproduzidos nessa pesquisa. Assim, tivemos de construir nosso próprio dicionário de intensificadores e amenizadores. 

Além do problema da não disponibilidade de dicionários de modificadores, também não há consenso de como a intensificação (ou amenização) das polaridades dos adjetivos pelos advérbios é feita. Nós decidimos nos basear no trabalho de \cite{taboada2011lexicon} que acrescenta ou diminui um percentual da polaridade dos adjetivos, a depender das classes dos advérbios. \citeauthoronline{taboada2011lexicon} define uma lista inicial (\ref{table:adv_seed}) de advérbios para criar seu dicionário e nós usamos a mesma lista para criar o nosso.

\begin{table}[!h]
	\centering
    \begin{tabular}{lll}
    Advérbio         				& Classe          & Percentual modificador \\ \hline
    pretty                   			& LOW 			   & -10\% \\
    somewhat                   	& VERY LOW  & -30\% \\
    slightly                   		& LOWEST 	   & -50\% \\
    really                   			& HIGH 			   & 15\% \\
    very                   			& VERY HIGH &  25\% \\
    extraordinarily             & HIGHEST 	   & 50\% \\
    most                   			& MOST HIGHEST & 100\% \\
    \end{tabular}
    \caption{Lista inicial de advérbios retirada de \cite{taboada2011lexicon}}
	\label{table:adv_seed}
\end{table}

Diferentemente do que foi feito em \citeauthoronline{taboada2011lexicon}, nós construímos o dicionário de modificadores automaticamente. Extraímos todos os advérbios existentes do Wordnet e, calculando a distância entre eles (os advérbios são organizados num grafo no Wordnet) e os advérbios da lista inicial, os associamos aos advérbios iniciais. Por exemplo, o advérbio \textit{truly} é mais próximo de \textit{very}, da classe VERY HIGH, do que o advérbio \textit{really}, assim ele é associado à classe de \textit{very}. Um outro exemplo é \textit{fairly} que é mais próximo de \textit{somewhat} que do advérbio \textit{pretty}. A partir daí o cálculo da polaridade $Pol$ de um bigram $b$ formado por um modificador $m$ e um unigram $u$ pode ser calculados assim: 

\begin{equation}
Pol(b) = Pol(u) + Pol(u) \cdot Perc(m)
\label{eq:pol_bigrams}
\end{equation}

Onde $Perc(m)$ é o percentual de intensificação ou de amenização do modificador $m$. O cálculo da polaridade de um trigram $t$ é análogo: primeiro calcula-se a polaridade do bigram mais interno e em seguida aplica-se o modificador mais externo à polaridade do bigram, conforme a equação \ref{eq:pol_trigrams}.

\begin{equation} 
Pol(t) = Pol(b) + Pol(u) \cdot Perc(m)
\label{eq:pol_trigrams}
\end{equation}.

Existe ainda um tipo especial de n-gram que é a negação. São todos os n-grams formados por advérbios de negação e por adjetivos ou por advérbios (e.g. \textit{not bad} ou \textit{not very good}). A abordagem inicial é inverter a polaridade do n-gram, mudando, por exemplo, $Pos(\textit{good}) = 0,72259$ para $Pos(\textit{not good}) = - 0,72259$. Essa abordagem é conhecida como interruptor da negação ou inversão \cite{sauri2008factuality}. Contudo, embora a negação seja um efeito moderadamente local, é preciso analisar o efeito de palavras negativas, como \textit{none}, \textit{nobody} e \textit{nothing}, que tem efeito equivalente à negação local \cite{taboada2011lexicon} mas que atuam numa distância maior. Por exemplo:

\begin{example}
\textit{Nobody gives a good performance in this movie.} (\textit{nobody} nega \textit{good})
\label{ex:far_neg_1}
\end{example}

\begin{example}
\textit{Out of every one of the fourteen tracks, none of them approach being weak and are all stellar.} (\textit{none} nega \textit{weak})
\label{ex:far_neg_2}
\end{example}

Este trabalho utiliza uma versão simplificada da abordagem de \cite{das2001yahoo} para tentar englobar o efeito de maior distância da negação. Daí, ainda na etapa de pré-processamento, quando uma palavra de negação \footnote{As palavras de negação são: \textit{nobody}, \textit{none}, \textit{nothing}, \textit{not}, \textit{don't}, \textit{no}, \textit{doesn't}, \textit{isn't}, \textit{aren't}} era encontrada, todos os termos seguintes eram negados (eram associadas ao advérbio \textit{not}), até ser encontrado um ponto de fim de sentença. Se os n-grams criados se encaixassem nos tipos de n-grams da etapa de pré-processamento, eram passados para a etapa de transformação. 

A distância do efeito da negação também implica em como calcular a polaridade de um n-gram negado. Embora a inversão da polaridade funcione em certos casos como \textit{good} e \textit{not good} \cite{choi2008learning}, ele falha muito em outros \cite{liu2009review}. Considere o adjetivo positivo \textit{awesome} com polaridade $Pos(s) = 0.71506 $: se aplicarmos a inversão de polaridade, nós temos \textit{not awesome}, que é intuitivamente menos negativo que $Neg(\textit{terrible}) = -0.72503$. De fato, \textit{not awesome} parece ser mais positivo que \textit{not fine}, por exemplo, que aplicando a inversão, temos -0.37506. Para também englobar esse efeito da negação, nós usamos a abordagem proposta por \cite{taboada2011lexicon} chamada de \textit{shift negation}. Em vez de inverter os sinais da polaridade, o grau de opinião do n-gram é deslocado em direção à polaridade oposta por um valor fixo (em nossa implementacão, 0.75). Por exemplo:

\begin{example}
\textit{She is not amazing} (0,68 - 0,75 = -0,06) \textit{but not terrible} (-0,72503 + 0,75 = 0,02497) \textit{either}.
\label{ex:shift_1}
\end{example}

A negação de um termo fortemente positivo ou negativo reflete a mistura de opiniões que é corretamente capturada pelo deslocamento da negação \cite{taboada2011lexicon}. Depois de unigrams, bigrams e trigrams terem sido transformados para suas polaridades correspondentes, ao fim da etapa de transformação, os vetores de n-grams da entrada dessa etapa são transformados em vetores numéricos de polaridades.  

\section{Extração e seleção de características}

A etapa de seleção de características é comumente encontrada em abordagens de mineração de opiniões. A ideia principal é fazer com que a etapa de classificação seja mais eficiente/efetiva, reduzindo a quantidade de características dos documentos a serem analisadas e identificando quais características são mais relevantes para a etapa de classificação \cite{moraes2012document}. Diferentemente da maioria dos trabalhos relacionados que consideram os n-grams como as características dos documentos, este trabalho realiza uma sub-etapa de extração de novas características utilizando as polaridades dos n-grams da etapa de transformação. Essa sub-etapa é a extração de características.

\subsection{Extração de características}

Nessa sub-etapa, nós extraímos características dos documentos dos vetores de polaridades da etapa de transformação. Nós decidimos usar essa abordagem, pois é efetiva em analisar os documentos por suas características em vez de seu conteúdo específico para decidir a orientação semântica destes. As características que utilizamos não são específicas para nenhum domínio e podem ser facilmente aplicadas para quaisquer outros tipos de base de dados \cite{pang2002thumbs}. Além disso, a extração dessas características reduz a dimensionalidade dos dados, já que os vetores resultantes da etapa é significativamente menor que um vetor comum de saco de palavras.

Por outro lado, abordagens de mineração de opinião que são dependentes de domínio, normalmente baseadas em métodos de aprendizado de máquina, conseguem taxas de acurácia iguais ou maiores que 95\%, pois alimentam o classificador com os vetores de palavras dos documentos. Assim, o classificador aprende com os documentos da base de dados quais palavras fazem parte dos documentos positivos e negativos. Contudo, para alcançar altas taxas de classificação, essa abordagens precisam de uma grande massa de dados anotadas para treinamento e considerável tempo de treino para poder classificar corretamente. Todavia, ainda, produzem resultados muito ruins entre domínios diferentes sem novas rodadas de treinamento.

Diferentes estudos propuseram diferentes características para descrever ou discriminar documentos entre eles para identificar suas polaridades \cite{wilson2005recognizing, ohana2009sentiment, taboada2011lexicon}. Com o fim de capturar diversos aspectos dos documentos, nós decidimos extrais um grande número de características. Assim, nós usamos as características apresentadas nesses trabalhos e criamos nossas próprias, derivando destas, resultando em 57 caraterísticas diferentes.

Três tipos de características foram definidas: somatório, contagem e valores máximos. Características de somatório envolvem a soma as polaridades dos diferentes tipos de n-grams, como a soma dos adjetivos de um documento e a soma dos advérbios. As características de contagem são criadas da mesma maneira para os n-grams selecionados, criando características como a quantidade de bigrams positivos formados por advérbios e adjetivos. As características de valores máximos se referem aos valores para um dado n-gram no documento. Por exemplo, se o valor máximo absoluto entre os unigrams de um documento for positivo, essa característica tem valor igual a 1. Por outro lado, se o valor máximo for negativo, a característica tem valor -1. Mais características foram derivadas dos três tipos que acaram de ser descritos, aplicando normalização e subtração entre elas. Por exemplo, a diferença entre bigrams positivos e negativos de um documento ou o somatório normalizado de adjetivos positivos são uma das características derivadas por este trabalho. A lista completa de características pode ser encontrada no apêndice \ref{sec:appendix1}.

Depois da extração de características, os vetores de polaridades da etapa de transformação são substituídos por vetores de características. Cada documento, agora, é representado por um vetor com 67 caraterísticas diferentes. 

\subsection{A seleção de características}

Depois de extraídas as características dos documentos faz-se necessário selecionar quais delas são mais aptas para representar os documentos do corpus escolhido. Além de reduzir quantidade de características a serem analisadas e tornar a etapa de classificação mais eficiente/efetiva, também reduz a dimensionalidade do vetor de características. Para fazer essa seleção, nós usamos dois algoritmos de seleção de características, o \textit{Correlation - based Feature Selection} (CFS) e o algoritmo de seleção do algoritmo C4.5 \cite{cintra2008fuzzy}. \citeonline{cintra2008fuzzy} apresenta o uso destes algoritmos de seleção de características no âmbito da lógica fuzzy e de regras fuzzy, conceitos chave do nosso trabalho. O CFS avalia subconjuntos de características que são altamente correlacionadas com a classificação, mas que são, ao mesmo tempo, não relacionadas entre si \cite{hall1999correlation}. O C4.5, por outro lado, é um algoritmo que gera uma árvore de decisão que é normalmente usada para a tarefa de classificação \cite{quinlan19934}. Todavia, para construir essa árvore de decisão, esse algoritmo precisa selecionar as melhores características entre as existentes. Daí, nós usamos o C4.5 também como um algoritmo de seleção para nossa tarefa de minerar opinião.

Após a seleção das características mais aptas para representar os documentos do corpus escolhido, estas são passadas para a etapa de classificação. 

\section{Classificação}

\begin{itemize}
\item Definição
\item Construção das regras
\begin{itemize}
\item Outliers e a regra dos 3 sigmas
\item Definição dos conjuntos fuzzy
\item Tipos dos conjuntos fuzzy
\item Quantidade (entrada e saida)
\item Uniformes vs Não uniformes
\end{itemize}
\item Metodo de criacao de regras Wang Mende (explicar bem detalhadamente o metodo)
\item Metodos de classificacao: CFRM e GFRM
\end{itemize}

A etapa de classificação é onde nós construímos nosso sistema baseado em regras fuzzy com o fim de predizer o sentimento geral das opiniões, positivo ou negativo, de cada documento da base dados. Para construir um classificador assim, algumas tarefas estão envolvidas, como a criação de um conjunto de regras baseadas nas características selecionadas previamente e a modelagem das características em variáveis lingüísticas com conjuntos fuzzy.

[CONTINUA]

\section{Avaliação}


%
%\documentclass[template.tex]{subfiles}
\begin{document}

\xchapter{Resultados obtidos}{}

Nessa seção são descritos e discutidos os experimentos realizados e os resultados obtidos desses experimentos. O objetivo é não só comparar qual cenário obtém melhor \textit{acuracia} ou TPR e TNR, mas também discutir em quais contextos os classificadores produzem melhores ou piores resultados.

%\section{Datasets}
\todo[inline]{Pelo que vi nos comentários da seção de metodologia essa seção deve sumir daqui e subir para para metodologia, certo?}
\todo[inline]{Isso mesmo, vai para metodologia, mas cuidado com duplicações de conteúdo lá}
\todo[inline]{matheus: Já tinha escrito bem na metodologia então somente removi. Essa subseção se foi}

%\section{Resultados experimentais}

Nós focamos em comparar os métodos de seleção de características e de inferência, variando as configurações em diferentes etapas do processo de mineração de opinião, comparando acurácia, TPR e TNR. Nós avaliamos a influência dos algoritmos de seleção de características, os sistemas de inferência fuzzy em si, o uso de pesos na regras, a quantidade usada de conjuntos fuzzy, a eficiência das regras entre domínios e usadas em outro,características mais selecionadas entre as bases utilizadas. 

\todo[inline]{usa folds ao inves de dobras, já troquei no parágrafo abaixo e refiz redação também}
\todo[inline]{matheus: ok}

\todo[inline]{matheus: esse paragrafo ja tem na metodologia. Daí eu eliminei e juntei com o paragrafo do inicio do capitulo e eliminei essa seção também. O texto eliminado está comentádo no latex.}
%Para cada base de dados, o processo de mineração de opinião é idêntico para as etapas de pré-processamento, transformação e extração de características. Aplicando validação cruzada de 10 folds, as 9 partes da base são utilizadas para treinamento são utilizadas para seleção de características, na modelagem dos conjuntos fuzzy e na construção da base de regras fuzzy. A parte restante é utilizada somente para teste, realizando a seleção da mesmas características escolhidas durante o treino e fornecendo os valores destas para o sistema fuzzy realizar a classificação. O mesmo processo é repetido para cada fold e os resultados, para todas as medidas usadas, são a média dos valores obtidos em cada fold de teste. 

\todo[inline]{falta mais discussão nas seções seguintes, apresente mais informações sobre o que ocorreu, e discuta os resultados a luz do comportamento do sistema}

Nossos resultados são fruto de uma uma extensa quantidade de combinações nas duas bases de dados, filmes e Amazon, entre os algoritmos de seleção de características, os tipos sistemas de inferência fuzzy (utilizando pesos na regras ou não) e a quantidade usada de conjuntos fuzzy para modelar as variáveis de entrada. É importante observar que o algoritmo c4.5 é mostrado com duas variações: altura 1 e 2. Deixando o algoritmo construir livremente a árvore do decisão para a seleção de características, em dois cenários diferentes, nós escolhemos utilizar as características que estavam nos nós das árvores até altura 1 e 2. Isso resultou na mudança da quantidade de características selecionadas nos dois cenários. Fizemos isso para analisar também em quanto a altura da árvore de decisão do c4.5 influenciaria nos resultados. Vale notar que fizemos até altura 3, mas as características selecionadas eram as mesmas para altura 2. 
As tabelas \ref{table:movies_3f} e \ref{table:movies_2f} mostram os resultados na base de filmes e as tabelas \ref{table:amazon_3f} e \ref{table:amazon_2f} na base mista da Amazon. Primeiramente vamos discutir os cenários para o uso de 3 conjuntos fuzzy e depois para o uso de somente 2.

\begin{table}[H]
\begin{tabular}{ @{} c*{11}c @{} }
	\rot{CFS} & \rot{C4.5 - Altura 1} & \rot{C4.5 - Altura 2} & \rot{MRFG} & \rot{MRFG C/ PESOS} & \rot{MRFC} & \rot{MRFC C/ PESOS} & \rot{ACURÁCIA} & \rot{TNR} & \rot{TPR}  \\ \hline
	X &  &  & X &  &  &  & 62.9\% +- 4.25\% & 63.1\% +- 22.46\% & 62.7\% & 22.69\% \\ \hline
	X &  &  &  & X &  &  & 67.1\% +- 3.39\% & 72.6\% +- 8.77\% & 61.6\% & 9.77\% \\ \hline
	X &  &  &  &  & X &  & 52.25\% +- 4.92\% & 40.3\% +- 32.94\% & 64.2\% & 30.55\% \\ \hline
	X &  &  &  &  &  & X & 56.2\% +- 5.16\% & 55.1\% +- 19.21\% & 57.3\% & 14.26\% \\ \hline
	 & X &  & X &  &  &  & 59.2\% +- 1.83\% & 53.8\% +- 34.96\% & 64.6\% & 37.08\% \\ \hline
	 & X &  &  & X &  &  & 70.05\% +- 0.04 & 70.4\% +- 7.11\% & 69.7\% & 9.81\% \\ \hline
	 & X &  &  &  & X &  & 54.4\% +- 1.72\% & 47.1\% +- 42.67\% & 61.7\% & 43.93\% \\ \hline
	 & X &  &  &  &  & X & 69.8\% +- 4.03\% & 69.3\% +- 10.15\% & 70.3\% & 11.73\% \\ \hline
	 &  & X & X &  &  &  & 59.65\% +- 2.42\% & 60.3\% +- 35.78\% & 0.59 & 36.9\% \\ \hline
	 &  & X &  & X &  &  & 70\% +- 3.96\% & 70.3\% +- 7.02\% & 69.7\% & 9.81\% \\ \hline
	 &  & X &  &  & X &  & 54.5\% +- 1.76\% & 55.6\% +- 43.49\% & 53.4\% & 44.5\% \\ \hline
	 &  & X &  &  &  & X & 69.8\% +- 4.03\% & 69.3\% +- 10.15\% & 70.3\% & 11.73\% \\ \hline
\end{tabular}
\caption{Resultados da base de filmes, utilizando 3 conjuntos fuzzy nas variáveis de entrada}
\label{table:movies_3f}
\end{table}
	
\begin{table}[H]
\begin{tabular}{ @{} c*{11}c @{} }
	\rot{CFS} & \rot{C4.5 - Altura 1} & \rot{C4.5 - Altura 2} & \rot{MRFG} & \rot{MRFG C/ PESOS} & \rot{MRFC} & \rot{MRFC C/ PESOS} & \rot{ACURÁCIA} & \rot{TNR} & \rot{TPR} \\ \hline
	X &  &  & X &  &  &  & 67.35\% +- 6.36\% & 57.8\% +- 11.83\% & 76.9\% +- 11.39\% \\ \hline
	X &  &  &  & X &  &  & 68.85\% +- 6.66\% & 62.4\% +- 9.72\% & 75.3\% +- 11.61\% \\ \hline
	X &  &  &  &  & X &  & 64.4\% +- 8.12\% & 51.3\% +- 13.56\% & 77.5\% +- 13.90\% \\ \hline
	X &  &  &  &  &  & X & 67.55\% +- 6.14\% & 58.2\% +- 10.04\% & 76.9\% +- 11.51\% \\ \hline
	 & X &  & X &  &  &  & 60.05\% +- 2.37\% & 44.6\% +- 35.73\% & 75.5\% +- 34.8\% \\ \hline
	 & X &  &  & X &  &  & 70.85\% +- 3.09\% & 76.8\% +- 4.57\% & 64.9\% +- 5.5\% \\ \hline
	 & X &  &  &  & X &  & 54.25\% +- 2.82\% & 34\% +- 43.26\% & 74.5\% +- 38.82\% \\ \hline
	 & X &  &  &  &  & X & 70.55\% +- 3.12\% & 75.2\% +- 5.68\% & 65.9\% +- 6.25\% \\ \hline
	 &  & X & X &  &  &  & 61.75\% +- 4.2\% & 79.4\% +- 29.30\% & 44.1\% +- 31.92\% \\ \hline
	 &  & X &  & X &  &  & 68.85\% +- 3.27\% & 85.4\% +- 8.27\% & 52.3\% +- 12.82\% \\ \hline
	 &  & X &  &  & X &  & 60.4\% +- 5.75\% & 73.7\% +- 35.86\% & 47.1\% +- 34.22\% \\ \hline
	 &  & X &  &  &  & X & 68.15\% +- 2.84\% & 80.4\% +- 8.91\% & 55.9\% +- 12.87\% \\ \hline
\end{tabular}
\caption{Resultados da base da Amazon, utilizando 3 conjuntos fuzzy nas variáveis de entrada}
\label{table:amazon_3f}
\end{table}

\begin{table}[H]
\begin{tabular}{ @{} c*{11}c @{} }
	\rot{CFS} & \rot{C4.5 - Altura 1} & \rot{C4.5 - Altura 2} & \rot{MRFG} & \rot{MRFG C/ PESOS} & \rot{MRFC} & \rot{MRFC C/ PESOS} & \rot{ACURÁCIA} & \rot{TNR} & \rot{TPR}  \\ \hline
	X &  &  & X &  &  &  & 68.60\% +- 7.6\% & 74.9\% +- 9.52\% & 62.3\% +- 13.84\% \\ \hline
	X &  &  &  & X &  &  & 69.05\% +- 7.59\% & 68.4\% +- 24.26\% & 69.7\% +- 13.15\% \\ \hline
	X &  &  &  &  & X &  & 66.75\% +- 7.97\% & 59.7\% +- 17.60\% & 73.8\% +- 7.79\% \\ \hline
	X &  &  &  &  &  & X & 67.5\% +- 6.73\% & 63.8\% +- 22.45\% & 71.2\% +- 12.63\% \\ \hline
	 & X &  & X &  &  &  & 70.55\% +- 3.40\% & 71.4\% +- 5.71\% & 69.7\% +- 6.03\% \\ \hline
	 & X &  &  & X &  &  & 70.9\% +- 3.07\% & 71.2\% +- 4.33\% & 70.6\% +- 3.35\% \\ \hline
	 & X &  &  &  & X &  & 70.55\% +- 3.40\% & 71.4\% +- 5.71\% & 69.7\% +- 6.03\% \\ \hline
	 & X &  &  &  &  & X & 70.9\% +- 3.07\% & 71.2\% +- 4.33\% & 70.6\% +- 3.35\% \\ \hline
	 &  & X & X &  &  &  & 70.55\% +- 3.4\% & 71.2\% +- 5.54\% & 69.9\% +- 6.04\% \\ \hline
	 &  & X &  & X &  &  & 69.65\% +- 4.12\% & 0.67 +- 12.74\% & 72.3\% +- 6.98\% \\ \hline
	 &  & X &  &  & X &  & 70.55\% +- 3.4\% & 71.4\% +- 5.71\% & 69.7\% +- 6.03\% \\ \hline
	 &  & X &  &  &  & X & 70.9\% +- 3.07\% & 71.2\% +- 4.33\% & 70.6\% +- 3.35\% \\ \hline
\end{tabular}
\caption{Resultados da base de filmes, utilizando 2 conjuntos fuzzy nas variáveis de entrada}
\label{table:movies_2f}
\end{table}

\begin{table}[H]
\begin{tabular}{ @{} c*{11}c @{} }
	\rot{CFS} & \rot{C4.5 - Altura 1} & \rot{C4.5 - Altura 2} & \rot{MRFG} & \rot{MRFG C/ PESOS} & \rot{MRFC} & \rot{MRFC C/ PESOS} & \rot{ACURÁCIA} & \rot{TNR} & \rot{TPR} \\ \hline
	X &  &  & X &  &  &  & 71.9\% +- 3.63\% & 69.2\% +- 9.80\% & 74.6\% +- 10.49\% \\ \hline
	X &  &  &  & X &  &  & 72.4\% +- 2.36\% & 68.5\% +- 6.62\% & 76.3\% +- 7.08\% \\ \hline
	X &  &  &  &  & X &  & 70.75\% +- 2.45\% & 73.2\% +- 8.71\% & 68.3\% +- 7.62\% \\ \hline
	X &  &  &  &  &  & X & 71.5\% +- 3.04\% & 71.4\% +- 6.45\% & 71.6\% +- 7.43\% \\ \hline
	 & X &  & X &  &  &  & 70.7\% +- 3.19\% & 77.6\% +- 2.49\% & 63.8\% +- 4.99\% \\ \hline
	 & X &  &  & X &  &  & 70.8\% +- 3.27\% & 77.2\% +- 3.54\% & 64.4\% +- 3.46\% \\ \hline
	 & X &  &  &  & X &  & 70.7\% +- 3.19\% & 77.6\% +- 2.49\% & 63.8\% +- 4.99\% \\ \hline
	 & X &  &  &  &  & X & 70.8\% +- 3.27\% & 77.2\% +- 3.54\% & 64.4\% +- 3.46\% \\ \hline
	 &  & X & X &  &  &  & 68.7\% +- 3.10\% & 86.9\% +- 6.56\% & 50.5\% +- 10.40\% \\ \hline
	 &  & X &  & X &  &  & 66.9\% +- 4.12\% & 88.9\% +- 8.50\% & 44.9\% +- 15.18\% \\ \hline
	 &  & X &  &  & X &  & 71.3\% +- 2.98\% & 76.4\% +- 3.41\% & 66.2\% +- 5.6\% \\ \hline
	 &  & X &  &  &  & X & 70.85\% +- 2.68\% & 83.5\% +- 4.96\% & 58.2\% +- 5.81\% \\ \hline
\end{tabular}
\caption{Resultados da base da Amazon, utilizando 2 conjuntos fuzzy nas variáveis de entrada}
\label{table:amazon_2f}
\end{table}

\section{Avaliação com o uso de 3 conjuntos fuzzy}

\subsection{Avaliação dos algoritmos de seleção de características}

Analisando as tabelas \ref{table:movies_3f} e \ref{table:amazon_3f}, é possível perceber que, em ambos os casos, o melhor resultado de acurácia do c4.5 (70.05\% em filmes e 70.85\% na Amazon), com altura 1, é maior que o melhor resultado produzido de acurácia pelo CFS (67.1\% me filmes e 68.85\% na Amazon). Todavia, a diferença entre eles não é significativa (Wilcoxon, $p<=0.01$). É importante observar, contudo, que o CFS utiliza, em média, quase 6 características para produzir as regras e classificar os documentos, criando mais regras e menos legíveis. O c4.5, com altura 1, por outro lado, utiliza somente duas características para que o classificador realize a mesma tarefa com desempenho equivalente. 

A equivalência dos resultados ocorre, pois o CFS tem uma cobertura maior do espaço de dados, já que tem mais características no sistema, diferentemente do c4.5. A importância dessa cobertura é melhor vista nos demais resultados do CFS que mostram desvios padrão menores,  na média, que os desvios apresentados pelo c4.5, principalmente quando não são utilizados os pesos (discutidos mais adiante). A tabela \ref{table:movie_folds} mostra um exemplo da oscilação na classificação entre os folds que resultam nos altos valores de desvio padrão.

\begin{table}[H]
	\centering
    \begin{tabular}{ll}
    Fold 0 \\ \hline
    Accuracy &  55.0\% \\
	TPR &  100.0\% \\
	TNR &  10.0\% \\ \\
	
	Fold 1 \\ \hline
    Accuracy &  52.5\% \\
	TPR &  5.0\% \\
	TNR &  100.0\% \\ \\
	
	Fold 2 \\ \hline
    Accuracy &  54.5\% \\
	TPR &  96.0\% \\
	TNR &  13.0\% \\ \\
	
	Fold 3 \\ \hline
    Accuracy &  52.0\% \\
	TPR &  98.0\% \\
	TNR &  6.0\% \\ \\
	
	Fold 4 \\ \hline
    Accuracy &  55.5\% \\
	TPR &  99.0\% \\
	TNR &  12.0\% \\ \\
	
	Fold 5 \\ \hline
    Accuracy &  51.5\% \\
	TPR &  4.0\% \\
	TNR &  99.0\% \\ \\
	
	Fold 6 \\ \hline
    Accuracy &  54.5\% \\
	TPR &  9.0\% \\
	TNR &  100.0\% \\ \\
	
	Fold 7 \\ \hline
    Accuracy &  57.0\% \\
	TPR &  97.0\% \\
	TNR &  17.0\% \\ \\
	
	Fold 8 \\ \hline
    Accuracy &  56.0\% \\
	TPR &  14.0\% \\
	TNR &  98.0\% \\ \\
	
	Fold 9 \\ \hline
    Accuracy &  55\% \\
	TPR &  95.0\% \\
	TNR &  16.0\% \\
    \end{tabular}
    \caption{Problema do desbalanceamento entre TPR e TNR - base de filmes}
    \label{table:movie_folds}
\end{table}

É importante frisar que os folds foram definidos por amostragem estratificada, mantendo a mesma proporção de documentos positivos e negativos do base de dados total. Dentre as características selecionadas por CFS e c4.5 duas se destacaram, sendo frequentemente selecionadas dentro dos cenários testados por ambos algoritmos. São elas:

\begin{table}
\resizebox*{\textwidth}{!}{ \begin{tabular}{|l|}
	\hline
	A diferença entre as somas positiva e negativa de adjetivos e bigrams compostos estritamente por advérbio e adjetivo; \\
	\hline
	E a diferença entre as somas positiva e negativa de unigrams e bigrams combinados. \\
	\hline
\end{tabular}}
\caption{Características mais selecionadas entre bases e entre algoritmos de seleção}
\label{table:features_most_selected}
\end{table}

É importante ressaltar que, para o melhor resultado apresentado na base de filmes pelo c4.5, somente essas duas características foram selecionadas para gerar a base de regras e classificar os documentos. As figuras \ref{figura:movies_dist_1} e \ref{figura:movies_dist_2} mostram a distribuição dessas características na base de filmes e as figuras \ref{figura:amazon_dist_1} e \ref{figura:amazon_dist_2} na base da Amazon. 

\begin{figure}[H]
\caption{Distribuição dos valores da característica "A diferença entre as somas positiva e negativa de adjetivos e bigrams compostos estritamente por advérbio e adjetivo" na base de filmes}
\centering
\includegraphics[scale=0.45]{movies_positive_to_negative_ratio_of_adjectives_sum_and_bigrams_with_adjectives}
\label{figura:movies_dist_1}
\end{figure}

\begin{figure}[H]
\caption{Distribuição dos valores da característica "A diferença entre as somas positiva e negativa de adjetivos e bigrams compostos estritamente por advérbio e adjetivo" na base de filmes}
\centering
\includegraphics[scale=0.45]{movies_positive_to_negative_ratio_of_unigrams_and_bigrams_sum}
\label{figura:movies_dist_2}
\end{figure}


\begin{figure}[H]
\caption{Distribuição dos valores da característica "A diferença entre as somas positiva e negativa de adjetivos e bigrams compostos estritamente por advérbio e adjetivo" na base da Amazon}
\centering
\includegraphics[scale=0.45]{amazon_positive_to_negative_ratio_of_adjectives_sum_and_bigrams_with_adjectives}
\label{figura:amazon_dist_1}
\end{figure}

\begin{figure}[H]
\caption{Distribuição dos valores da característica "A diferença entre as somas positiva e negativa de adjetivos e bigrams compostos estritamente por advérbio e adjetivo" na base da Amazon}
\centering
\includegraphics[scale=0.45]{amazon_positive_to_negative_ratio_of_unigrams_and_bigrams_sum}
\label{figura:amazon_dist_2}
\end{figure}

Na base de filmes, as características se aproximam de uma distribuição normal simétrica, enquanto na base da Amazon são mais assimétricas. Essa falta de equilibrio explica a mais significativa discrepância entre os valores de TPR  e TNR para o c4.5 na base da Amazon, enquanto que na base de filmes, o desequilíbrio entre essas duas medidas é menor. 

\todo[inline]{a acurácia está péssima, quase jogar uma moeda, porque?}
\todo[inline]{o desvio padrão aqui chamada muita atenção, está muito muito alto, , mas o cfs tem desvio menos de TNR e TPR, mas o c4.5 consegui menor desvio de acurácia, porque?}
\todo[inline]{as caracteristicas selecionadas são sempre as mesmas? quantas diferentes? quais as mais populares? o CFS e c4.5 escolhem alguma caracteristica em comum? coloca no texto}
\todo[inline]{pega cada tabela para discutir o que ela apresenta, TNR, TPR, acurácia, caracteristicas selecionadas, não somente descreva mas tente discutir o que ocorreu, e complemente com informações que ajudam o leitor a saber o que está ocorrendo, como informando algo sobre as caracteristicas selecionadas, as regras geradas, etc}

%SVM Filmes

%---> Avg TPR:  69.1 %
%Standard Deviation:  7.3 %
%---> Avg TNR:  72.1 %
%Standard Deviation:  6.87677249878 %
%---> Avg acuracia:  70.6 %
%Standard Deviation:  3.44818792991 %

%WILCOXON
%two-tailed

%Result 1 - Z-value
%The Z-value is -1.1212. The p-value is 0.26272. The result is not significant at p≤ 0.01.
%
%Result 2 - W-value
%The W-value is 16.5. The critical value of W for N = 10 at p≤ 0.01 is 3. Therefore, the result is not significant at p≤ 0.01.


%SVM
%
%---> Avg TPR:  65.8 %
%Standard Deviation:  4.6432747065 %
%---> Avg TNR:  75.5 %
%Standard Deviation:  3.20156211872 %
%---> Avg acuracia:  70.65 %
%Standard Deviation:  2.99207286008 %

%WILCOSOX
%
%Result 1 - Z-value
%The Z-value is -2.3953. The p-value is 0.0164. The result is not significant at p≤ 0.01.
%
%Result 2 - W-value
%The W-value is 4. The critical value of W for N = 10 at p≤ 0.01 is 3. Therefore, the result is not significant at p≤ 0.01.

%Em ambas as bases as diferenças entre os resultados não são significativas usando o teste \textit{Wilcoxon signed-rank} para $p <= 0.01$. \todo[inline]{o teste foi para acurácia? tem que dizer qual medida foi utilizada} Isso mostra que mesmo que o CFS utiliza quase 5 vezes mais características em ambas as bases e não consegue produzir resultados significativamente melhores, criando ainda regras mais complexas e de difícil compreensão para seres humanos. Assim, já que o algoritmo c4.5 somente precisou de 2 características, produzindo regras menos complexas e mais claras, para produzir resultados próximos ou iguais ao do CFS, decidiu-se em manter o c4.5 para os próximos cenários de avaliação. 
\todo[inline]{c4.5 precisou de 2 caract? sua tabela indica 1 caracteristica, então seriam caracteristicas diferentes para cada base? ou na mesma base, folds diferentes? apresenta o histograma destas 2 caracteristicas para ilustar}

\todo[inline]{vocÊ consegue justificar porque o classificador estava jogando quase tudo para uma classe? o problema é na features selecionada, na modelagem do conjunto fuzzy, no metodo de inferencia? explica um pouco}

\todo[inline]{ressalte que os folds foram definidos por amostragem estratificada, mantendo a mesma proporção de documentos positivos e negativos do dataset total, e assim cada fold também está balanceado, isso ajuda a mostrar que um eventual enviesamento dos folds não foi a causa deste comportamento}

%\subsubsection{Avaliação do impacto da altura da árvore de decisão do algoritmo de seleção de características do c4.5}
\todo[inline]{acho melhor não fazer uma seção separada para arvore de altura  3, sugiro incorporar a seção anterior e incluir mais uma coluna nas tabelas, ficando c4.5 altura 1 e c4.5 altura 3}

%É importante ressaltar que os resultados anteriores, para o c4.5, foram obtidos com o algoritmo otimizado, utilizando a ferramenta Weka [CITE], para que a árvore do algoritmo de seleção de características do algoritmo tivesse altura 1, para fins de simplificação das regras geradas. Dessa forma, decidimos avaliar o quanto o aumento da altura da árvore de decisão do algoritmo de seleção de características do c4.5 influenciaria na classificação dos documentos. Definimos que o limite da altura da árvore de decisão fosse 3 para não aumentar demais a complexidade das regras que seriam geradas. A tabela (\ref{table:movies_h3}) e tabela (\ref{table:amazon_h3}) mostram, respectivamente, os resultados para este cenário nas bases de filmes e da Amazon.

\todo[inline]{o que seria 'algoritmo otimizado'? não devemos 'forçar' a arvore a ficar com altura 3, devemos dizer que utilizamos as características  selecionadas foram aquelas existentes até altura 3, porque se a árvore crescer além disso, é só ignorar os demais nós}

% \begin{table}[!h]
%    \begin{tabular}{lll}
%    Movies                                            & c4.5 - Altura 3                                          & c4.5 - Altura 1                               \\ \hline
%    TNR                                               & 55.6\% $\pm$ 43.49\%                            & 47.1\% $\pm$ 42.67\% \\
%    TPR                                               & 53.4\% $\pm$ 44.50\%                            & 61.7\% $\pm$ 43.93\% \\
%    acuracia                                      & 54.5\% $\pm$ 1.76\%                             & 54.4\% $\pm$ 1.72\% \\
%    Características selecionadas      & 1.1 $\pm$ 0.3                                           & 1                                     \\
%    \end{tabular}
%    \caption{Resultados da base de filmes}
%    \label{table:movies_h3}
%\end{table}

%Result 1 - Z-value
%
%The Z-value is -1.7838. The p-value is 0.07508. The result is not significant at p≤ 0.01.
%
%Result 2 - W-value
%
%The W-value is 10. The critical value of W for N = 10 at p≤ 0.01 is 3. Therefore, the result is not significant at p≤ 0.01.

%\begin{table}[!h]
%    \begin{tabular}{lll}
%    Movies                                              & c4.5 - Altura 3                               & c4.5 - Altura 1                              \\ \hline
%    TNR                                                     & 73.7\% $\pm$ 35.86\%                  & 34.0\% $\pm$ 43.26\%  \\
%    TPR                                                 & 47.1\% $\pm$ 34.22\%                  & 74.5\% $\pm$ 38.82\% \\
%    acuracia                                        & 60.4\% $\pm$ 5.75\%                       & 54.25\% $\pm$ 2.82\% \\
%    Características selecionadas        & 1.9 $\pm$ 0.7                                 & 1 $\pm$                                  \\
%    \end{tabular}
%    \caption{Resultados da base da Amazon}
%    \label{table:amazon_h3}
%\end{table}

\todo[inline]{nas tabelas que mostram resultado do c4.5 altura 3, indicam um valor baixo da quant media de caract selecionadas, a mesma caracteristica está sendo selecionada por vários nós? diga quais são e discuta no texto esse comportamento, isso pode indicar que uma característica é repeditamente relevante em vários splits dos dados (cada nó da árvore de decisão, divide a base para os nós seguintes)}

%Result 1 - Z-value
%
%The Z-value is -2.1915. The p-value is 0.02852. The result is not significant at p≤ 0.01.
%
%Result 2 - W-value
%
%The W-value is 6. The critical value of W for N = 10 at p≤ 0.01 is 3. Therefore, the result is not significant at p≤ 0.01.

%Mais uma vez, os resultados apresentados em ambas as bases para o aumento da altura da árvore de decisão do algoritmo de seleção de características não produziram resultados significativamente melhores para $p <= 0.01$ no teste \textit{Wilcoxon signed-rank}. \todo[inline]{sempre precisa dizer qual medida está sendo avalida pelo testes estatístico, reveja aqui e no restante do texto} Assim, o crescimento da árvore, que resulta no aumento da complexidade das regras em relação a altura 1, não contribui para o aumento de performance do nosso classificador e na simplificação das regras geradas para a classificação. 

\subsection{Avaliação dos sistemas de inferência fuzzy}

Nessa subseção são avaliados os desempenhos dos sistemas de inferência escolhidos, o Método do Raciocínio Fuzzy Geral (MRFG) e o Método do Raciocínio Fuzzy Clássico (MRFC). Os resultados mostram que em todos os cenários, independente do algoritmo de seleção usado, o MRFG produz melhores melhores percentuais de acurácia que o MRFC, mesmo que em alguns casos a diferença não seja significativa (para $p <= 0.01$). Isso indica que a abordagem do MRFC de analisar individualmente a classificação de um único documento em vez de analisar todo o conjunto de dados frente às regras, como faz o MRFG, é uma abordagem que tente a ter menos eficácia. É importante observar, contudo, que ambos os métodos apresentam altos desvios padrão nas medidas de TPR e TNR em todos os métodos de seleção, mostrando ainda instabilidade no sistema, como exemplificado na tabela \ref{table:movie_folds}, mesmo que menor. 

O uso de pesos nas regras, por outro lado, diminuiu bastante os desvios padrão de TPR e TNR, além de aumentar o desempenho geral (acurácia) do classificador em todos os cenários. Destaque principalmente para os cenários que usam o c.4.5, com menos características e, por conseguinte, menos regras, para a aumento significativo (para $p <= 0.01$) na base de filmes, usando c4.5 com altura 1 e o MRFG usando pesos, que saltou de 59,2\% (sem pesos) para 70,05\% de acurácia; e na base da Amazon, no mesmo cenário, saltando de 60,05\% para 70,85\% de acurácia. Ainda nesses cenários, os desvios padrão de TPR e TNR estavam acima de 30\% e, utilizando os pesos, caíram para menos de 10\% em filmes e 6\% na base mista da Amazon.

Isso acontece pois a aplicação dos pesos aumenta o espaço de cobertura das regras mais relevantes (ou com capacidade maior de classificar mais e melhor documentos) e diminui o espaço de regras que não tem a mesma importância e que são capazes de classificar somente uma menor região do espaço de dados. Por exemplo, as regras a seguir foram geradas um alguns dos folds para classificação usando c4.5 na base de filmes:

\begin{itemize}
\item Fold 0
\begin{itemize}
\item SE a diferença entre as somas positiva e negativa de adjetivos e bigrams compostos estritamente por advérbio e adjetivo é BAIXA então a polaridade é NEGATIVA;
\item SE a diferença entre as somas positiva e negativa de adjetivos e bigrams compostos estritamente por advérbio e adjetivo é MÉDIA então a polaridade é POSITIVA;
\item SE a diferença entre as somas positiva e negativa de adjetivos e bigrams compostos estritamente por advérbio e adjetivo é ALTA então a polaridade é POSITIVA;
\end{itemize}
\item Fold 1
\begin{itemize}
\item SE a diferença entre as somas positiva e negativa de unigrams e bigrams combinados é BAIXA então a polaridade NEGATIVA;
\item SE a diferença entre as somas positiva e negativa de unigrams e bigrams combinados é MÉDIA então a polaridade NEGATIVA;
\item SE a diferença entre as somas positiva e negativa de unigrams e bigrams combinados é ALTA então a polaridade POSITIVA;
\end{itemize}
\end{itemize}

Lembrando que "BAIXA", "MÉDIA" e "ALTA" são as modelagens dos conjuntos fuzzy para nossas variáveis de entrada e em quase todos os folds, as regras foram iguais. Contudo, em cada fold, as regras tem igual importância, quando não se utiliza os pesos. O resultado da classificação usando essas (e as demais geradas) com MRFG como sistema de inferência e sem pesos é apresentado na tabela \ref{table:movie_folds_2}.

\begin{table}[H]
	\centering
    \begin{tabular}{ll}
    Fold 0 \\ \hline
    Accuracy &  59.0\% \\
	TPR &  97.0\% \\
	TNR &  21.0\% \\ \\
	
	Fold 1 \\ \hline
    Accuracy &  57.5\% \\
	TPR &  17.0\% \\
	TNR &  98.0\% \\ \\
	
	Fold 2 \\ \hline
    Accuracy &  60.0\% \\
	TPR &  91.0\% \\
	TNR &  29.0\% \\ \\
	
	Fold 3 \\ \hline
    Accuracy &  60.0\% \\
	TPR &  98.0\% \\
	TNR &  22.0\% \\ \\
	
	Fold 4 \\ \hline
    Accuracy &  61\% \\
	TPR &  98.0\% \\
	TNR &  24.0\% \\ \\
	
	Fold 5 \\ \hline
    Accuracy &  55\% \\
	TPR & 14.0\% \\
	TNR &  99.0\% \\ \\
	
	Fold 6 \\ \hline
    Accuracy &  58\% \\
	TPR &  18.0\% \\
	TNR &  98.0\% \\ \\
	
	Fold 7 \\ \hline
    Accuracy &  60.5\% \\
	TPR &  94.0\% \\
	TNR &  27.0\% \\ \\
	
	Fold 8 \\ \hline
    Accuracy &  61.5\% \\
	TPR &  29.0\% \\
	TNR &  94.0\% \\ \\
	
	Fold 9 \\ \hline
    Accuracy &  59.5\% \\
	TPR &  90.0\% \\
	TNR &  29.0\% \\
    \end{tabular}
    \caption{Resultados com c4.5 e MRFG sem pesos - base de filmes}
    \label{table:movie_folds_2}
\end{table}

É possível perceber a instabilidade do sistema e recordar a tabela \ref{table:movies_3f} que nos mostra o resultado final de 59.2\% de acurácia, 53.8\% +- 34.96\% de TNR e 64.6\% +- 37.08\% com altos percentuais nos desvios padrão. O uso dos pesos, porém, vai alterar a importância entre as regras, enaltecendo àquelas mais aptas à classificação. As mesmas regras acima recebem os seguintes pesos:

\begin{itemize}
\item Fold 0
\begin{itemize}
\item SE a diferença entre as somas positiva e negativa de adjetivos e bigrams compostos estritamente por advérbio e adjetivo é BAIXA então a polaridade é NEGATIVA - \textbf{Grau: 0.60801671};
\item SE a diferença entre as somas positiva e negativa de adjetivos e bigrams compostos estritamente por advérbio e adjetivo é MÉDIA então a polaridade é POSITIVA - \textbf{Grau: 0.01536631};
\item SE a diferença entre as somas positiva e negativa de adjetivos e bigrams compostos estritamente por advérbio e adjetivo é ALTA então a polaridade é POSITIVA  - \textbf{Grau: 0.6912395};
\end{itemize}
\item Fold 1
\begin{itemize}
\item SE a diferença entre as somas positiva e negativa de unigrams e bigrams combinados é BAIXA então a polaridade NEGATIVA  - \textbf{Grau: 0.49867233};
\item SE a diferença entre as somas positiva e negativa de unigrams e bigrams combinados é MÉDIA então a polaridade NEGATIVA  - \textbf{Grau: 0.01143232};
\item SE a diferença entre as somas positiva e negativa de unigrams e bigrams combinados é ALTA então a polaridade POSITIVA  - \textbf{Grau: 0.67380056};
\end{itemize}
\end{itemize}

É possível perceber que as regras do conjunto "MEDIA" recebem os menores graus, enquanto que as regras dos conjuntos "BAIXA" e "ALTA" recebem os maiores - esse comportamento se repete para todos os demais folds. Isso mostra que o conjunto "BAIXA" e as regras relacionadas não são boas ou tão importantes para classificar os documentos usando essas características. Assim, a disputa entre as regras no momento da classificação ficará somente entre as demais. A tabela TAL mostra os resultados para a mesma configuração da tabela \ref{table:movie_folds_2}, excetuando o uso dos pesos. 

\begin{table}[H]
	\centering
    \begin{tabular}{ll}
    Fold 0 \\ \hline
    Accuracy &  71.0\% \\
	TPR &  65.0\% \\
	TNR &  77.0\% \\ \\
	
	Fold 1 \\ \hline
    Accuracy &  71.0\% \\
	TPR &  65.0\% \\
	TNR &  77.0\% \\ \
	
	Fold 2 \\ \hline
    Accuracy &  73.5\% \\
	TPR &  70.0\% \\
	TNR &  77.0\% \\ \
	
	Fold 3 \\ \hline
    Accuracy &  67.0\% \\
	TPR &  80.0\% \\
	TNR &  54.0\% \\ \\
	
	Fold 4 \\ \hline
    Accuracy &  75.5\% \\
	TPR &  80.0\% \\
	TNR &  71.0\% \\ \\
	
	Fold 5 \\ \hline
    Accuracy &  61.5\% \\
	TPR & 50.0\% \\
	TNR &  73.0\% \\ \\
	
	Fold 6 \\ \hline
    Accuracy &  67\% \\
	TPR &  58.0\% \\
	TNR &  76.0\% \\ \\
	
	Fold 7 \\ \hline
    Accuracy &  75\% \\
	TPR &  81.0\% \\
	TNR &  69.0\% \\ \\
	
	Fold 8 \\ \hline
    Accuracy &  69\% \\
	TPR &  71.0\% \\
	TNR &  67.0\% \\ \\
	
	Fold 9 \\ \hline
    Accuracy &  70\% \\
	TPR &  77.0\% \\
	TNR &  63.0\% \\
    \end{tabular}
    \caption{Resultados com c4.5 e MRFG sem pesos - base de filmes}
    \label{table:movie_folds_3}
\end{table}

Agora, diferentemente de antes, é possível perceber a maior estabilidade do sistema e recordar a tabela \ref{table:movies_3f} que nos mostra o resultado final de 70.05\% de acurácia, 70.4\% +- 7.11\% de TNR e 69.7\% +- 9.81\% com baixos percentuais nos desvios padrão.

%Da mesma maneira que foi feita na seção anterior, nós fixamos os demais parâmetros do experimento para melhor avaliar os sistemas de inferência, mantendo o algoritmo de seleção de característica c4.5 (com altura 1, para continuar buscando a geração de regras claras e de fácil leitura para humanos) e 3 conjuntos fuzzy nas variáveis de entrada. A tabela (\ref{table:movies2}) e tabela (\ref{table:amazon2}) mostram, respectivamente, os resultados para este cenário nas bases de filmes e da Amazon.
%
%\begin{table}[!h]
%    \begin{tabular}{lll}
%    ~                   & CFRM                              & GFRM \\ \hline
%    TNR                 & 47.1\% $\pm$ 42.67\%   & 53.8\% $\pm$ 34.96\%    \\
%    TPR             & 61.7\% $\pm$ 43.93\%   & 64.6\% $\pm$ 37.08\%   \\
%    acuracia        & 54.4\% $\pm$ 1.72\%       & 59.2\% $\pm$ 1.83\%    \\
%    \end{tabular}
%    \caption{Resultados dos sistemas de inferência na base de filmes}
%    \label{table:movies2}
%\end{table}

%Result 1 - Z-value
%
%The Z-value is -2.8031. The p-value is 0.00512. The result is significant at p≤ 0.01.
%
%Result 2 - W-value
%
%The W-value is 0. The critical value of W for N = 10 at p≤ 0.01 is 3. Therefore, the result is significant at p≤ 0.01.

%\begin{table}[!h]
%    \begin{tabular}{lll}
%    ~                   & CFRM                                  & GFRM \\ \hline
%    TNR                 & 34.0\% $\pm$ 43.26\%      & 44.6\% $\pm$ 35.73\%    \\
%    TPR             & 74.5\% $\pm$ 38.82\%      & 75.5\% $\pm$ 34.80\%    \\
%    acuracia        & 54.25\% $\pm$ 2.82\%      & 60.05\% $\pm$ 2.37\%   \\
%    \end{tabular}
%    \caption{Resultados dos sistemas de inferência na base da Amazon}
%    \label{table:amazon2}
%\end{table}

%Result 1 - Z-value
%
%The Z-value is -2.8031. The p-value is 0.00512. The result is significant at p≤ 0.01.
%
%Result 2 - W-value
%
%The W-value is 0. The critical value of W for N = 10 at p≤ 0.01 is 3. Therefore, the result is significant at p≤ 0.01.

%Os resultados mostram que Método Geral do Raciocínio Fuzzy (MRFG) aumenta o \textit{acuracia} sobre o MRFC, em ambas bases,  mantendo a seleção de características e a quantidade de conjuntos fuzzy inalterados. O resultado do teste \textit{Wilcoxon signed-rank} confirma a melhora significativa do MRFG sobre o MRFC, para $p <= 0.01$. Assim, nessa tarefa de classificação de somente duas classes, positivo e negativo, os resultados mostraram que uma melhor abordagem é considerar todas as regras de uma classe, em vez de uma única com maior grau. Daí, o MRFG foi a nossa escolha para prosseguir nos próximos experimentos com o fim de alcançar melhores resultados nesse trabalho.

%\subsection{Avaliação de uso de pesos nas regras dos sistemas de inferência}
%
%Em \cite{ishibuchi2001effect} foi mostrado que é possível aumentar a performance da classificação de regras fuzzy IF-THEN, aplicando pesos à elas, além do grau de compatibilidade das regras. No referido artigo, os autores descreveram o processo em que é possível melhorar o desempenho da classificação sem alterar os conjuntos fuzzy das variáveis de saída e de entrada. Baseando-se neste artigo, este trabalho calculou os pesos como se segue. Para cada regra da base de regras gerada, foi calculado o grau de compatibilidade com todos documentos do conjunto de teste. Se o documento fosse positivo, o grau de compatibilidade era acumulado em $\beta_{positivo}$; se o documento fosse negativo, o grau de compatibilidade era acumulado em $\beta_{negativo}$. Ao fim desse processo, caso ambos os betas fosse iguais, não haveria peso a ser considerado, já que as regras tem igual influência sobre o conjunto de dados que elas foram geradas. De outra forma, o peso da regra era definido pela equação \ref{eq:pesos}.
%
%\begin{equation}
%P_j = |\beta_{positivo} - \beta_{negativo}| / (\beta_{positivo} + \beta_{negativo})
%\label{eq:pesos}
%\end{equation}
%
%onde $P_j$ é o peso da regra, para $0 <= P_j <= 1$. 

\todo[inline]{discuta qual é idéia por trás do uso de pesos, valorizar regras que de fato auxiliar a distinguir as duas classes e desvalorizar aquelas que não conseguem separar as classes}

%Uma vez que todas as regras tiveram seus pesos calculados, estes são adicionados ao processo de classificação de ambos os métodos utilizados até aqui, o MRFG e o MRFC. O peso torna-se um fator multiplicador do grau de cada regra. Assim, o MRFC em vez de somente levar em consideração a regra com maior grau de compatibilidade com o documento de teste, vai, agora, levar em consideração a regra com maior grau multiplicado pelo peso da regra. O mesmo acontece para o MRFG ao considerar o grau médio entre as classes positivo e negativo. A tabela (\ref{table:movies2_pesos}) e tabela (\ref{table:amazon2_pesos}) mostram, respectivamente, os resultados para o uso dos pesos nas bases de filmes e da Amazon usando os parâmetros até agora estabelecidos. 

\todo[inline]{discuta mais esses resultados, há uma grande mudança qualitativa dos resultados com uso de pesos, principalmente porque o desvio padrão cai muito, quais foram as regras que ganharam pesos e quais perderam peso? quais os valores de pesos encontrados, são próximos dos valores extremos, 0 e 1? como isso ajuda a explicar os resultados ruins sem peso? discuta o desvio padrão também }

%\todo[inline]{as tabelas são para classico ou geral?}
%\begin{table}[!h]
%    \begin{tabular}{lll}
%    ~                   & S/ Pesos                              & C/ Pesos \\ \hline
%    TNR                 & 53.8\% $\pm$ 34.96\%      & 70.4\% $\pm$ 7.11\%    \\
%    TPR             & 64.6\% $\pm$ 37.08\%      & 69.7\% $\pm$ 9.81\%   \\   
%    acuracia        & 59.2\% $\pm$ 1.83\%       & 70.05\% $\pm$ 4.00\%    \\
%    \end{tabular}
%    \caption{Resultados dos sistemas de inferência na base de filmes utilizando pesos nas regras}
%    \label{table:movies2_pesos}
%\end{table}

%Result 1 - Z-value
%
%The Z-value is -2.8031. The p-value is 0.00512. The result is significant at p≤ 0.05.
%
%Result 2 - W-value
%
%The W-value is 0. The critical value of W for N = 10 at p≤ 0.05 is 8. Therefore, the result is significant at p≤ 0.05.

%\begin{table}[!h]
%    \begin{tabular}{lll}
%    ~                   & S/ Pesos                                  & C / Pesos \\ \hline
%    TNR                 & 44.6\% $\pm$ 35.73\%      & 76.80\% $\pm$ 4.57\%    \\
%    TPR             & 75.5\% $\pm$ 34.80\%      & 64.9\% $\pm$ 5.50\%    \\
%    acuracia        & 60.05\% $\pm$ 2.37\%      & 70.85\% $\pm$ 3.09\%   \\
%    \end{tabular}
%    \caption{Resultados dos sistemas de inferência na base da Amazon utilizando pesos nas regras}
%    \label{table:amazon2_pesos}
%\end{table}

%Result 1 - Z-value
%
%The Z-value is -2.8031. The p-value is 0.00512. The result is significant at p≤ 0.05.
%
%Result 2 - W-value
%
%The W-value is 0. The critical value of W for N = 10 at p≤ 0.05 is 8. Therefore, the result is significant at p≤ 0.05.

%Os resultados corroboraram as conclusões de \cite{ishibuchi2001effect} que é possível melhorar o desempenho da classificação sem alterar os conjuntos fuzzy das variáveis de saída e de entrada apenas aplicando pesos às regras geradas. O resultado do teste \textit{Wilcoxon signed-rank} também confirma a melhora significativa do MRFG usando pesos nas regras em vez de somente o grau de compatibilidade, para $p <= 0.01$.

\section{Avaliação com o uso de 2 conjuntos fuzzy}
\subsection{Avaliação dos algoritmos de seleção de características}

O efeito que os pesos produziram nos sistemas de inferência, mostrou que o conjunto fuzzy "MEDIA" não estava sendo utilizado na classificação dos documentos, pois as regras geradas com esse conjunto fuzzy sempre recebiam pesos próximos de zero ou bem distantes dos pesos recebidos pelas regras que utilizavam os conjuntos "BAIXA" e "ALTA". De fato, a presença desse conjunto para a geração das regras estava influenciando negativamente nas taxas de classificação dos documentos. Assim, nós decidimos remover esse conjunto para ratificar ou não essa hipótese, além de verificar se menos regras e mais simples terão o mesmo desempenho que antes, já que um conjunto a menos na entrada significa menos regras geradas e com antecedentes menores e mais legíveis. Os resultados com dois conjuntos fuzzy na entrada podem ser vistos nas tabelas \ref{table:movies_2f} e \ref{table:amazon_2f}.

É possível observar que os melhores resultados que vimos com três conjuntos fuzzy na entrada, melhoraram um pouco, embora não tenha sido significativo ($p <= 0.01$). Na verdade, podemos notar que, diferentemente de antes, os resultados para os cenários entre c4.5 e CFS não estão tão mais distantes como antes e, na base da Amazon, os resultados obtidos utilizando o CFS superaram muitos dos resultados produzidos com o c4.5. Isso acontece, pois além do CFS selecionar mais características que cobrem mais o espaço dos dados, a regras geradas utilizam agora somente os conjuntos fuzzy que realmente tem capacidade de classificar melhor os documentos, potencializando o conjunto selecionado pelo CFS. 

Mas, a despeito da redução dos conjuntos fuzzy, a média de características selecionadas se manteve próxima a de 6 entre as duas bases. Além disso, as mesmas características mostradas na tabela \ref{table:features_most_selected} que despontaram antes, se mantiveram como as mais selecionadas entre as bases e entre os métodos de seleção. Por fim, pudemos testar nossas hipóteses de que o conjunto fuzzy "MEDIA" era, de fato, descartável e que menos regras e mais simples puderam produzir resultados tão bons quanto demonstrado anteriormente com mais conjuntos fuzzy.

\subsection{Avaliação dos sistemas de inferência fuzzy}

Os resultados mostram agora que não há diferenças significativas ($p <= 0.01$) em nenhum dos cenários entre os algoritmos  MRFC e MRFG. Em alguns casos, diferentemente do que ocorria antes com 3 conjuntos, o MRFC obteve, mesmo que somente numericamente, melhor acurácia que o MRFG quando utilizado o c4.5 com altura 2. Outra mudança foi a eliminação do efeito negativo do MRFC de analisar individualmente a classificação de um único documento em vez de analisar todo o conjunto de dados frente às regras, como é feito no MRFG. Isso pode indicar que, essa abordagem do MRFC pode ser afetada diretamente e de maneira mais expressiva pela modelagem das variáveis de entrada, enquanto que o MRFG não sofre igualmente desses efeitos. Além disso, os desvios padrão das medidas de TPR e TNR, mesmo sem utilizar os pesos nas regras, foram reduzidos significativamente, em média, para 8\% entre as bases, contra 21\% utilizando 3 conjuntos fuzzy. 

Como os pesos mostraram quais regras seriam descartáveis e, por conseguinte, o conjunto associado, a remoção deste produziu efeitos similares nos resultados do MRFC e do MRFG sem utilizar peso algum. Os resultados melhoraram em todos os cenários, destaque para o uso com o CFS na base da Amazon, pois foram melhores significativamente ($p <= 0.01$), comparando par a par com os cenários da mesma base utilizando 3 conjuntos fuzzy na entrada. 

A aplicação dos pesos utilizando 2 conjuntos, contudo, não pareceu surtir o mesmo efeito que produziu anteriormente. Os resultados foram inconclusivos quanto à aplicação de pesos nestes cenários com somente 2 conjuntos fuzzy na entrada. Os cenários que utilizaram o c4.5 com altura 1 não surtiram efeito algum. Com altura 2 os pesos reduziram, inclusive, o desempenho da classificação. E nos cenários com CFS, os pesos sempre aumentaram os valores de acurácia da classificação, mesmo que não tenham sido significativos. 

%Através das últimas seções, foram mostrados resultados utilizando 3 conjuntos fuzzy para modelar nossas variáveis lingüísticas. Seguindo a nossa decisão de usar o c4.5 com altura máxima da árvore de decisão para 1 para reduzir a complexidade das regras geradas e torna-las mais legíveis para seres humanos, nós tentamos reduzir a quantidade de conjuntos fuzzy, usando somente os conjuntos "Baixo" e "Alto", para as variáveis de entrada. Esse experimento tem por fim verificar se há aumento da performance da classificação com regras mais simples e gerais. A tabela (\ref{table:movies2_pesos_2f}) e tabela (\ref{table:amazon2_pesos_2fs}) mostram, respectivamente, os resultados para o uso de somente dois conjuntos fuzzy nas variáveis de entrada nas bases de filmes e da Amazon.
%
%\begin{table}[!h]
%    \begin{tabular}{lll}
%    ~                   & 3 conjuntos fuzzy                             & 2 conjuntos fuzzy \\ \hline
%    TNR                 & 70.4\% $\pm$ 7.11\%                   & 71.2\% $\pm$ 4.33\%    \\
%    TPR             & 69.7\% $\pm$ 9.81\%                   & 70.6\% $\pm$ 3.35\%   \\
%    acuracia        & 70.05\% $\pm$ 4.00\%              & 70.9\% $\pm$ 3.07\%    \\
%    \end{tabular}
%    \caption{Resultados dos sistemas de inferência na base de filmes utilizando 2 conjuntos fuzzy na entrada}
%    \label{table:movies2_pesos_2fs}
%\end{table}
%
\todo[inline]{parece haver um redução no desvio padrão com uso de 2 conjuntos, porque será que reduziu?}
%
%%Result 1 - Z-value
%%
%%The Z-value is -1.2741. The p-value is 0.20408. The result is not significant at p≤ 0.05.
%%
%%Result 2 - W-value
%%
%%The W-value is 15. The critical value of W for N = 10 at p≤ 0.05 is 8. Therefore, the result is not significant at p≤ 0.05.
%
%\begin{table}[!h]
%    \begin{tabular}{lll}
%    ~                   & 3 conjuntos fuzzy                             & 2 conjuntos fuzzy \\ \hline
%    TNR                 & 76.80\% $\pm$ 4.57\%              & 77.2\% $\pm$ 3.54\%    \\
%    TPR             & 64.9\% $\pm$ 5.50\%                   & 64.4\% $\pm$ 3.46\%   \\
%    acuracia        & 70.85\% $\pm$ 3.09\%              & 70.8\% $\pm$ 3.27\%    \\
%    \end{tabular}
%    \caption{Resultados dos sistemas de inferência na base da Amazon utilizando 2 conjuntos fuzzy na entrada}
%    \label{table:amazon2_pesos_2fs}
%\end{table}
%
%%Result 1 - Z-value
%%
%%The Z-value is -0.3568. The p-value is 0.71884. The result is not significant at p≤ 0.01.
%%
%%Result 2 - W-value
%%
%%The W-value is 24. The critical value of W for N = 10 at p≤ 0.01 is 3. Therefore, the result is not significant at p≤ 0.01.
%
%A tabela (\ref{table:movies2_pesos_2fs}) mostra pequena melhora na base de filmes e empate técnico na base da Amazon. Mas, estatisticamente, os resultados não são significativamente diferentes para $p <= 1$. Todavia, a redução da quantidade dos conjuntos fuzzy para as variáveis de entrada produz regras mais simples e legíveis para serem humanos. Assim, nós temos regras menos complexas com a mesma performance de regras com maior número de antecedentes. 
%
%%positive_to_negative_ratio_of_adjectives_sum_and_bigrams_with_adjectives
%%positive_to_negative_ratio_of_unigrams_and_bigrams_sum
%Em ambos conjuntos de dados somente duas características foram selecionadas pelo c4.5:
%\begin{itemize}
%\item Diferença entre a soma positiva e negativa dos adjetivos e dos bigrams formados por advérbio e adjetivo
%\item Diferença entre a soma positiva e negativa dos unigrams e bigrams
%\end{itemize}
%
%Na base de filmes as duas foram necessárias para atingir os resultados mostrados na tabela ((\ref{table:movies2_pesos_2fs})). Isso remete ao fato de documentos de filmes serem mais difíceis de ser classificados, precisando de mais características para caracterizar corretamente as opiniões \cite{turney2002thumbs, pang2004sentimental, chaovalit2005movie, ohana2009sentiment}. Já na base da Amazon, somente a segunda característica foi utilizada para classificar os documentos. Embora o algoritmo de seleção de características do c4.5 tenha decidido que somente essa característica seja suficiente, não é difícil concluir que, sendo a base da Amazon bastante diversa (e que também inclui filmes), mais características podem ser necessárias para caracterizar melhor documentos tão variados.
%Todavia, nós conseguimos classificar um pouco mais de 70\% dos documentos de filmes e da Amazon com poucas regras simples, legíveis para humanos, geradas pelo método de Wang-Mendel:
%
%\begin{itemize}
%\item IF a \textit{diferença entre a soma positiva e negativa dos unigrams e bigrams} is ALTO then POLARIDADE is POSITIVA
%\item IF a \textit{diferença entre a soma positiva e negativa dos unigrams e bigrams} is BAIXO then POLARIDADE is NEGATIVA
%\item IF a \textit{diferença entre a soma positiva e negativa dos adjetivos e dos bigrams formados por advérbio e adjetivo} is ALTO then POLARIDADE is POSITIVA
%\item IF a \textit{diferença entre a soma positiva e negativa dos adjetivos e dos bigrams formados por advérbio e adjetivo} is BAIXO then POLARIDADE is NEGATIVA
%\end{itemize}
%
%As quatro regras foram utilizadas na base de filmes e as duas primeiras regras na base da Amazon.

\section{Avaliação do uso de regras entre domínios}
\todo[inline]{faz aplicação cruzada tambem, amazon na de filmes e vice versa}
\todo[inline]{valorize mais esta seção, o resultado é MUITO interessante, mas o texto não valoriza tanto, compare com os resultados de epinions com regras amazon e filmes. Quando aplicar amazon em filmes e vice versa compare com os resultados obtidos usando as regras da própria base. O uso do classificador de uma base em outra pode indicar a existencia de regras universais e realmente independentes de domínio, mas pode haver necessidade de ajuste na modelagem dos conjuntos ou em outro aspecto talvez}
\todo[inline]{matheus: como haviamos combinado na sexta, deixaria essa avaliações cruzadas se houvesse tempo ou para o dia da defesa}

Outra avaliação que fizemos foi a validação do uso de regras entre domínios, utilizando a base inteira da Epinions como teste. Foram usadas as regras geradas da base de filmes e da base da Amazon e aplicadas à base da Epinions. Todas as regras de cada fold, em ambas as bases, foram utilizadas para classificar os documentos da Epinions. O resultado final é a media entre esses resultados.\todo[inline]{foram usadas as regras geradas para qual fold amazon ou filmes?}
\todo[inline]{matheus: todas. Para cada fold eu apliquei as regras na base da epinions. O resultado final é a media entre esses resultados. Coloquei no texto.} Para isso, a base da Epinions passou por pré-processamento, transformação e extração de características, para que as regras pudessem avaliar corretamente os documentos. É importante frisar que nenhuma adaptação foi feita às regras, ou qualquer alteração nas características selecionadas e tão pouco na modelagem dos conjuntos fuzzy para que pudesse essa avaliação entre bases pudesse funcionar corretamente. Da maneira que foram geradas na bases de filmes e Amazon, foram utilizadas na classificação dos documentos da base da Epinions. A tabela \ref{table:epinions} mostra os resultados obtidos.
\todo[inline]{nenhuma mudança foi feita nas regras fuzzy, mas nenhuma foi feita também nas características selecionadas e na modelagem dos conjuntos, então reforça isso tambem}
\todo[inline]{matheus: feito} 

\begin{table}[!h]
    \begin{tabular}{lll}
    ~               & Regras da Amazon                  & Regras dos Filmes \\ \hline
    TNR             & 47.05\% $\pm$ 0.35\%           & 61.7\% $\pm$ 1.41\%    \\
    TPR         & 90.5\% $\pm$ 1.11\%               & 85.8\% $\pm$ 1.22\%   \\
    acuracia    & 68.77\% $\pm$ .0.17\%             & 73.75\% $\pm$ 0.27\%    \\
    \end{tabular}
    \caption{Resultados da aplicação de regras da base de filmes e Amazon na base Epinions}
    \label{table:epinions}
\end{table}

Os resultados mostram que as regras geradas podem ser usadas entre domínios diferentes, produzindo resultados próximos ou melhores que os produzidos nas próprias bases, como pode ser visto na tabela \ref{table:epinions}. 

\section{Outros resultados}

Cinquenta e sete características foram extraídas dos documentos para realizar os experimentos dessa pesquisa e as características mostradas na tabela \ref{table:features_most_selected} sempre foram selecionadas pelos algoritmos de seleção de características (CFS e c4.5 com diferentes alturas). Essas características englobam unigrams e bigrams formados por advérbios e adjetivos. Esses resultados corroboram quase que completamente os n-grams propostos por \citeonline{turney2002thumbs}, formados em sua maioria por advérbios e adjetivos; reiteram também a importância central dos adjetivos na mineração e classificação de opiniões, como apontado por \citeonline{voll2007not}; além de reforçar o proposto por \citeonline{benamara2007sentiment} que advérbios tem significativa influência como modificadores de intensidade dos adjetivos.

Também comparamos a configuração final do nosso classificador com um método clássico de aprendizado de máquina muito utilizado na tarefa de mineração de opinião, o \textit{Support Vector Machine} (SVM), e que geralmente produz bons resultados, como ser visto em \citeonline{moraes2012document}, \citeonline{pang2002thumbs}, \citeonline{pang2004sentimental} e \citeonline{wilson2004just}. 

\begin{table}[!h]
    \begin{tabular}{lll}
    ~                   & Wang-Mendel 2 conjuntos fuzzy                     & SVM \\ \hline
    TNR                 & 71.2\% $\pm$ 4.33\%                                           & 72.9\% $\pm$ 5.80\%    \\
    TPR             & 70.6\% $\pm$ 3.35\%                                       & 68.6\% $\pm$ 7.04\%   \\
    acuracia        & 70.9\% $\pm$ 3.07\%                                       & 70.75\% $\pm$ 3.70\%    \\
    \end{tabular}
    \caption{Comparação entre os resultados do método de Wang-Mendel e SVM na base de filmes}
    \label{table:movies_svm}
\end{table}

%Result 1 - Z-value
%
%The Z-value is -0.5606. The p-value is 0.57548. The result is not significant at p≤ 0.01.
%
%Result 2 - W-value
%
%The W-value is 22. The critical value of W for N = 10 at p≤ 0.01 is 3. Therefore, the result is not significant at p≤ 0.01.

\begin{table}[!h]
    \begin{tabular}{lll}
    ~                       & Wang-Mendel 2 conjuntos fuzzy                             & SVM \\ \hline
    TNR                     & 77.2\% $\pm$ 3.54\%                                               & 75.5\% $\pm$ 3.20\%    \\
    TPR                 & 64.4\% $\pm$ 3.46\%                                               & 65.8\% $\pm$ 4.64\%   \\
    acuracia           & 70.8\% $\pm$ 3.27\%                                            & 70.65\% $\pm$ 2.99\%    \\
    \end{tabular}
    \caption{Comparação entre os resultados do método de Wang-Mendel e SVM na base da Amazon}
    \label{table:amazon_svm}
\end{table}

%Result 1 - Z-value
%
%The Z-value is -1.2232. The p-value is 0.22246. The result is not significant at p≤ 0.01.
%
%Result 2 - W-value
%
%The W-value is 15.5. The critical value of W for N = 10 at p≤ 0.01 is 3. Therefore, the result is not significant at p≤ 0.01.

Nossos resultados mostram que o classificador construído nesse trabalho equivale aos resultados do SVM, pois em ambas as bases as diferenças dos resultados não são significativas para $p <= 0.01$. Contudo, nosso classificador possui regras legíveis para seres humanos, é independente de domínio e não precisa de nova rodada de treino para diferentes bases, diferentemente do SVM.

\todo[inline]{falta uma seção final de discussão geral dos resultados para fechar esse capítulo, revisando os achados e fazendo uma avaliação geral de tudo}
\todo[inline]{matheus: não ficaria melhor fazer esse apanhado no inicio da conclusao? Dai eu retomava a introducao, com proposta e objetivos, passava pelos resultados e iniciava a conclusão}

\end{document}

\subfile{introducao}

\subfile{fundamentacao}

\subfile{metodologia}

\subfile{resultados}

\subfile{conclusao}

\backmatter

% Apêndices
% Comente se não houver apêndices
\appendix
%\xchapter{Apêndices}{}
% É aconselhável criar cada apêndice em um arquivo à parte, digamos
% "apendice1.tex", "apendice.tex", ... "apendiceM.tex" e depois
% incluí-los com:
%\section{Características extraídas dos documentos} \label{sec:appendix1}

\begin{itemize}
\item Características de contagem %20
\begin{itemize}
\item Contagem dos adjetivos positivos
\item Contagem dos adjetivos negativos
\item Contagem dos advérbios positivos
\item Contagem dos advérbios negativos
\item Diferença entre a contagem de adjetivos positivos e negativos
\item Diferença entre a contagem de advérbios positivos e negativos
\item Contagem de adjetivos e bigrams positivos compostos por advérbio e adjetivo
\item Contagem de adjetivos e bigrams negativos compostos por advérbio e adjetivo
\item Contagem de advérbios e bigrams positivos compostos somente por advérbios
\item Contagem de advérbios e bigrams negativos compostos somente por advérbios
\item Contagem dos unigrams e bigrams positivos
\item Contagem dos unigrams e bigrams negativos
\item Contagem de unigrams, bigrams e trigrams positivos 
\item Contagem de unigrams, bigrams e trigrams negativos
\item Diferença entre a contagem positiva e negativa de adjetivos e bigrams compostos por advérbio e adjetivo
\item Diferença entre a contagem positiva e negativa de advérbios e bigrams compostos somente por advérbios
\item Diferença entre a contagem positiva e negativa de unigrams e bigrams
\item Diferença entre a contagem positiva e negativa de unigrams, bigrams e trigrams
\item Quantidade de n-grams extraídos do documento
\item Tamanho do documento (contabiliza todos os n-grams existentes no documento)
\end{itemize}

\item Características de soma %31
\begin{itemize}
\item Soma dos adjetivos positivos
\item Soma dos adjetivos negativos
\item Soma dos advérbios positivos
\item Soma dos advérbios negativos
\item Soma normalizada dos adjetivos positivos
\item Soma normalizada dos advérbios positivos
\item Soma normalizada dos adjetivos negativos
\item Soma normalizada dos advérbios negativos
\item Diferença entre a soma dos adjetivos positivos e negativos
\item Diferença entre a soma dos advérbios positivos e negativos
\item Soma dos adjetivos e dos bigrams positivos formados por advérbio e adjetivo
\item Soma dos adjetivos e dos bigrams negativos formados por advérbio e adjetivo
\item Soma dos advérbios e dos bigrams positivos somente formados por advérbios
\item Soma dos advérbios e dos bigrams negativos somente formados por advérbios
\item Soma dos unigrams e bigrams positivos
\item Soma dos unigrams e bigrams negativos
\item Soma dos unigrams, bigrams e trigrams positivos
\item Soma dos unigrams, bigrams e trigrams negativos
\item Soma normalizada dos adjetivos e dos bigrams positivos formados por advérbio e adjetivo
\item Soma normalizada dos adjetivos e dos bigrams negativos formados por advérbio e adjetivo
\item Soma normalizada dos advérbios e dos bigrams positivos somente formados por advérbios
\item Soma normalizada dos advérbios e dos bigrams negativos somente formados por advérbios
\item Soma normalizada dos unigrams e bigrams positivos
\item Soma normalizada dos unigrams e bigrams negativos
\item Soma normalizada dos unigrams, bigrams e trigrams positivos
\item Soma normalizada dos unigrams, bigrams e trigrams negativos
\item Diferença entre a soma positiva e negativa dos adjetivos e dos bigrams formados por advérbio e adjetivo
\item Diferença entre a soma positiva e negativa dos advérbios e dos bigrams formados somente por advérbios
\item Diferença entre a soma positiva e negativa dos unigrams e bigrams
\item Diferença entre a soma positiva e negativa dos unigrams, bigrams e trigrams
\item Percentual de n-grams negados
\end{itemize}

\item Características de regra máxima %6
\begin{itemize}
\item Classe da polaridade máxima entre os adjetivos de um documento (1 se o maior valor absoluto entre as polaridades for positivo ou -1 se for negativo)
\item Classe da polaridade máxima entre os advérbios de um documento (1 se o maior valor absoluto entre as polaridades for positivo ou -1 se for negativo)
\item Classe da polaridade máxima entre adjetivos e bigrams formados por advérbio e adjetivo (1 se o maior valor absoluto entre as polaridades for positivo ou -1 se for negativo)
\item Classe da polaridade máxima entre advérbios e bigrams formados somente por advérbios (1 se o maior valor absoluto entre as polaridades for positivo ou -1 se for negativo)
\item Classe da polaridade máxima entre unigrams e bigrams de um documento (1 se o maior valor absoluto entre as polaridades for positivo ou -1 se for negativo)
\item Classe da polaridade máxima entre unigrams, bigrams e trigrams de um documento (1 se o maior valor absoluto entre as polaridades for positivo ou -1 se for negativo)
\end{itemize}
\end{itemize}
% \include{apendice2}
% ...
% \include{apendiceM}


% Bibliografia
% É aconselhável utilizar o BibTeX a partir de um arquivo, digamos "biblio.bib".
% Para ajuda na criação do arquivo .bib e utilização do BibTeX, recorra ao
% BibTeXpress em www.cin.ufpe.br/~paguso/bibtexpress
\bibliographystyle{abnt-alf}
\bibliography{../pesquisa}

%% Fim do documento
\end{document}